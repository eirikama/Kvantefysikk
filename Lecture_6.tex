\documentclass{article}

\usepackage{standalone}
\usepackage{pgfplots}
\pgfplotsset{compat=newest}
\usepackage{tikz}
\usetikzlibrary{decorations.markings}

\usepackage{subfig}
\usepackage[margin=2.5cm]{geometry}
\usepackage{amsmath}
\usepackage{amssymb}


\newcommand\greybox[1]{
	\vskip\baselineskip
	\par\noindent\colorbox{lightgray}{
		\begin{minipage}{\textwidth}#1\end{minipage}
	}
	\vskip\baselineskip
}

\title{Kvantemekanikk i 3 dimensjoner}


\begin{document}
\maketitle

I det følgende skal vi betrakte tre-dimensjonale kvantemekaniske systemer.
Mye er nesten 1-til-1 analogt med en-dimensjonale potensialer, men det dukker også 
opp helt nye konsepter i 3D som drivmoment.

\section*{Partikkel i 3D boks}

Vi starter med å se på partikkel i en tre-dimensjonal boks, dette kan beregnes helt analogt med 
i en 1D. Vi har nå at Hamilton operatoren som 
\begin{align}
	 \hat{H} &= \frac{ \mathbf{ \hat{p}}^2}{2m} + V( \mathbf{ \hat{x}}) \\
	  &= \frac{ \hat{p}_x^2 + \hat{p}_y^2 + \hat{p}_z^2}{2m} + V(x,y,z) \\
	  &= -\frac{\hbar^2}{2m} ( \frac{\partial^2}{\partial x^2} + 
		\frac{\partial^2}{\partial y^2} + \frac{\partial^2}{\partial z^2} ) + V(x,y,z)
\end{align}
hvor vi har at energien er summen av kinetisk og potensiell energi og at bevegelsesmengden kan 
dekomponeres i bevegelsesmengden i $x$, $y$ og $z$ retning.
Vi får da T.U.S.L. som 
\begin{align}
	\Big(-\frac{\hbar^2}{2m} ( \frac{\partial^2}{\partial x^2} + 
 \frac{\partial^2}{\partial y^2} + \frac{\partial^2}{\partial z^2} ) + V(x,y,z)\Big) \psi(x,y,z) 
 = E \psi(x,y,z)
\end{align}
en partiell differensialligning. Vi har nå at 
\begin{align}
	 V(x,y,z) = \begin{cases}
		0 \hspace{1.2cm} x,y,z \in (0, L) \\
		\infty \hspace{1cm} x,y,z \notin (0,L)
		\end{cases}
\end{align}
Vi kan dekomponere dette probleme ved hjelp av separasjon av variable og følgende Ansatz 
\begin{align}
	 \psi(x,y,z) = X(x)Y(y)Z(z)
\end{align}
slik at vi får 3 ordinære differensialligninger istedenfor den mer vriene-å-løse partielle 
differensialligningen. Vi setter inn for Ansatzen vår i T.U.SL. og finner
\begin{align}
	-\frac{\hbar^2}{2m} ( YZ\frac{\partial^2}{\partial x^2} X + 
   XZ\frac{\partial^2}{\partial y^2}Y +  XY\frac{\partial^2}{\partial z^2} Z) 
  + V(x,y,z) XYZ 
 = E XYZ \\
\Rightarrow	-\frac{\hbar^2}{2m} ( \frac{1}{X}\frac{\partial^2}{\partial x^2} X + 
   \frac{1}{Y}\frac{\partial^2}{\partial y^2}Y +  \frac{1}{Z}\frac{\partial^2}{\partial z^2} Z) 
  + V(x,y,z)  = E
\end{align}
Vi får nå tre ordinære differensialligninger siden vi har at inne i boksen hvor $V(x)=0$ og må ha 
at hvert av leddene på venstre side må være lik en konstant og vi får da 
\begin{align}
	-\frac{\hbar^2}{2m} \frac{\partial^2}{\partial x^2} X(x)  &= E_xX(x) \\
	-\frac{\hbar^2}{2m} \frac{\partial^2}{\partial y^2} Y(y)  &= E_yY(y) \\
	-\frac{\hbar^2}{2m} \frac{\partial^2}{\partial z^2} Z(z)  &= E_zZ(z)
\end{align}
hvor $E=E_x+E_y+E_z$.
Dette er for hver komponent helt identisk med det en-dimensjonale problemet og vi får da 
\begin{align}
	 \psi(x,y,z) = \Big( \frac{2}{L}\Big)^{ \frac{3}{2}}
	\sin( \frac{n_x \pi x}{L}) \sin( \frac{n_y\pi y}{L})\sin( \frac{n_z \pi z}{L})
	\hspace{1cm} x,y,z \in (0,L)
\end{align}
og vi har energien som 
\begin{align}
	 E &= E_{n_x} + E_{n_y} + E_{n_z} \\
	   &= \frac{(n_x^2 + n_y^2 + n_z^2)\hbar^2 \pi^2}{2mL^2}
\end{align}







\section*{Drivmoment}

Vi ser nå på systemer som er sfærisk symmetrisk. Noen potensialer er ikke separable
i Cartesiske koordinater, men det viser seg at de ofte er separable i sfæriske koordinater 
om vi antar at $V=V(r)$. Dvs. sfærisk symmetriske potensialer er separable i sfæriske koordinater.
Vi starter med å betrakte bevegelsesmengde operatoren i 3D i Cartesiske koordinater 
\begin{align}
	 \mathbf{ \hat{p}} = \frac{\hbar}{i} 
	(  \frac{\partial}{\partial x} \mathbf{i}_x
	+  \frac{\partial}{\partial y} \mathbf{i}_y
	+  \frac{\partial}{\partial z} \mathbf{i}_z ) = \frac{\hbar}{i} \nabla
\end{align}
og vi har 
\begin{align}
	 \mathbf{ \hat{p}^2} = -\hbar^2 ( \frac{\partial^2}{\partial x^2} +
		\frac{\partial^2}{\partial y^2} + \frac{\partial^2}{\partial z^2}  ) 
		= -\hbar \nabla^2
\end{align}
hvor 
\begin{align}
	 \nabla^2 = \frac{\partial^2}{\partial x^2} +
		\frac{\partial^2}{\partial y^2} + \frac{\partial^2}{\partial z^2}  
\end{align}
kalles Laplace operatoren.
I sfæriske koordinater har vi Laplace operatoren som 
\begin{align}
 \nabla^2 = \frac{1}{r^2} \frac{\partial }{\partial r} \Big(r^2 \frac{\partial}{\partial r} \Big)
   + \frac{1}{r^2 \sin(\theta)} \frac{\partial}{\partial \theta}
       \Big(\sin(\theta) \frac{\partial}{\partial \theta}\Big)
  + \frac{1}{r^2 \sin(\theta)} \frac{\partial^2}{\partial \phi^2} 
\end{align}
Vi skal i det følgende betrakte drivmoment til en kvantemakanisk partikkel.
Vi har fra klassisk fysikk at 
\begin{align}
	 \mathbf{L} = \mathbf{r} \times \mathbf{p}
\end{align}
Som vi kan betrakte komponent for komponent i Cartesiske koordinater
\begin{align}
	 L_y &= y p_z - z p_y \\
	 L_y &= z p_x - x p_z \\
	 L_z &= x p_y - y p_x 
\end{align}
Som i kvantemakanikken da blir, når vi opphøyer klassiske variabler til kvantemakaniske operatorer
\begin{align}
	 \hat{L}_x &= \hat{y} \hat{p}_z - \hat{z} \hat{p_y} =
	  y\frac{\hbar}{i} \frac{\partial}{\partial z} - z\frac{\hbar}{i} \frac{\partial}{\partial y}    \\
	 \hat{L}_y &= \hat{z} \hat{p}_x - \hat{x} \hat{p}_z =
	 z \frac{\hbar}{i} \frac{\partial}{\partial x}  - x\frac{\hbar}{i} \frac{\partial}{\partial z} \\
	 \hat{L}_z &= \hat{x} \hat{p}_y - \hat{y} \hat{p}_x = 
	  x\frac{\hbar}{i}  \frac{\partial }{\partial y}  - y \frac{\hbar}{i} \frac{\partial}{\partial x} 
\end{align}
Siden vi nå vil beskrive angulær bevegelsesmengde så er det enklest å bruke sfæriske koordinater.
Vi har at relasjonen mellom sfæriske og Cartesiske koordinater er 
\begin{align}
	 x &= r \sin(\theta) \cos(\phi) \\
	 y &= r \sin(\theta) \sin(\phi) \\
	 z &= r \cos(\theta)  
\end{align}
Dette gir 
\begin{align}
	 \frac{\partial x}{\partial \phi} &= -y \\ 
	 \frac{\partial y}{\partial \phi} &= x \\ 
	 \frac{\partial z}{\partial \phi} &= 0 \\ 
\end{align}
Dette gir videre at 
\begin{align}
	  \frac{\partial }{\partial \phi} &= 
	\frac{\partial x}{\partial \phi} \frac{\partial}{\partial x}  +
 	 \frac{\partial y }{\partial \phi} \frac{\partial}{\partial y} 
 	+ \frac{\partial z}{\partial \phi} \frac{\partial}{\partial z} \\
	&= -y \frac{\partial}{\partial x}  + x \frac{\partial}{\partial y} 
\end{align}
Dette gir da at z-komponenten av drivmoment operatoren er forholdsvis enkel 
\begin{align}
	 \hat{L}_z = \frac{h}{i}\frac{\partial}{\partial \phi} 
\end{align}
siden rotasjon om z-aksen bare krever forrandring i azimutal vinkelen.
Rotasjon om x- eller y-aksen krevere endring i både azimutal og polar vinkel og er derfor noe mer 
kompleks.
Vi har at 
\begin{align}
	 \hat{L}_x &= \frac{\hbar }{i} \big( -\sin(\phi) \frac{\partial}{\partial \theta}
	-cot(\theta) \cos(\phi) \frac{\partial}{\partial \phi} \big) \\
	 \hat{L}_y &= \frac{\hbar}{i}\big( \cos(\phi) \frac{\partial }{\partial \theta} 
	-cot(\theta) \sin(\phi) \frac{\partial}{\partial \phi} \big)
\end{align}
Vi har da 
\begin{align}
	 \mathbf{ \hat{L}}^2 &= \hat{L}_x^2 + \hat{L}_y^2 + \hat{L}_z^2 \\
	&= -\hbar^2 \Big[  \frac{1}{\sin(\theta)} \frac{\partial}{\partial \theta} 
		\big(\sin(\theta) \frac{\partial}{\partial \theta}\big) +
	\frac{1}{\sin^2(\theta)} \frac{\partial^2}{\partial \phi^2} \Big]
\end{align}
Vi vil nå finne eigenfunksjonene til $ \mathbf{ \hat{L}}^2 $. Vi starter med å merke oss at 
\begin{align}
	[ \mathbf{\hat{L}}^2, \hat{L}_z] = 0 	 
\end{align}
som vi ser må være tilfellet siden $ \hat{L}_z $ deriverer mhp. $ \phi $ og $ \mathbf{ \hat{L}}^2 $ 
ikke har noen direkte $ \phi $ avhengighet annen enn andrederivert-operatorenm og vi vet at rekkefølgen 
av partiell-deriverte er likegyldig hvilket gir oss at  $ \mathbf{\hat{L}}^2$ og $\hat{L}_z$ kommuterer.
Det viser seg at vi også har 
\begin{align}
	[ \mathbf{\hat{L}}^2, \hat{L}_x] = [ \mathbf{\hat{L}}^2, \hat{L}_y] = 0 	 
\end{align}
men dette er litt emr infløkt å vise; man må regne ut kommutatoren for å se det.

Dette betyr da at $ \mathbf{ \hat{L}}^2$ og $ \hat{L}_z $ har et komplett sett av simultane 
eigenfunksjoner. Dette betyr at 
vi kan beregne eigenfunksjonene til $ \hat{L}_z $ identisk til $ \mathbf{ \hat{L}}^2 $ 
som formodentlig er enklere.

Vi har benytter oss av seperasjon av variabler og skriver $ Y(\theta, \phi) = \Theta(\theta) \Phi(\phi) $ 
som er eigenfunksjonen til både $ \mathbf{ \hat{L}}^2 $ og $ \hat{L}_z $ slik at
\begin{align}
	\hat{L}_z Y(\theta, \phi)  &= L_z Y(\theta, \phi)	 \\
	\hat{L}_z \Theta(\theta) \Phi(\phi)  &= L_z \Theta(\theta) \Phi(\phi)	 
\end{align}
Slik at vi får
\begin{align}
 \hat{L}_z Y(\theta, \phi)  &= L_z Y(\theta, \phi)	 \\
  \Rightarrow \frac{\hbar}{i} \frac{\partial}{\partial \phi} \Theta(\theta)\Phi(\phi)  
			    &= L_z \Theta(\theta)\Phi(\phi)\\
 \Rightarrow \Theta(\theta) \frac{\hbar}{i} \frac{\partial}{\partial \phi} \Phi(\phi)  
			    &= \Theta(\theta) L_z\Phi(\phi) \\
  \Rightarrow \frac{\hbar}{i} \frac{\partial}{\partial \phi} \Phi(\phi)  &= L_z \Phi(\phi)
\end{align}
Som gir at 
\begin{align}
	 \Phi(\phi) = e^{i\frac{L_z\phi}{\hbar}}
\end{align}
hvor vi ser bort i fra konstanten som burde ha kommet fra løsningen i differensialligningen, vi setter 
heller konstanten til slutt når vi normaliserer $Y(\theta, \phi)$.
hvor vi ser bort i fra konstanten som burde ha kommet fra løsningen i differensialligningen, vi setter 
heller konstanten til slutt når vi normaliserer $Y(\theta, \phi)$.
Dette er da en eigenfunksjon for $ \hat{L}_z $,
siden vi kan ha $\Theta(\theta)$ som hva som helst, e.g. en konstant.

Dette uttrykket ligener umiskinnelig på uttrykket for bevegelsesmengdens 
eigenfunksjon. Men det er en stor forskjell, siden vi her trenger å ha at når $ \phi $ forrandrer seg 
med  $2\pi$ så må vi komme tilbake til det samme stedet. Vi trenger derfor å ha at 
\begin{align}
	 \Phi(\phi) = \Phi(\phi + 2\pi)
\end{align}
Siden om dette ikke er tilfellet og vi har e.g. $ \Phi(0) \neq \Phi(2\pi) $ så er ikke den deriverte 
definert ved $ \phi = 0 $. Dette er ekvivalent med at den deriverte mhp. x (og derfor bevegelsesmengden) 
er uendelig om bølgefunksjonen ikke er kontinuerlig.
Dette gir da 
\begin{align}
	 e^{i \frac{L_z(\phi + 2\pi)}{\hbar}} &=  e^{i \frac{L_z\phi}{\hbar}} \\
	\Rightarrow e^{i \frac{2\pi L_z}{\hbar}} &=  1
\end{align}
Hvor vi vet at vi har at $e^{ix}=1$ når $x$ er heltallsmultiplum av $\pi$ slik at vi får 
\begin{align}
	 L_z = m_l \hbar \hspace{1cm} \forall m_l \in \mathbb{N}
\end{align}
Dvs vi har alltid at z-komponenten av drivmomentet er kvantisert, og de tillatte verdiene er 
heltallsmultiplum av $\hbar$.

Vi vil nå bestemme $ \theta $ avhengigheten til $ Y(\theta, \phi) $ og vi har eigenverdiligningen
\begin{align}
	\mathbf{ \hat{L}}^2 Y(\theta, \phi) = l(l+1) \hbar^2 Y(\theta, \phi) 
\end{align}
hvor vi faktorerer ut en faktor av $\hbar^2$ av eigenverdien, slik at $ l(l+1) $ 
er dimensjonsløs. Vi taper ingen generalitet på å uttrykke eigenverdien på denne måten,
og vi skal snart se hvorfor vi velger å skrive den på denne formen. Dette gir 
\begin{align}
	-\hbar^2 \Big[  \frac{1}{\sin(\theta)} \frac{\partial}{\partial \theta} 
		\big(\sin(\theta) \frac{\partial}{\partial \theta}\big) +
	\frac{1}{\sin^2(\theta)} \frac{\partial^2}{\partial \phi^2} \Big] \Theta(\theta)\Phi(\phi) 
	&= l(l+1) \hbar^2 \Theta(\theta) \Phi(\phi)  \\
	\Rightarrow -\hbar^2 \Big[  \frac{1}{\sin(\theta)} \frac{\partial}{\partial \theta} 
		\big(\sin(\theta) \frac{\partial}{\partial \theta}\big) +
	\frac{1}{\sin^2(\theta)} \frac{\partial^2}{\partial \phi^2} \Big] \Theta(\theta) e^{i\phi m_l}
	&= l(l+1) \hbar^2 \Theta(\theta)  e^{i\phi m_l}  \\
	\Rightarrow -\hbar^2 \Big[  \frac{1}{\sin(\theta)} \frac{\partial}{\partial \theta} 
		\big(\sin(\theta) \frac{\partial}{\partial \theta}\big) -
	\frac{m_l^2}{\sin^2(\theta)}  \Big] \Theta(\theta) 
	&= l(l+1) \hbar^2 \Theta(\theta)  \\
	\Rightarrow \Big[  \sin(\theta) \frac{\partial}{\partial \theta} 
		\big(\sin(\theta) \frac{\partial}{\partial \theta}\big)  + l(l+1) \sin^2(\theta) -
	m_l^2  \Big] \Theta(\theta) 
	&=  0
\end{align}
Dette er Legendre's ligning for argument $\cos(\theta)$ og løsningene til dette er 
\begin{align}
	 \Theta(\theta) = N P_l ^{m_l}(\cos(\theta)) 
\end{align}
assosierte Legendre funksjonene 
\begin{align}
	 P^{m_l}_l(x) =  (1-x^2)^{ \frac{|m_l|}{2}} \big(\frac{\partial}{\partial x} \big)^{|m_l|} P_l(x)
\end{align}
hvor $P_l(x)$ igjen er Legendre polynomene gitt e.g. ved Rodrigues formel som sier at
\begin{align}
	 P_l(x) = \frac{1}{2^l l! } \big( \frac{\partial}{\partial x} \big)^{l} (x^2 - 1)^l
\end{align}
Det $l$'te Legendre polynomet $ P_l(x) $  er et $l$'te ordens polynom som er enten jevnt eller 
odde avhengig av pariteten til $l$. Men $ P^m_l_l(x) $ er ikke generelt et polynom siden vi får en faktor 
$ \sqrt{1-x^2} $ når $m_l$ er odde. 
Vi ser videre at når $m_l < l$ så får vi at $P^{m_l}_l(x) = 0$ og vi ser at Rodrigues formel kun 
gir mening for ikke-negative heltall.
\begin{align}
	l &= 0,1,2,3,\ldots \\
	m_l &= -l, -l+1, \ldots, -1, 0, 1, \ldots , l-1, l
\end{align}
Det finnes dog løsninger for andre verdier av $m_l$ og $l$ av Legendre's ligning, men det viser seg at
de løsningen da ikke er endelige ved $\theta=0$ og $ \theta = \pi $. Disse løsningene er derfor 
ikke normaliserbar og følgelig ikke fysikalske, og vi må forkaste de.
Vi har da at løsningene er $Y_l^{m_l}(\theta, \phi) =N P_l^{m_l}(\theta)e^{im_l\phi}$ som vi kan 
bestemme konstanten til vha normaliseringsbetingelsen. Vi finner da de sfæriske harmoniske er 
\begin{align}
 Y^{m_l}_l = (-1)^{m_l} \sqrt{ \frac{(2l + 1)}{4\pi} \frac{(l-m)!}{(l+m)!}} 
P^{m_l}_l(\cos(\theta))e^{im_l\phi}
\end{align}
Vi har at de sfæriske harmoniske er ortonormale 
\begin{align}
 \int_{0}^{2\pi} \int_{0}^{\pi} (Y^{m_l'}_l')^*Y^{m_l}_l \sin(\theta) d\theta d\phi 
= \delta_{m_l, m_l'} \delta_{l, l'}
\end{align}
Slik at vi har både eigenverdienen til kvadratet og z-komponenten til drivmomentet som kvantisert.
Vi skriver de sfæriske harmoniske som $ Y^{m_l}_l(\theta, \phi) $ og eigenverdiligningene 
\begin{align}
	 \hat{L}_zY^{m_l}_l(\theta, \phi) &= m_l\hbar  Y^{m_l}_l(\theta, \phi)
\hspace{1.6cm} m_l = 0, \pm 1, \pm 2, \ldots \\
	 \mathbf{\hat{L}}^2 Y^{m_l}_l(\theta, \phi) &= l(l+1)\hbar^2 Y^{m_l}_l(\theta, \phi)
\hspace{1cm} l = 0, 1, 2, 3, \ldots
\end{align}
Vi ser på absoluttverdien til drivmomentet 
\begin{align}
	 | \mathbf{L}|= \sqrt{l(l+1)}\hbar 
\end{align}
og merker oss at dette alltid er strengt større enn den maksimale projeksjonen av drivmomentet på 
z-aksen
\begin{align}
	L_{z, max} = l \hbar 
\end{align}
Siden $m_l$ alltid er mindre enn eller lik $l$. Hvilket vil si at vi aldri kan legge hele drivmomentet, 
som er en vektor, stringent i z-retning. 

Vi kan forstå dette ved å betrakte kommutatoren og uskarphetsrelasjonene mellom drivmoment-operatorene 
i de forskjellige retningene. Vi har e.g. 
\begin{align}
[ \hat{L}_x, \hat{L}_y] 
&= [ y\frac{\hbar}{i} \frac{\partial}{\partial z} - z \frac{\hbar}{i}\frac{\partial}{\partial y}  
 , \; z\frac{\hbar}{i} \frac{\partial}{\partial x} - x \frac{\hbar}{i}\frac{\partial}{\partial z}] \\
&= [ y\frac{\hbar}{i} \frac{\partial}{\partial z} , z \frac{\hbar}{i}\frac{\partial}{\partial x}]  
 +  [z\frac{\hbar}{i} \frac{\partial}{\partial y}, x \frac{\hbar}{i}\frac{\partial}{\partial z}] \\
&= y[\frac{\hbar}{i} \frac{\partial}{\partial z} , z]  \frac{\hbar}{i}\frac{\partial}{\partial x} 
 +  \frac{\hbar}{i} \frac{\partial}{\partial y}[z,  \frac{\hbar}{i}\frac{\partial}{\partial z}]x \\
&= y[ \hat{p}_z, \hat{z}]  \frac{\hbar}{i}\frac{\partial}{\partial x} 
 +  \frac{\hbar}{i} \frac{\partial}{\partial y}[ \hat{z}, \hat{p}_z]x \\
&= -i\hbar y \frac{\hbar}{i}\frac{\partial}{\partial x} 
 + \frac{\hbar}{i} \frac{\partial}{\partial y}x i\hbar \\
&= i\hbar (-y \frac{\hbar}{i}\frac{\partial}{\partial x} 
 + \frac{\hbar}{i} \frac{\partial}{\partial y}x) \\ 
&= i\hbar \hat{L}_z
\end{align}
hvor vi har brukt kommutator-identiten 
\begin{align}
[ \hat{A} + \hat{B}, \hat{C} + \hat{D}] = [ \hat{A}, \hat{C}] + [ \hat{A}, \hat{D}] 
+[ \hat{B}, \hat{C}]  + [ \hat{B}, \hat{D}] 
\end{align}
og at hvert av leddene som deriverer med hensyn på en variabel kommutere med ledd som ikke 
avhenger av den variabelen.

Vi ser da at x- og y-komponenten til drivmomentet ikke kommuterer og vi vet da at vi har 
en uskarphetsrelasjonen mellom de. Vi får 
\begin{align}
	 \Delta L_x \Delta L_y \ge \frac{\hbar}{2}| \langle L_z  \rangle |
\end{align}
Det vil si at om vi har en partikkel i en eigentilstand av $ \hat{L}_z $ med eigenverdi $m_l \hbar \neq 0$
så følger det at z-komponenten til partikkel er kjent med sikkerhet, men siden $ \langle L_z  \rangle $ 
så er $ \Delta L_x $ og $ \Delta L_y $ begge ulik 0. Dette betyr at det er en iboende uskarphet i $L_x$ og
$ L_y $ og vi ser at vi i kvantemekanikk ikke har et begrep om at drivmomentet er en vektor i rommet som 
peker i en bestemt retning, siden det impliserer at vi kjenner $ L_x $, $ L_y $ og $ L_z $ skarpt 
samtidig.

Vi betrakter dette i kontrast til lineær bevegelsesmengde hvor vi har 
\begin{align}
 [ \hat{p}_x, \hat{p}_y]
 = [ \frac{\hbar}{i} \frac{\partial}{\partial x} , \frac{\hbar}{i} \frac{\partial V}{\partial x} ] 
 = 0
\end{align}
slik at vi kan bestemme de forskjellige komponentene til bevegelsesmengden skarpt samtidig.

\section*{Hydrogenatomet}
Vi skal nå se og regne på Hydrogenatomet. Vi vet at vi har et elektristk påtensiale mellom protonet og 
elektronet i Hydrogen atomet, så vi anntar derfor et Coulomb potensialet som vi har som 
\begin{align}
	 V_C(r) = - \frac{Ze^2}{4\pi \epsilon_0 r}
\end{align}
vi vet er sentralsymmetrisk. Vi har at $e$ er elementærladningen altså ladningen til elektronet, 
$\epsilon_0$ er vakuumpermittiviteten og $Z$ er atomtallet som for Hydrogenatomet må være $Z=1$, men 
vi kan også uten videre betrakte ionisert Helium med $Z=2$.
Vi har da T.U.S.L. som
\begin{align}
	 E \psi &= \hat{H}\psi \\ 
		&= (-\frac{\hbar^2}{2m} \nabla^2 + V_C(r)) \psi  \\
	 &= -\frac{\hbar^2}{2m}\Big(\frac{1}{r^2} \frac{\partial }{\partial r} 
	\Big(r^2 \frac{\partial}{\partial r} \Big)
   + \frac{1}{r^2 \sin(\theta)} \frac{\partial}{\partial \theta}
     \Big(\sin(\theta) \frac{\partial}{\partial \theta}\Big)
     + \frac{1}{r^2 \sin(\theta)} \frac{\partial^2}{\partial \phi^2} \Big)\psi
    - \frac{Ze^2}{4\pi \epsilon_0 r}\psi  \\
 &= -\frac{\hbar^2}{2m}\Big(\frac{1}{r^2} \frac{\partial }{\partial r} 
	\Big(r^2 \frac{\partial}{\partial r} \Big)
   + \frac{1}{r^2} \mathbf{ \hat{L}}^2 \Big)\psi
    - \frac{Ze^2}{4\pi \epsilon_0 r}\psi 
\end{align}
hvor vi eksplisit ser at all vinkel-avhengigheten til Hamilton-operatoren er gitt i $ \mathbf{ \hat{L}}^2$
som gjør at vi får følgende Ansatz
\begin{align}
	 \psi = R(r)Y_l^{m_l}(\theta, \phi)
\end{align}
Dette gir ved innsettelse i T.U.S.L 
\begin{align}
  ER(r) Y_l^{m_l}&= -\frac{\hbar^2}{2m} \frac{1}{r^2} \frac{\partial }{\partial r} 
	\Big(r^2 \frac{\partial}{\partial r} \Big)R(r) Y_l^{m_l}
   + \frac{1}{2mr^2} \mathbf{ \hat{L}}^2 R(r) Y_l^{m_l}
    - \frac{Ze^2}{4\pi \epsilon_0 r}R(r) Y_l^{m_l} \\
  \Rightarrow ER(r) &= -\frac{\hbar^2}{2m}\frac{1}{r^2} \frac{\partial }{\partial r} 
	\Big(r^2 \frac{\partial}{\partial r} \Big) R(r)
   + \frac{l(l+1)\hbar^2}{2mr^2} R(r)
    - \frac{Ze^2}{4\pi \epsilon_0 r}R(r) 
\end{align}
Vi gjennomfører et variabel bytte for å forenkle dette litt 
\begin{align}
	 u(r) = rR(r)
\end{align}
Dette gir 
\begin{align}
   Eu(r) &= -\frac{\hbar^2}{2m}\frac{\partial^2 }{\partial r^2} u(r)
   + \Big(\frac{l(l+1)\hbar^2}{2mr^2}  - \frac{Ze^2}{4\pi \epsilon_0 r}\Big)u(r) 
\end{align}
som vi merker at er lik den en-dimensjonale S.L. men med et nytt effektivt potensiale
\begin{align}
	 V_{eff} = \frac{l(l+1)\hbar^2}{2mr^2}  - \frac{Ze^2}{4\pi \epsilon_0 r}
\end{align}
Vi har her at den første termen er et sentrifugal ledd, og den kalles ofte sentrifugal barrieren. 
Dette leddet holder for $l>0$ elektronet vekke ifra protonet.

 \end{document}

