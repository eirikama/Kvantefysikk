\documentclass{article}

\usepackage{standalone}
\usepackage{pgfplots}
\pgfplotsset{compat=newest}
\usepackage{tikz}
\usetikzlibrary{decorations.markings}

\usepackage{subfig}
\usepackage[margin=2.5cm]{geometry}
\usepackage{amsmath}
\usepackage{amssymb}


\newcommand\greybox[1]{
	\vskip\baselineskip
	\par\noindent\colorbox{lightgray}{
		\begin{minipage}{\textwidth}#1\end{minipage}
	}
	\vskip\baselineskip
}

\title{Kvantemekanikk i 3 dimensjoner}


\begin{document}
\maketitle

I det følgende skal vi betrakte tre-dimensjonale kvantemekaniske systemer.
Mye er nesten 1-til-1 analogt med en-dimensjonale potensialer, men det dukker også 
opp helt nye konsepter i 3D som drivmoment.

\section*{Partikkel i 3D boks}

Vi starter med å se på partikkel i en tre-dimensjonal boks, dette kan beregnes helt analogt med 
i en 1D. Vi har nå at Hamilton operatoren som 
\begin{align}
	 \hat{H} &= \frac{ \mathbf{ \hat{p}}^2}{2m} + V( \mathbf{ \hat{x}}) \\
	  &= \frac{ \hat{p}_x^2 + \hat{p}_y^2 + \hat{p}_z^2}{2m} + V(x,y,z) \\
	  &= -\frac{\hbar^2}{2m} ( \frac{\partial^2}{\partial x^2} + 
		\frac{\partial^2}{\partial y^2} + \frac{\partial^2}{\partial z^2} ) + V(x,y,z)
\end{align}
hvor vi har at energien er summen av kinetisk og potensiell energi og at bevegelsesmengden kan 
dekomponeres i bevegelsesmengden i $x$, $y$ og $z$ retning.
Vi får da T.U.S.L. som 
\begin{align}
	\Big(-\frac{\hbar^2}{2m} ( \frac{\partial^2}{\partial x^2} + 
 \frac{\partial^2}{\partial y^2} + \frac{\partial^2}{\partial z^2} ) + V(x,y,z)\Big) \psi(x,y,z) 
 = E \psi(x,y,z)
\end{align}
en partiell differensialligning. Vi har nå at 
\begin{align}
	 V(x,y,z) = \begin{cases}
		0 \hspace{1.2cm} x,y,z \in (0, L) \\
		\infty \hspace{1cm} x,y,z \notin (0,L)
		\end{cases}
\end{align}
Vi kan dekomponere dette probleme ved hjelp av separasjon av variable og følgende Ansatz 
\begin{align}
	 \psi(x,y,z) = X(x)Y(y)Z(z)
\end{align}
slik at vi får 3 ordinære differensialligninger istedenfor den mer vriene-å-løse partielle 
differensialligningen. Vi setter inn for Ansatzen vår i T.U.SL. og finner
\begin{align}
	-\frac{\hbar^2}{2m} ( YZ\frac{\partial^2}{\partial x^2} X + 
   XZ\frac{\partial^2}{\partial y^2}Y +  XY\frac{\partial^2}{\partial z^2} Z) 
  + V(x,y,z) XYZ 
 = E XYZ \\
\Rightarrow	-\frac{\hbar^2}{2m} ( \frac{1}{X}\frac{\partial^2}{\partial x^2} X + 
   \frac{1}{Y}\frac{\partial^2}{\partial y^2}Y +  \frac{1}{Z}\frac{\partial^2}{\partial z^2} Z) 
  + V(x,y,z)  = E
\end{align}
Vi får nå tre ordinære differensialligninger siden vi har at inne i boksen hvor $V(x)=0$ og må ha 
at hvert av leddene på venstre side må være lik en konstant og vi får da 
\begin{align}
	-\frac{\hbar^2}{2m} \frac{\partial^2}{\partial x^2} X(x)  &= E_xX(x) \\
	-\frac{\hbar^2}{2m} \frac{\partial^2}{\partial y^2} Y(y)  &= E_yY(y) \\
	-\frac{\hbar^2}{2m} \frac{\partial^2}{\partial z^2} Z(z)  &= E_zZ(z)
\end{align}
hvor $E=E_x+E_y+E_z$.
Dette er for hver komponent helt identisk med det en-dimensjonale problemet og vi får da 
\begin{align}
	 \psi(x,y,z) = \Big( \frac{2}{L}\Big)^{ \frac{3}{2}}
	\sin( \frac{n_x \pi x}{L}) \sin( \frac{n_y\pi y}{L})\sin( \frac{n_z \pi z}{L})
	\hspace{1cm} x,y,z \in (0,L)
\end{align}
og vi har energien som 
\begin{align}
	 E &= E_{n_x} + E_{n_y} + E_{n_z} \\
	   &= \frac{(n_x^2 + n_y^2 + n_z^2)\hbar^2 \pi^2}{2mL^2}
\end{align}







\section*{Drivmoment}

Vi ser nå på systemer som er sfærisk symmetrisk. Noen potensialer er ikke separable
i Cartesiske koordinater, men det viser seg at de ofte er separable i sfæriske koordinater 
om vi antar at $V=V(r)$. Dvs. sfærisk symmetriske potensialer er separable i sfæriske koordinater.
Vi starter med å betrakte bevegelsesmengde operatoren i 3D i Cartesiske koordinater 
\begin{align}
	 \mathbf{ \hat{p}} = \frac{\hbar}{i} 
	(  \frac{\partial}{\partial x} \mathbf{i}_x
	+  \frac{\partial}{\partial y} \mathbf{i}_y
	+  \frac{\partial}{\partial z} \mathbf{i}_z ) = \frac{\hbar}{i} \nabla
\end{align}
og vi har 
\begin{align}
	 \mathbf{ \hat{p}^2} = -\hbar^2 ( \frac{\partial^2}{\partial x^2} +
		\frac{\partial^2}{\partial y^2} + \frac{\partial^2}{\partial z^2}  ) 
		= -\hbar \nabla^2
\end{align}
hvor 
\begin{align}
	 \nabla^2 = \frac{\partial^2}{\partial x^2} +
		\frac{\partial^2}{\partial y^2} + \frac{\partial^2}{\partial z^2}  
\end{align}
kalles Laplace operatoren.
I sfæriske koordinater har vi Laplace operatoren som 
\begin{align}
 \nabla^2 = \frac{1}{r^2} \frac{\partial }{\partial r} \Big(r^2 \frac{\partial}{\partial r} \Big)
   + \frac{1}{r^2 \sin(\theta)} \frac{\partial}{\partial \theta}
       \Big(\sin(\theta) \frac{\partial}{\partial \theta}\Big)
  + \frac{1}{r^2 \sin(\theta)} \frac{\partial^2}{\partial \phi^2} 
\end{align}
Vi skal i det følgende betrakte drivmoment til en kvantemakanisk partikkel.
Vi har fra klassisk fysikk at 
\begin{align}
	 \mathbf{L} = \mathbf{r} \times \mathbf{p}
\end{align}
Som vi kan betrakte komponent for komponent i Cartesiske koordinater
\begin{align}
	 L_y &= y p_z - z p_y \\
	 L_y &= z p_x - x p_z \\
	 L_z &= x p_y - y p_x 
\end{align}
Som i kvantemakanikken da blir, når vi opphøyer klassiske variabler til kvantemakaniske operatorer
\begin{align}
	 \hat{L}_x &= \hat{y} \hat{p}_z - \hat{z} \hat{p_y} =
	  y\frac{\hbar}{i} \frac{\partial}{\partial z} - z\frac{\hbar}{i} \frac{\partial}{\partial y}    \\
	 \hat{L}_y &= \hat{z} \hat{p}_x - \hat{x} \hat{p}_z =
	 z \frac{\hbar}{i} \frac{\partial}{\partial x}  - x\frac{\hbar}{i} \frac{\partial}{\partial z} \\
	 \hat{L}_z &= \hat{x} \hat{p}_y - \hat{y} \hat{p}_x = 
	  x\frac{\hbar}{i}  \frac{\partial }{\partial y}  - y \frac{\hbar}{i} \frac{\partial}{\partial x} 
\end{align}
Siden vi nå vil beskrive angulær bevegelsesmengde så er det enklest å bruke sfæriske koordinater.
Vi har at relasjonen mellom sfæriske og Cartesiske koordinater er 
\begin{align}
	 x &= r \sin(\theta) \cos(\phi) \\
	 y &= r \sin(\theta) \sin(\phi) \\
	 z &= r \cos(\theta)  
\end{align}
Dette gir 
\begin{align}
	 \frac{\partial x}{\partial \phi} &= -y \\ 
	 \frac{\partial y}{\partial \phi} &= x \\ 
	 \frac{\partial z}{\partial \phi} &= 0 \\ 
\end{align}
Dette gir videre at 
\begin{align}
	  \frac{\partial }{\partial \phi} &= 
	\frac{\partial x}{\partial \phi} \frac{\partial}{\partial x}  +
 	 \frac{\partial y }{\partial \phi} \frac{\partial}{\partial y} 
 	+ \frac{\partial z}{\partial \phi} \frac{\partial}{\partial z} \\
	&= -y \frac{\partial}{\partial x}  + x \frac{\partial}{\partial y} 
\end{align}
Dette gir da at z-komponenten av drivmoment operatoren er forholdsvis enkel 
\begin{align}
	 \hat{L}_z = \frac{h}{i}\frac{\partial}{\partial \phi} 
\end{align}
siden rotasjon om z-aksen bare krever forrandring i azimutal vinkelen.
Rotasjon om x- eller y-aksen krevere endring i både azimutal og polar vinkel og er derfor noe mer 
kompleks.
Vi har at 
\begin{align}
	 \hat{L}_x &= \frac{\hbar }{i} \big( -\sin(\phi) \frac{\partial}{\partial \theta}
	-cot(\theta) \cos(\phi) \frac{\partial}{\partial \phi} \big) \\
	 \hat{L}_y &= \frac{\hbar}{i}\big( \cos(\phi) \frac{\partial }{\partial \theta} 
	-cot(\theta) \sin(\phi) \frac{\partial}{\partial \phi} \big)
\end{align}
Vi har da 
\begin{align}
	 \mathbf{ \hat{L}}^2 &= \hat{L}_x^2 + \hat{L}_y^2 + \hat{L}_z^2 \\
	&= -\hbar^2 \Big[  \frac{1}{\sin(\theta)} \frac{\partial}{\partial \theta} 
		\big(\sin(\theta) \frac{\partial}{\partial \theta}\big) +
	\frac{1}{\sin^2(\theta)} \frac{\partial^2}{\partial \phi^2} \Big]
\end{align}
Vi vil nå finne eigenfunksjonene til $ \mathbf{ \hat{L}}^2 $. Vi starter med å merke oss at 
\begin{align}
	[ \mathbf{\hat{L}}^2, \hat{L}_z] = 0 	 
\end{align}
som vi ser må være tilfellet siden $ \hat{L}_z $ deriverer mhp. $ \phi $ og $ \mathbf{ \hat{L}}^2 $ 
ikke har noen direkte $ \phi $ avhengighet annen enn andrederivert-operatorenm og vi vet at rekkefølgen 
av partiell-deriverte er likegyldig hvilket gir oss at  $ \mathbf{\hat{L}}^2$ og $\hat{L}_z$ kommuterer.
Det viser seg at vi også har 
\begin{align}
	[ \mathbf{\hat{L}}^2, \hat{L}_x] = [ \mathbf{\hat{L}}^2, \hat{L}_y] = 0 	 
\end{align}
men dette er litt emr infløkt å vise; man må regne ut kommutatoren for å se det.

Dette betyr da at $ \mathbf{ \hat{L}}^2$ og $ \hat{L}_z $ har et komplett sett av simultane 
eigenfunksjoner. Dette betyr at 
vi kan beregne eigenfunksjonene til $ \hat{L}_z $ identisk til $ \mathbf{ \hat{L}}^2 $ 
som formodentlig er enklere.

Vi har benytter oss av seperasjon av variabler og skriver $ Y(\theta, \phi) = \Theta(\theta) \Phi(\phi) $ 
som er eigenfunksjonen til både $ \mathbf{ \hat{L}}^2 $ og $ \hat{L}_z $ slik at
\begin{align}
	\hat{L}_z Y(\theta, \phi)  &= L_z Y(\theta, \phi)	 \\
	\hat{L}_z \Theta(\theta) \Phi(\phi)  &= L_z \Theta(\theta) \Phi(\phi)	 
\end{align}
Slik at vi får
\begin{align}
 \hat{L}_z Y(\theta, \phi)  &= L_z Y(\theta, \phi)	 \\
  \Rightarrow \frac{\hbar}{i} \frac{\partial}{\partial \phi} \Theta(\theta)\Phi(\phi)  
			    &= L_z \Theta(\theta)\Phi(\phi)\\
 \Rightarrow \Theta(\theta) \frac{\hbar}{i} \frac{\partial}{\partial \phi} \Phi(\phi)  
			    &= \Theta(\theta) L_z\Phi(\phi) \\
  \Rightarrow \frac{\hbar}{i} \frac{\partial}{\partial \phi} \Phi(\phi)  &= L_z \Phi(\phi)
\end{align}
Som gir at 
\begin{align}
	 \Phi(\phi) = e^{i\frac{L_z\phi}{\hbar}}
\end{align}
hvor vi ser bort i fra konstanten som burde ha kommet fra løsningen i differensialligningen, vi setter 
heller konstanten til slutt når vi normaliserer $Y(\theta, \phi)$.
hvor vi ser bort i fra konstanten som burde ha kommet fra løsningen i differensialligningen, vi setter 
heller konstanten til slutt når vi normaliserer $Y(\theta, \phi)$.
Dette er da en eigenfunksjon for $ \hat{L}_z $,
siden vi kan ha $\Theta(\theta)$ som hva som helst, e.g. en konstant.

Dette uttrykket ligener umiskinnelig på uttrykket for bevegelsesmengdens 
eigenfunksjon. Men det er en stor forskjell, siden vi her trenger å ha at når $ \phi $ forrandrer seg 
med  $2\pi$ så må vi komme tilbake til det samme stedet. Vi trenger derfor å ha at 
\begin{align}
	 \Phi(\phi) = \Phi(\phi + 2\pi)
\end{align}
Siden om dette ikke er tilfellet og vi har e.g. $ \Phi(0) \neq \Phi(2\pi) $ så er ikke den deriverte 
definert ved $ \phi = 0 $. Dette er ekvivalent med at den deriverte mhp. x (og derfor bevegelsesmengden) 
er uendelig om bølgefunksjonen ikke er kontinuerlig.
Dette gir da 
\begin{align}
	 e^{i \frac{L_z(\phi + 2\pi)}{\hbar}} &=  e^{i \frac{L_z\phi}{\hbar}} \\
	\Rightarrow e^{i \frac{2\pi L_z}{\hbar}} &=  1
\end{align}
Hvor vi vet at vi har at $e^{ix}=1$ når $x$ er heltallsmultiplum av $\pi$ slik at vi får 
\begin{align}
	 L_z = m_l \hbar \hspace{1cm} \forall m_l \in \mathbb{N}
\end{align}
Dvs vi har alltid at z-komponenten av drivmomentet er kvantisert, og de tillatte verdiene er 
heltallsmultiplum av $\hbar$.

Vi vil nå bestemme $ \theta $ avhengigheten til $ Y(\theta, \phi) $ og vi har eigenverdiligningen
\begin{align}
	\mathbf{ \hat{L}}^2 Y(\theta, \phi) = l(l+1) \hbar^2 Y(\theta, \phi) 
\end{align}
hvor vi faktorerer ut en faktor av $\hbar^2$ av eigenverdien, slik at $ l(l+1) $ 
er dimensjonsløs. Vi taper ingen generalitet på å uttrykke eigenverdien på denne måten,
og vi skal snart se hvorfor vi velger å skrive den på denne formen. Dette gir 
\begin{align}
	-\hbar^2 \Big[  \frac{1}{\sin(\theta)} \frac{\partial}{\partial \theta} 
		\big(\sin(\theta) \frac{\partial}{\partial \theta}\big) +
	\frac{1}{\sin^2(\theta)} \frac{\partial^2}{\partial \phi^2} \Big] \Theta(\theta)\Phi(\phi) 
	&= l(l+1) \hbar^2 \Theta(\theta) \Phi(\phi)  \\
	\Rightarrow -\hbar^2 \Big[  \frac{1}{\sin(\theta)} \frac{\partial}{\partial \theta} 
		\big(\sin(\theta) \frac{\partial}{\partial \theta}\big) +
	\frac{1}{\sin^2(\theta)} \frac{\partial^2}{\partial \phi^2} \Big] \Theta(\theta) e^{i\phi m_l}
	&= l(l+1) \hbar^2 \Theta(\theta)  e^{i\phi m_l}  \\
	\Rightarrow -\hbar^2 \Big[  \frac{1}{\sin(\theta)} \frac{\partial}{\partial \theta} 
		\big(\sin(\theta) \frac{\partial}{\partial \theta}\big) -
	\frac{m_l^2}{\sin^2(\theta)}  \Big] \Theta(\theta) 
	&= l(l+1) \hbar^2 \Theta(\theta)  \\
	\Rightarrow \Big[  \sin(\theta) \frac{\partial}{\partial \theta} 
		\big(\sin(\theta) \frac{\partial}{\partial \theta}\big)  + l(l+1) \sin^2(\theta) -
	m_l^2  \Big] \Theta(\theta) 
	&=  0
\end{align}
Dette er Legendre's ligning for argument $\cos(\theta)$ og løsningene til dette er 
\begin{align}
	 \Theta(\theta) = N P_l ^{m_l}(\cos(\theta)) 
\end{align}
assosierte Legendre funksjonene 
\begin{align}
	 P^{m_l}_l(x) =  (1-x^2)^{ \frac{|m_l|}{2}} \big(\frac{\partial}{\partial x} \big)^{|m_l|} P_l(x)
\end{align}
hvor $P_l(x)$ igjen er Legendre polynomene gitt e.g. ved Rodrigues formel som sier at
\begin{align}
	 P_l(x) = \frac{1}{2^l l! } \big( \frac{\partial}{\partial x} \big)^{l} (x^2 - 1)^l
\end{align}
Det $l$'te Legendre polynomet $ P_l(x) $  er et $l$'te ordens polynom som er enten jevnt eller 
odde avhengig av pariteten til $l$. Men $ P^m_l_l(x) $ er ikke generelt et polynom siden vi får en faktor 
$ \sqrt{1-x^2} $ når $m_l$ er odde. 
Vi ser videre at når $m_l < l$ så får vi at $P^{m_l}_l(x) = 0$ og vi ser at Rodrigues formel kun 
gir mening for ikke-negative heltall.
\begin{align}
	l &= 0,1,2,3,\ldots \\
	m_l &= -l, -l+1, \ldots, -1, 0, 1, \ldots , l-1, l
\end{align}
Det finnes dog løsninger for andre verdier av $m_l$ og $l$ av Legendre's ligning, men det viser seg at
de løsningen da ikke er endelige ved $\theta=0$ og $ \theta = \pi $. Disse løsningene er derfor 
ikke normaliserbar og følgelig ikke fysikalske, og vi må forkaste de.
Vi har da at løsningene er $Y_l^{m_l}(\theta, \phi) =N P_l^{m_l}(\theta)e^{im_l\phi}$ som vi kan 
bestemme konstanten til vha normaliseringsbetingelsen. Vi finner da de sfæriske harmoniske er 
\begin{align}
 Y^{m_l}_l = (-1)^{m_l} \sqrt{ \frac{(2l + 1)}{4\pi} \frac{(l-m)!}{(l+m)!}} 
P^{m_l}_l(\cos(\theta))e^{im_l\phi}
\end{align}
Vi har at de sfæriske harmoniske er ortonormale 
\begin{align}
 \int_{0}^{2\pi} \int_{0}^{\pi} (Y^{m_l'}_l')^*Y^{m_l}_l \sin(\theta) d\theta d\phi 
= \delta_{m_l, m_l'} \delta_{l, l'}
\end{align}
Slik at vi har både eigenverdienen til kvadratet og z-komponenten til drivmomentet som kvantisert.
Vi skriver de sfæriske harmoniske som $ Y^{m_l}_l(\theta, \phi) $ og eigenverdiligningene 
\begin{align}
	 \hat{L}_zY^{m_l}_l(\theta, \phi) &= m_l\hbar  Y^{m_l}_l(\theta, \phi)
\hspace{1.6cm} m_l = 0, \pm 1, \pm 2, \ldots \\
	 \mathbf{\hat{L}}^2 Y^{m_l}_l(\theta, \phi) &= l(l+1)\hbar^2 Y^{m_l}_l(\theta, \phi)
\hspace{1cm} l = 0, 1, 2, 3, \ldots
\end{align}
Vi ser på absoluttverdien til drivmomentet 
\begin{align}
	 | \mathbf{L}|= \sqrt{l(l+1)}\hbar 
\end{align}
og merker oss at dette alltid er strengt større enn den maksimale projeksjonen av drivmomentet på 
z-aksen
\begin{align}
	L_{z, max} = l \hbar 
\end{align}
Siden $m_l$ alltid er mindre enn eller lik $l$. Hvilket vil si at vi aldri kan legge hele drivmomentet, 
som er en vektor, stringent i z-retning. 

Vi kan forstå dette ved å betrakte kommutatoren og uskarphetsrelasjonene mellom drivmoment-operatorene 
i de forskjellige retningene. Vi har e.g. 
\begin{align}
[ \hat{L}_x, \hat{L}_y] 
&= [ y\frac{\hbar}{i} \frac{\partial}{\partial z} - z \frac{\hbar}{i}\frac{\partial}{\partial y}  
 , \; z\frac{\hbar}{i} \frac{\partial}{\partial x} - x \frac{\hbar}{i}\frac{\partial}{\partial z}] \\
&= [ y\frac{\hbar}{i} \frac{\partial}{\partial z} , z \frac{\hbar}{i}\frac{\partial}{\partial x}]  
 +  [z\frac{\hbar}{i} \frac{\partial}{\partial y}, x \frac{\hbar}{i}\frac{\partial}{\partial z}] \\
&= y[\frac{\hbar}{i} \frac{\partial}{\partial z} , z]  \frac{\hbar}{i}\frac{\partial}{\partial x} 
 +  \frac{\hbar}{i} \frac{\partial}{\partial y}[z,  \frac{\hbar}{i}\frac{\partial}{\partial z}]x \\
&= y[ \hat{p}_z, \hat{z}]  \frac{\hbar}{i}\frac{\partial}{\partial x} 
 +  \frac{\hbar}{i} \frac{\partial}{\partial y}[ \hat{z}, \hat{p}_z]x \\
&= -i\hbar y \frac{\hbar}{i}\frac{\partial}{\partial x} 
 + \frac{\hbar}{i} \frac{\partial}{\partial y}x i\hbar \\
&= i\hbar (-y \frac{\hbar}{i}\frac{\partial}{\partial x} 
 + \frac{\hbar}{i} \frac{\partial}{\partial y}x) \\ 
&= i\hbar \hat{L}_z
\end{align}
hvor vi har brukt kommutator-identiten 
\begin{align}
[ \hat{A} + \hat{B}, \hat{C} + \hat{D}] = [ \hat{A}, \hat{C}] + [ \hat{A}, \hat{D}] 
+[ \hat{B}, \hat{C}]  + [ \hat{B}, \hat{D}] 
\end{align}
og at hvert av leddene som deriverer med hensyn på en variabel kommutere med ledd som ikke 
avhenger av den variabelen.

Vi ser da at x- og y-komponenten til drivmomentet ikke kommuterer og vi vet da at vi har 
en uskarphetsrelasjonen mellom de. Vi får 
\begin{align}
	 \Delta L_x \Delta L_y \ge \frac{\hbar}{2}| \langle L_z  \rangle |
\end{align}
Det vil si at om vi har en partikkel i en eigentilstand av $ \hat{L}_z $ med eigenverdi $m_l \hbar \neq 0$
så følger det at z-komponenten til partikkel er kjent med sikkerhet, men siden $ \langle L_z  \rangle $ 
så er $ \Delta L_x $ og $ \Delta L_y $ begge ulik 0. Dette betyr at det er en iboende uskarphet i $L_x$ og
$ L_y $ og vi ser at vi i kvantemekanikk ikke har et begrep om at drivmomentet er en vektor i rommet som 
peker i en bestemt retning, siden det impliserer at vi kjenner $ L_x $, $ L_y $ og $ L_z $ skarpt 
samtidig.

Vi betrakter dette i kontrast til lineær bevegelsesmengde hvor vi har 
\begin{align}
 [ \hat{p}_x, \hat{p}_y]
 = [ \frac{\hbar}{i} \frac{\partial}{\partial x} , \frac{\hbar}{i} \frac{\partial V}{\partial x} ] 
 = 0
\end{align}
slik at vi kan bestemme de forskjellige komponentene til bevegelsesmengden skarpt samtidig.

\section*{Hydrogenatomet}
Vi skal nå se og regne på Hydrogenatomet. Vi vet at vi har et elektristk påtensiale mellom protonet og 
elektronet i Hydrogen atomet, så vi anntar derfor et Coulomb potensialet som vi har som 
\begin{align}
	 V_C(r) = - \frac{Ze^2}{4\pi \epsilon_0 r}
\end{align}
vi vet er sentralsymmetrisk. Vi har at $e$ er elementærladningen altså ladningen til elektronet, 
$\epsilon_0$ er vakuumpermittiviteten og $Z$ er atomtallet som for Hydrogenatomet må være $Z=1$, men 
vi kan også uten videre betrakte ionisert Helium med $Z=2$.
Vi har da T.U.S.L. som
\begin{align}
	 E \psi &= \hat{H}\psi \\ 
		&= (-\frac{\hbar^2}{2m} \nabla^2 + V_C(r)) \psi  \\
	 &= -\frac{\hbar^2}{2m}\Big(\frac{1}{r^2} \frac{\partial }{\partial r} 
	\Big(r^2 \frac{\partial}{\partial r} \Big)
   + \frac{1}{r^2 \sin(\theta)} \frac{\partial}{\partial \theta}
     \Big(\sin(\theta) \frac{\partial}{\partial \theta}\Big)
     + \frac{1}{r^2 \sin(\theta)} \frac{\partial^2}{\partial \phi^2} \Big)\psi
    - \frac{Ze^2}{4\pi \epsilon_0 r}\psi  \\
 &= -\frac{\hbar^2}{2m}\Big(\frac{1}{r^2} \frac{\partial }{\partial r} 
	\Big(r^2 \frac{\partial}{\partial r} \Big)
   + \frac{1}{r^2} \mathbf{ \hat{L}}^2 \Big)\psi
    - \frac{Ze^2}{4\pi \epsilon_0 r}\psi 
\end{align}
hvor vi eksplisit ser at all vinkel-avhengigheten til Hamilton-operatoren er gitt i $ \mathbf{ \hat{L}}^2$
som gjør at vi får følgende Ansatz
\begin{align}
	 \psi = R(r)Y_l^{m_l}(\theta, \phi)
\end{align}
Dette gir ved innsettelse i T.U.S.L 
\begin{align}
  ER(r) Y_l^{m_l}&= -\frac{\hbar^2}{2m} \frac{1}{r^2} \frac{\partial }{\partial r} 
	\Big(r^2 \frac{\partial}{\partial r} \Big)R(r) Y_l^{m_l}
   + \frac{1}{2mr^2} \mathbf{ \hat{L}}^2 R(r) Y_l^{m_l}
    - \frac{Ze^2}{4\pi \epsilon_0 r}R(r) Y_l^{m_l} \\
  \Rightarrow ER(r) &= -\frac{\hbar^2}{2m}\frac{1}{r^2} \frac{\partial }{\partial r} 
	\Big(r^2 \frac{\partial}{\partial r} \Big) R(r)
   + \frac{l(l+1)\hbar^2}{2mr^2} R(r)
    - \frac{Ze^2}{4\pi \epsilon_0 r}R(r) 
\end{align}
Vi gjennomfører et variabel bytte for å forenkle dette litt 
\begin{align}
	 u(r) = rR(r)
\end{align}
Dette gir 
\begin{align}
   Eu(r) &= -\frac{\hbar^2}{2m}\frac{\partial^2 }{\partial r^2} u(r)
   + \Big(\frac{l(l+1)\hbar^2}{2mr^2}  - \frac{Ze^2}{4\pi \epsilon_0 r}\Big)u(r) 
\end{align}
som vi merker at er lik den en-dimensjonale S.L. men med et nytt effektivt potensiale
\begin{align}
	 V_{eff} = \frac{l(l+1)\hbar^2}{2mr^2}  - \frac{Ze^2}{4\pi \epsilon_0 r}
\end{align}
Vi har her at den første termen er et sentrifugal ledd, og den kalles ofte sentrifugal barrieren. 
Dette leddet holder for $l>0$ elektronet vekke ifra protonet.


Vi starter med å se på ligningene for partikler som er langt ifra origo, altså for $ r \gg 0 $. 
Vi har her at 
\begin{align}
	 \frac{1}{r} &\to 0 \\
	 \frac{1}{r^2} &\to 0 \\
\end{align}
Slik at vi får ligningen 
\begin{align}
 - \frac{\hbar^2}{2m}\frac{\partial^2}{\partial r^2} u(r) = E u(r) \hspace{1cm} (r \to \infty)
\end{align}
Dette ser vi har løsninger 
\begin{align}
	 u(r) = e^{\pm \frac{\sqrt{-2mE}}{\hbar}r} \hspace{1cm} (r\to\ \infty)
\end{align}
hvor vi antar at vi er interessert i bundne partikler med $ E < 0 $. Vi forkaster de løsningene 
som divergerer for $ r\to \infty $, slik at vi bare har leddet med negativt fortegn i 
eksponenten.

Vi ser videre på det andre ekstremtilfellet får partikler som er veldig nære origo, slik at 
$ r \ll 0 $, og vi får da 
\begin{align}
	\frac{\partial^2}{\partial r^2} u(r) = \frac{l(l+1)}{r^2} u(r) 	 \hspace{1cm} (r\to 0)
\end{align}
Vi får da følgende løsninger
\begin{align}
	 u(r) &= r^{-l} \hspace{1.1cm} (r\to 0) \\
	 u(r) &= r^{l+1} \hspace{1cm} (r\to 0) 
\end{align}
Hvor vi skal se bort i fra den første løsningen $ r^{-l} $ siden dette vil føre til 
normaliseringproblemer. Vi kan se at vi har en løsning 
\begin{align}
	 \frac{\partial^2}{\partial r^2} u(r) &= \frac{\partial^2}{\partial r^2} r^{l+1} \\
	  &= (l+1)lr^{l-1}   \\
	  &= (l+1)l r^{-2} r^{l+1} \\   
	  &= \frac{l(l+1)}{r^2} u(r)
\end{align}
Vi har altså nå funnet oppførselen til den radiale delen av bølgefunksjonen for store og små 
verdier av $ r $. Vi kan derfor skrive 
\begin{align}
	 u(r) = r^{l+1} F(r) e^{- \frac{\sqrt{-2mE}}{\hbar}r }
\end{align}
eller ekvivalent 
\begin{align}
	 R(r) = r^{l} F(r) e^{- \frac{\sqrt{-2mE}}{\hbar}r }
\end{align}
På denne formen kan vi enklere finne bølgefunksjonen, men vi må fortsatt massere ligningene noe 
før vi kan løse de.
Vi innfører nå et variabelbytte
\begin{align}
	 \rho = \frac{\sqrt{-8mE}}{\hbar}r
\end{align}
Vi får da 
\begin{align}
	 u(\rho) = \rho^{l+1} F(\rho) e^{ - \frac{\rho}{2}}
\end{align}
hvor vi har inkorporert en konstant i $ F(r) $, 
og radial delen av T.U.S.L blir 
\begin{align}
	 \frac{\partial^2 u(\rho)}{\partial \rho^2}  - \frac{l(l+1)}{\rho^2} u(\rho) 
	+ ( \frac{\lambda}{\rho} - \frac{1}{4}) u(\rho) = 0
\end{align}
hvor 
\begin{align}
	 \lambda = \frac{Z e^2}{4\pi \epsilon_0 \hbar} \sqrt{ \frac{m}{-2E}}
\end{align}
Ved innsettelse for $ u(\rho) $ i T.U.S.L får vi da følgende uttrykk for $ F(r) $  
\begin{align}
 \frac{\partial^2 F(\rho)}{\partial r^2} + ( \frac{2l + 2}{\rho} - 2) 
\frac{\partial F(\rho)}{\partial \rho} + ( \frac{\lambda}{\rho} - \frac{l+1}{\rho})F(\rho) = 0 
\end{align}
Dette ser ved første øyenkast ikke så mye bedre ut en der vi startet, men dette kan vi løse ved 
en rekke-utvikling Ansatz. Vi har
\begin{align}
	F(\rho) = \sum_{k=0}^{ \infty}  c_k \rho^k 
\end{align}
Ved innsettelse får vi nå 
\begin{align}
\sum_{k=0}^{\infty} k(k-1)c_k\rho^{k-2} + (\frac{2l+2}{\rho} -1)\sum_{j=0}^{ \infty}jc_j\rho^{j-1} 
+ ( \frac{\lambda}{\rho} - \frac{l+1}{\rho}) \sum_{t=0}^{ \infty} c_t \rho^t &= 0 \\
\Rightarrow \sum_{k=0}^{ \infty} k(k-1)c_k\rho^{k-2} 
    + ( 2l + 2 - \rho)\sum_{j=0}^{ \infty}jc_j\rho^{j-2} 
    + (\lambda - (l+1)) \sum_{t=0}^{ \infty} c_t \rho^{t-1} &= 0 \\
\Rightarrow \sum_{k=0}^{ \infty} k(k-1)c_k\rho^{k-2} 
    + ( 2l + 2)\sum_{j=0}^{ \infty}j c_j\rho^{j-2} 
     - \sum_{j=0}^{ \infty}j c_j\rho^{j-1} 
    + (\lambda - (l+1)) \sum_{t=0}^{ \infty} c_t \rho^{t-1} &= 0 \\
\Rightarrow \sum_{k=0}^{ \infty} k(k-1)c_k\rho^{k-2} 
    + \sum_{j=0}^{ \infty} ( 2l + 2) j  c_j\rho^{j-2} 
    + \sum_{t=0}^{ \infty} ( -t + \lambda - (l+1))  c_t \rho^{t-1} &= 0
\end{align}
Vi gjør variabel bytte $k'=k-1$ og $j'=j-1$ slik at 
\begin{align}
\Rightarrow \sum_{k'=0}^{ \infty} k'(k'+1)c_{k'+1}\rho^{k'-1} 
    + \sum_{j'=0}^{ \infty} ( 2l + 2) (j'+1)  c_{j'+1}\rho^{j'-1} 
    + \sum_{t=0}^{ \infty} ( -t + \lambda - (l+1))  c_t \rho^{t-1} &= 0
\end{align}
Vi kan nå kalle alle summasjonsvariablene det samme, e.g. $ k $ og trekke de inn i samme sum 
\begin{align}
\Rightarrow \sum_{k=0}^{ \infty} 
  \Big[(k(k+1)+  ( 2l + 2) (k-1) ) c_{k+1}
    +  ( -k + \lambda - (l+1))  c_k \Big] \rho^{k-1} &= 0
\end{align}
Som gir oss 
\begin{align}
	 \frac{c_{k+1}}{c_k} = \frac{k + l + 1 -\lambda}{k(k+1)+  ( 2l + 2) (k-1) }
\end{align}
Ligningen er oppfylt for en rekke-utvikling løsning så lenge koeffisientene oppfyler dette.
Vi ser på hva som skjær for de høyere ordens leddene $ k \to \infty $ hvor vi har 
\begin{align}
	 \frac{c_{k+1}}{c_k} \sim \frac{1}{k}
\end{align}
Dette er dog den asymptotiske oppførselen til $ e^{\rho} $ og hvis dette er tilfellet 
er det ikke mulig å ha en normaliserbar løsning. Vi må derfor kreve at rekke-løsningen 
teminerer for $ k $ går helt til uendelig. 
Dette er tilfellet når 
\begin{align}
	 \lambda = l + 1 + n_r \hspace{1cm} n \in \mathbb{N}_+
\end{align}
hvor altså vi må kreve at $n_r$ er et heltall. Videre definerer vi 
\begin{align}
	 n = l + 1 + n_r
\end{align}
Det betyr at $F(\rho)$ vil være et polynom av $ n $'te grad, og disse polynomene kalles 
Laguerre-polynomer, og vi ser at vi trenger å alltid ha 
\begin{align}
	 l < n-1
\end{align}
Vi har videre at siden $ \lambda $ er en funksjon av $ E $ og visa versa 
så har vi funnet at energien her igjen er kvantisert.
Vi har 
\begin{align}
	 \lambda = l + 1 + n &= \frac{Z e^2}{4\pi \epsilon_0 \hbar} \sqrt{ \frac{m}{-2E}} \\
	\Rightarrow n^2 &= -\frac{Z^2 e^4}{(4\pi \epsilon_0)^2 \hbar^2} \frac{m}{2E} \\
	\Rightarrow E &= -\frac{mZ^2 e^4}{2(4\pi \epsilon_0)^2 \hbar^2n^2}  \\
\end{align}
som nå gir oss energinivåene for et elektron bundet i et Hydrogen-atom som 
\begin{align}
	E_n &= -\frac{mZ^2 e^4}{2(4\pi \epsilon_0)^2 \hbar^2} \frac{1}{n^2} \\
	&= -\frac{mc Z^2 \alpha^2}{2 } \frac{1}{n^2} 
\end{align}
og vi har finstruktur-konstanten som
\begin{align}
	 \alpha = \frac{e^2}{4\pi\epsilon_0\hbar c} \approx \frac{1}{137}
\end{align}
dette er en viktig størrelse i fysikken og beskriver koblingen mellom elementærpartikler med 
ladning i elektriske felter.
Vi kan også skrive $ E_n $ git ved grunntilstandsenergien til et enkelt Hydorgen-atom som 
har $ Z=1 $ 
\begin{align}
	 E_n =  \frac{E_0 Z^2}{n^2} = - 13.6eV \frac{Z^2}{n^2}
\end{align}

\subsection*{Bølgefunksjonen}

For ordens skyld tar vi med oss hvordan den fullstendige bølgefunksjonen til slike en-partikkel
atomer ser ut.
Vi får at løsningene våre er
\begin{align}
	F(\rho) = L^{2l + 1}_{n-l-1}(\rho)	 
\end{align}
hvor vi har de assosierte Laguerre-polynomene 
\begin{align}
	 L^q_{q-p}(x) = (-1)^p ( \frac{\partial}{\partial x} )^p L_q(x)
\end{align}
og det q'ende Laguerre-polynomene er 
\begin{align}
	 L_q(x) = e^x ( \frac{\partial}{\partial x} )^q (e^{-x} x^q)
\end{align}
Det vil si at den radielle delen av bølgefunksjonen er gitt som 
\begin{align}
	 R(r) = \frac{1}{r}e^{ - \frac{r}{n\alpha}} \big( \frac{2r}{n \alpha}\big)^{l+1}
	L^{2l + 1}_{n-l-1}( \frac{2r}{n\alpha})	 
\end{align}
Slik at vi har den fullstendige bølgefunksjonen etter normalisering til Hydrogen-atomet som 
\begin{align}
	 \psi_{nlm}(r,\theta,\phi) = 
	\sqrt{\big(\frac{2}{n\alpha}\big)^3 \frac{(n-l-1)!}{2n[(n+l)!]^3}} e^{- \frac{r}{n\alpha}}
	\; \big( \frac{2r}{n\alpha} \big)^l   \;
	L^{2l + 1}_{n-l-1}\big( \frac{2r}{n\alpha}\big) \; Y^{m_l}_{l} (\theta, \phi)	 
\end{align}

\subsection*{Kvantetall}
Vi har nå 3 kvantetall som bestemmer egenskapene til dette systemet nemlig 
\begin{description}
\item[Hovedkvantetall $ n $ ] eller prinsipalkvantetallet. Bestemmer energien til elektronet 
		i Hydorgen-atom hvor vi har $ E_n \sim n^{-2} $ . Vi har at hovedkvantetallet tar heltallsverdier, slik at 
		\begin{align}
			 n = 1,2,3,4,\ldots
		\end{align}
\item[Orbital kvantetall $ l $ ] eller det asimutale kvantetallet. Bestemmer banespinnet til 
		en partikkel hvor vi har $|L| = \sqrt{l(l+1)}\hbar$. Vi har at 
		\begin{align}
			 l = 0, 1, 2, \ldots n-1
		\end{align}
\item[Magnetisk kvantetall $ m_l $ ] Angir projeksjonen av banespinnet i en bestemmt retning og 
	vi har at for eksempel så er $L_z = m_l\hbar$.
	Vi har at $m_l$ kan ta følgende verdier
	\begin{align}
		 m_l = -l, -l+1, \ldots, -1,0,1,\ldots, l-1, l
	\end{align}
\end{description}
Disse kvantetallene forteller beskriver tilstanden til Hydrogen-atomet og andre atomære 
tilstander, dvs for $ Z \neq 1 $, og vi bruker dem til å argumentere rundt fler-elektron atomer.

Vi merker oss at energinivåene bare avhenger av kvantetallet $ n $ men ikke av $ l $ og $ m_l $.
At energien ikke avhenger av $ m_l $ er lett å forstå, vi har et sentral-symmetrisk potensial 
og det spiller derfor ingen rolle hvilken akse vi kaller z-aksen. Energien avhenger derfor ikke 
på projeksjonen av banespinn på denne aksen.

Men at energien ikke avhenger av $ l $ er mer overraskende, men den radielle delen 
av bølgefunksjonen avhenger derimot av $ l $. Så den har noe å si for hvor (i hvilken avstand
fra kjernen) vi forventer å finne partikkelen.

Dette fører til en degenerasjon av spektrumet, siden veldig mange forskjellige tilstander 
med forskjellige verdier av $ n $, $ l $ og $ m_l $ gir de samme energiene.
Vi vet at $ l $ kan være alle heltall opp til $ n-1 $ og for hver $ l $ så er det $ (2l+1) $ 
tillatte verdier av $ m_l $ det vil si at degenerasjonen for et gitt energinivå $ E_n $ er
\begin{align}
	 d(n) &= \sum_{l=0}^{n-1} (2l+1)  \\
	  &= 2\sum_{l=0}^{n-1} l + \sum_{l=0}^{n-1}1  \\ 
	  &= 2 \Big( \frac{(n-1)((n-1) +1)}{2}\Big) +  \sum_{l=1}^{n}1   \\
	  &=  (n-1)n +  n \\
	  &= n^2
\end{align}


\subsection*{Spektral overganger}
En av hovedgrunnene til å studere energinivåene til Hydrogen-atomet er å kunne se på
overgangene mellom energinivåer som danner grunnlaget for mange spektroskopiske teknikker.
Vi har at et elektron kan gå fra en energitilstand med energi $ E_{n_i} $ til $ E_{n_f} $ ved 
å sende ut et foton med energi 
\begin{align}
	 h\nu = E_{n_i} - E_{n_f}
\end{align}
Vi har da bølgelengdene til disse overgangene som 
\begin{align}
	 \frac{1}{\lambda} = R \big( \frac{1}{n_f^2} - \frac{1}{n_i^2}\big)
\end{align}
hvor 
\begin{align}
	 R = \frac{m}{4\pi c \hbar^3} \Big( \frac{e^2}{4\pi \epsilon_0}\Big)^3
\end{align}
Vi deler nå inne i flere typer overganger som er kjent i literaturen
og som er historisk veldig viktig og som skjer med utsendt stråling med forskjellige 
bølgelengder.

\begin{description}
\item[Balmer-serien] Vi har spektral linjene som kommer når et elektron gjennomgår 
	en overgang fra en tilstand med $ n_i >2 $ til $ n_f = 2$. De utsendte fotonen
	fra de fire laveste $ n_i $ gir at  denne transisjonen er i den synlige delen av det 
	elektromagnetiske spektrumet.
\item[Lyman-serien] overganger som har $ n_f = 1 $ har høyere energi og de utsendte fotonene 
	er i den ultraviolette delen av det elektromagnetiske spektret.
\item[Paschen-serien] overganger som ender i med $ n_f = 3 $ sender ut fotoner som infrarød 
	stråling.
\end{description}



\section*{Zeeman effekten}
Vi skal nå betrakte den såkalte Zeeman-effekten.
Vi betrakter enpartikkel med ladning $ q $ og masse $ m $ som beveger seg i en sirkuler bane med 
radius $ r $ om med hastighet $ v $. Partiklen beveger seg da i sirkel med en periode 
$ T = (2\pi r) / v $ og genererer en strømførende løkke med areal $ A = \pi r^2 $ som har strøm 
$ I = q / T = qv  / 2\pi r $. Fra klassisk elektrodynamikk vet vi at en slik strøm-løkke vil 
generere et magnetisk dipole moment $ \mu $ med absoluttverdi 
\begin{align}
	 |\mathbf{\mu}| &=  IA = \frac{qrv}{2} \\
	  &= \frac{qmrv}{2m} \\
	  &= \frac{q}{2m} | \mathbf{L}|
\end{align}
Vi definerer derfor for et kvantemekanisk system dipol-momentet operatoren til å være
\begin{align}
	\mathbf{\hat{\mu}}  &= \frac{q}{2m} \mathbf{\hat{L}}
\end{align}
hvor vi har at fortegnet til $ q $, altså om vi har negativ eller positiv ladning, bestemmer om 
dipol-momentet er parallell eller antiparallel til banespinnet.
Vi nevner her at kvantemekanikk ikke kan utledes fra klassisk fysikk og argumentet som ledet fram 
til dette uttrykket er strengt heuristisk. Partikler beveger seg ikke i sirkler og befinner seg 
ikke ved gitte punkter i rommet. Men det finnes ikke noen bedre enkel måte å motivere dette 
uttrykket.

Om dette magnetiske momentet settes i et ekstern magnetisk felt $ \mathbf{B} $ så vil 
vi ha en interaksjonsenergi
\begin{align}
	 - \mathbf{ \hat{\mu}} \cdot \mathbf{ \hat{B}} 
	= - \frac{q}{2m} \mathbf{ \hat{L}} \cdot \mathbf{ B} 
\end{align}
Vi får da at Hamilton operatoren får et ekstra ledd når vi har Hydrogen-atomet i et
eksternt magnetfelt
\begin{align}
	 H_{Zeeman} =  H_{Hydrogen} - \mathbf{ \hat{\mu}} \cdot \mathbf{B} 
\end{align}
Hvis vi antar at $ \mathbf{B} = B i_{z} $ og at vi ser på elektroner (i.e. at $ q=-e $ ) da får 
vi at
\begin{align}
	 H_{Zeeman} =  H_{Hydrogen}  +  \frac{eB}{2m} \hat{L}_z 
\end{align}
Siden vi vet at bølgefunksjonen til Hydrogen-atom er eigenfunksjoner til $ \hat{L}_z $ 
så får vi at energien til elektronet i et Hydrogen-atom som befinner seg i et eksternt 
magnetfelt er 
\begin{align}
	 E_{n, m_l} = E_n + \frac{eB}{2m} m_l \hbar 
	 = E_n + \mu_B B m_l 
\end{align}
hvor vi har Bohr magnetonen 
\begin{align}
	 \mu_B = \frac{e \hbar}{2m}
\end{align}

Vi har her altså at energien også avhenger av det magnetiske kvantetallet $ m_l $, og hvert 
energinivå med en gitt $ n $ splittes opp i flere energinivåer med forskjellige $ m_l $. 
Vi får altså mindre degenerasjon, vi sier at vi har brytt degenerasjonen. 
Dette kommer av at symmetrien brytes, siden vi har lagt til 
et ledd i Hamilton-operatoren som har en foretrukket retning. Siden vi ikke lenger har 
en sentralsymmetrisk ligning så spiller det en rolle hva projeksjonen av banespinnet på z-aksen 
er, siden energien avhenger av om dipol-momentet er rettet parallelt eller antiparallelt med 
magnetfeltet.


\section*{Egenspinn}

I klassisk mekanikk så deler man ofte inn rigide legemers banespinn i to deler,
banespinn $ \mathbf{L} = \mathbf{r}\times \mathbf{p} $  
som forteller hvordan massesentrumet beveger seg rundt en akse og spinnet $ \mathbf{S} = I\omega $
som er bevegelsen til legemet rundt massesenteret, altså hvordan det roterer.
Vi vet e.g. at jorden har banespinn $ \mathbf{L} $  assosiert med sin bevegelse rundt solen mens
spinnet $ \mathbf{S} $ assosiert med sin rotasjon rundt sin egen akse. 

Vi vet dog at $ \mathbf{S} $ ikke er noe mer enn summen av banespinnet $ \mathbf{L} $ for alle 
små deler av objektet rundt sitt massesentrum.

I kvantemekanikken har vi også at partikler har en slags spinn i tillegg til banespinnet.
Vi kaller dette spinnet egenspinn. Dette har dog ingenting med rotasjon i rommet, og kan derfor 
ikke beskrives av romlige variabler $ r $ , $ \theta $ og $ \phi $. Dette kan vi se at er umulig 
ifra at vi vet at elementærpartikler ikke har noe utrstrekning i rommet, siden de 
er strukturløse punktpartikler. Egenspinnet til en partikkel er en iboende egenskap til partikkelen
som den har i tillegg til det romlige, vanlige banespinnet. 

Egenspinnet er en observabel og vi representerer det med en hermitisk operator 
$ \mathbf{ \hat{S}} $. Teorien rundt egenspinn er helt ekvivalent til banespinnet og vi har 
\begin{align}
	  [\hat{S}_x, \hat{S}_y] = i\hbar \hat{S}_z \\
	  [\hat{S}_y, \hat{S}_z] = i\hbar \hat{S}_x \\
	  [\hat{S}_z, \hat{S}_x] = i\hbar \hat{S}_y
\end{align}
og
\begin{align}
	  [\hat{S}_x, \mathbf{\hat{S}}^2] 
	= [\hat{S}_y, \mathbf{\hat{S}}^2] =  [\hat{S}_z, \mathbf{\hat{S}}^2] = 0
\end{align}
Eigentilstandene til $ \mathbf{ \hat{S}}^2 $ og $ \hat{S}_z $ kaller vi $|\psi_{s, m_s} \rangle$
og oppfyller 
\begin{align}
	\mathbf{ \hat{S}}^2 | \psi_{s, m_s} \rangle  &= \hbar^2 s(s+1) | \psi_{s,m_s} \rangle  \\
	\hat{S}_z | \psi_{s, m_s} \rangle  &= \hbar m_s | \psi_{s,m_s} \rangle  
\end{align}
Men siden vi nå har at $ | \psi_{s,m_s} \rangle  $ ikke er kuleflatefunksjoner siden 
vi vet at de ikke kan være en funksjone av romlige koordinater som $\theta$ og $ \phi $ 
så har vi ingen 
grunn til å ikke ta med løsninger med halvtalls-verdier av $ s $ og $ m_s $ slik at vi har 
\begin{align}
	 s &= 0, \frac{1}{2}, 1, \frac{3}{2}, 2, \frac{5}{2}, \ldots \\
	 m_s &= -s, -s+1, \ldots, -1, 0, 1, \ldots, s-1, s
\end{align}
Det viser seg at alle elementærpartikler har en spesifikk of uforranderlig verdi av $ s $, som vi 
kaller spinnet til partiklen. Vi har at elektroner has spinn-$ 1 / 2 $, fotoner has spinn $ 1 $
pi-mesoner har spinn $ 3 / 2 $ og gravitoner har spinn $ 2 $. 
Vi skal nå se nærmere på spinn-$ 1 / 2 $ partikler.

\subsection*{Spinn-$ \frac{1}{2}$}
Mange essesielle partikler som elektroner, positroner, kvarker etc. er spinn- $ \frac{1}{2} $
partikler og vi skal derfor se nærmere på slike systemer.
For $ s = \frac{1}{2}$ partikler har vi bare to muligheter for $ m_s = \pm \frac{1}{2} $. Dvs. vi 
har et to-tilstands system, siden vi alltid har $ s=\frac{1}{2} $ har bare to muligheter 
for $ m_s $. Vi har altså et diskret system med diskret muligheter for for kvantetilstandene våre.
I dette tilfellet er det mer nærliggende å representere kvantetilstandene våre som matriser enn 
som bølgefunksjoner.
Vi har eigentilstandene som to-dimensojnale vektorer
\begin{align}
	 | \psi_{s, \frac{1}{2}} \rangle &= \chi^+ = \begin{bmatrix} 1 \\ 0 \end{bmatrix} \\
	 | \psi_{s, -\frac{1}{2}} \rangle &= \chi^+ = \begin{bmatrix} 0 \\ 1 \end{bmatrix}
\end{align}
Vi har altså et to dimensjonalt Hilbert rom som disse to spinorene er en basis for.
Operatorene er nå $ 2\times 2 $-matriser 
\begin{align}
	 \hat{S}_z = \frac{\hbar}{2} \begin{bmatrix} 1  & 0 \\ 0 & -1 \end{bmatrix}
\end{align}
Vi ser enkelt at $ \chi^{\pm} $ er eigentilstander til $ \hat{S}_z $ representert på denne 
måten 
\begin{align}
	 \hat{S}_z  \chi^+ &= \frac{\hbar}{2} \begin{bmatrix} 1  & 0 \\ 0 & -1 \end{bmatrix}
	\begin{bmatrix} 1  \\ 0 \end{bmatrix}
	 = \frac{\hbar}{2}\begin{bmatrix} 1  \\ 0 \end{bmatrix}
	 = \frac{1}{2}\hbar \chi^+ \\
	 \hat{S}_z  \chi^- &= \frac{\hbar}{2} \begin{bmatrix} 1  & 0 \\ 0 & -1 \end{bmatrix}
	\begin{bmatrix} 0  \\ 1 \end{bmatrix}
	 = -\frac{\hbar}{2}\begin{bmatrix} 0  \\ 1 \end{bmatrix}
	 = -\frac{1}{2}\hbar \chi^- 
\end{align}
Den kanonsike representasjonen av $ \hat{S}_x $ og $ \hat{S}_y $ er 
\begin{align}
	 \hat{S}_x &= \frac{\hbar}{2} \begin{bmatrix} 0  & 1 \\ 1 & 0 \end{bmatrix}\\
	 \hat{S}_y &= \frac{\hbar}{2} \begin{bmatrix} 0  & -i \\ i & 0 \end{bmatrix}
\end{align}
Vi utrykker spinn-operatoren ved hjelp av såkelate Pauli matriser 
\begin{align}
	 \hat{\sigma}_x &= \begin{bmatrix} 0  & 1 \\ 1 & 0 \end{bmatrix}\\
	\hat{\sigma}_y &= \begin{bmatrix} 0  & -i \\ i & 0 \end{bmatrix}\\
	 \hat{\sigma}_z &= \begin{bmatrix} 1  & 0 \\ 0 & -1 \end{bmatrix}
\end{align}
Slik at vi har $ \hat{S}_x = \frac{\hbar}{2} \hat{\sigma}_x $,
$ \hat{S}_y = \frac{\hbar}{2} \hat{\sigma}_y $ og $ \hat{S}_z = \frac{\hbar}{2} \hat{\sigma}_z $.
Det er trivielt å vise at spinn-projeksjons operatorene på matriseform oppfyler kommutasjons-
relasjonene. Vi har e.g. 
\begin{align}
	 [ \hat{S}_x, \hat{S}_y] &= \frac{\hbar^2}{4}
\begin{bmatrix} 0  & 1 \\ 1 & 0 \end{bmatrix}\begin{bmatrix} 0  & -i \\ i & 0 \end{bmatrix} 
 - \frac{\hbar^2}{4} \begin{bmatrix} 0  & -i \\ i & 0 \end{bmatrix}
 \begin{bmatrix} 0  & 1 \\ 1 & 0 \end{bmatrix}\\
	&= \frac{\hbar^2}{4} \begin{bmatrix} i  & 0 \\ 0 & -i \end{bmatrix} 
 - \frac{\hbar^2}{4} \begin{bmatrix} -i  & 0 \\ 0 & i \end{bmatrix} \\
	&= \frac{\hbar^2}{4} \begin{bmatrix} 2i  & 0 \\ 0 & -2i \end{bmatrix} \\
	&= i\hbar\frac{\hbar}{2} \begin{bmatrix} 1  & 0 \\ 0 & -1 \end{bmatrix} \\
	&= i\hbar \hat{S}_z
\end{align}
og tilsvarende kan vi vise at alle komponentene til egenspinn-operatoren kommuterer. Vi har også  
at 
\begin{align}
	 \mathbf{ \hat{S}}^2 &= \hat{S}_x^2 + \hat{S}_y^2 + \hat{S}_z^2 \\
    &=\frac{\hbar^2}{4}  \begin{bmatrix} 0  & 1 \\ 1 & 0 \end{bmatrix}
	\begin{bmatrix} 0  & 1 \\ 1 & 0 \end{bmatrix} 
    +  \frac{\hbar^2}{4}\begin{bmatrix} 0  & -i \\ i & 0 \end{bmatrix}
	\begin{bmatrix} 0  & -i \\ i & 0 \end{bmatrix} 
    +\frac{\hbar^2}{4}  \begin{bmatrix} 1  & 0 \\ 0 & -1 \end{bmatrix}
	\begin{bmatrix} 1  & 0 \\ 0 & -1 \end{bmatrix} \\
    &=\frac{\hbar^2}{4}  \begin{bmatrix} 1  & 0 \\ 0 & 1 \end{bmatrix}
    +  \frac{\hbar^2}{4}\begin{bmatrix} 1  & 0 \\ 0 & 1 \end{bmatrix}
    +\frac{\hbar^2}{4}  \begin{bmatrix} 1  & 0 \\ 0 & 1 \end{bmatrix} \\
    &= \frac{3\hbar^2}{4}  \begin{bmatrix} 1  & 0 \\ 0 & 1 \end{bmatrix} \\
    &= \frac{3\hbar^2}{4}  \mathbb{I}
\end{align}  

og altså $  $ 






 \end{document}

