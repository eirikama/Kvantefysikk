\documentclass{article}

\usepackage{standalone}
\usepackage{pgfplots}
\pgfplotsset{compat=newest}
\usepackage{tikz}
\usetikzlibrary{decorations.markings}

\usepackage{subfig}
\usepackage[margin=2.5cm]{geometry}
\usepackage{amsmath}
\usepackage{amssymb}


\newcommand\greybox[1]{
	\vskip\baselineskip
	\par\noindent\colorbox{lightgray}{
		\begin{minipage}{\textwidth}#1\end{minipage}
	}
	\vskip\baselineskip
}

\title{Prinsipper i kvantemekanikk}


\begin{document}
\maketitle

\section*{Paritetsoperatoren}

Vi har allerede sett på Hamilton-, posisjons- og bevegelsesmengde-operatorene. Nå skal 
vi introdusere en ny operator, nemlig paritetsoperatoren. Paritetsoperatoren
er en operator som gjør transformasjonen $x \to -x$, vi definerer den som 
\begin{align}
	 \hat{\Pi}\psi(x) = \psi(-x)
\end{align}
Vi har da følgende eigenverdiligning 
\begin{align}
	 \hat{\Pi}\psi_{\lambda}(x) = \lambda\psi_{\lambda}(x)
\end{align}
Vi kan løse denne ligningen ved å merke oss at $ \hat{\Pi}^2=1$ siden hvis vi transformerer 
$x \to -x$ to ganger er tilbake der vi startet. Vi har
\begin{align}
	 \hat{\Pi}^2 \psi_{\lambda} = \hat{\Pi}\lambda \psi_{\lambda} = \lambda^2 \psi_{\lambda}
\end{align}
som gir 
\begin{align}
	 \lambda^2 &= 1 \\
	\Rightarrow \lambda &= \pm 1
\end{align}
Dvs. eigentilstandene til paritetsoperatoren med eigenverdi $+1$ er jevne tilstander 
og $-1$ eigenverdier har odde tilstander.
Vi har sett at for symmetriske potensialer $V(x) = V(-x)$ er energi eigentilstandene 
også symmetriske, enten jevne eller odde. Så vi har derfor generelt at energi eigentilstander 
er også paritets eigentilstander.

Det er enkelt å vise at alle tilstander kan skrives som en lineærkombinasjon av jevne og odde 
tilstander. Som vi kan se ved å skrive om en arbitrær $\psi(x)$ på følgende måte
\begin{align}
	 \psi(x) = \frac{1}{2}(\psi(x) + \psi(-x)) + \frac{1}{2}(\psi(x) - \psi(-x))
\end{align}
hvor vi nå ser at det første leddet er jevnt mens det andre er odde.
Det betyr at også eigentilstandene til paritetsoperatoren danner et komplett sett.


\section*{Hermitiske operatorer og observable}

Vi starter med litt notasjon; vi innfører notasjonen hvor 
vi skriver en kvante tilstand som $ | \psi \rangle = \psi$ (dette kaller vi en ket) 
og vi lar den kompleks konjugert være $ \langle \psi| = \psi^*$ (som vi kaller en bra).
Fordelen med en slik notasjon er at vi 
kan nå har en veldig kompakt måte å skrive indreproduktet mellom tilstandene 
$\psi_n$ og $\psi_m$ som $ \langle \psi_n | \psi_m \rangle $, dvs vi har 
\begin{align}
	 \langle \psi_n|\psi_m  \rangle = \int_{- \infty}^{ \infty} \psi_n^*\psi_m dx
\end{align}
Dette indreproduktet kalles nå en braket.
Vi har også for en  operator $\hat{O}$ at 
\begin{align}
	 | \hat{O}\psi \rangle &= \hat{O} | \psi \rangle = \hat{O} \psi \\
(| \hat{O}\psi \rangle)^* = \langle \hat{O} \psi|  &=\langle \psi| \hat{O}^*  = \hat{O}^* \psi^*
\end{align}
Denne notasjonen stammer fra Paul Dirac og kalles 
Dirac notasjon, eller bra-ket notasjon, og er en veldig effektiv måte å uttrykke kvantetilstander 
på.

Videre sier vi at en operator $ \hat{O}$ er \textbf{hermitisk} hvis vi har 
\begin{align}
	\langle \phi| \hat{O}\psi \rangle &= \langle \hat{O}\phi| \psi \rangle  
\hspace{0.5cm}	\Leftrightarrow \hspace{0.5cm} \int_{- \infty}^{ \infty} \phi^* \hat{O}\psi dx =
  \int_{- \infty}^{ \infty} (\hat{O}\phi)^* \psi dx 
\end{align}
Vi har videre at om vi har to distinkte eigenverdier til en hermitisk operator 
så er eigentilstandene deres ortogonal, som vi ser fra 
\begin{align}
	 \langle \psi_{\lambda_1}| \hat{O} \psi_{\lambda_2} \rangle 
	&= \langle \hat{O} \psi_{\lambda_1}| \psi_{\lambda_2} \rangle \\
	\Rightarrow \langle \psi_{\lambda_1}|\lambda_2 | \psi_{\lambda_2} \rangle 
	&= \langle \psi_{\lambda_1}| \lambda_1 | \psi_{\lambda_2} \rangle \\
	\Rightarrow (\lambda_2 - \lambda_1) \langle \psi_{\lambda_1}| \psi_{\lambda_2}\rangle &= 0 
\end{align}
Som gir 
\begin{align}
	 \langle \psi_{\lambda_1}| \psi_{\lambda_2} \rangle = 0 \hspace{1cm}\lambda_1 \neq \lambda_2
\end{align}
at hermitiske operatorer har ortogonale tilstander. Vi har allerede sett at dette gjelder for 
eigentilstandene til Hamilton operatoren, men nå har vi at dette gjelder helt generelt for 
hermitiske oepratorer. Siden vi også krever at tilstandene er normaliserte har vi at eigentilstander
utgjør et ortonormalt sett tilstander $ \{ | \psi_n \rangle  \} $ slik at 
\begin{align}
	 \langle \psi_n| \psi_m \rangle  = \delta_{n,m}
\end{align}
Vi ar også at $ | \psi_n \rangle  $ er et komplett sett 
\begin{align}
	 | \psi \rangle = \sum^{ \infty}_{n=1} c_n | \psi_n \rangle 
\end{align}
for en hvilken som helst tilstand $ | \psi \rangle  $. Når vi antar at $ | \psi \rangle  $ 
er normalisert får vi da 
\begin{align}
	 1 &= \langle \psi| \psi \rangle \\
  &=\Big(\sum^{ \infty}_{m=1} c_m^* \langle \psi_m | \Big)
\sum^{ \infty}_{n=1} c_n | \psi_n \rangle  \\
  &=\sum^{ \infty}_{m=1} \sum^{ \infty}_{n=1} c_m^*c_n \langle \psi_m  | \psi_n \rangle \\
  &=\sum^{ \infty}_{m=1} \sum^{ \infty}_{n=1} c_m^*c_n \delta_{n,m} \\
  &= \sum^{ \infty}_{n=1} |c_n|^2
\end{align}
hvor vi har at $|c_n|^2$ er sannsynligheten for å måle eigenverdien $\lambda_n$ til operatoren 
$ \hat{O} $.

Vi har igjen at vi kan finne vektene til lineærkombinasjonen ved 
\begin{align}
	 \langle \psi_n| \psi \rangle  &= 
	 \langle \psi_n| \Big( \sum_{m=1}^{ \infty} c_m | \psi_m \rangle \Big)  \\
	 &= \sum_{m=1}^{ \infty} c_m  \langle \psi_n|  \psi_m \rangle  \\
	 &= \sum_{m=1}^{ \infty} c_m  \delta_{n,m}  =  c_n
\end{align}
Siden $|C_n|^2$ er sannsynligheten for å måle eigenverdi $\lambda_n$ har vi 
\begin{align}
	 \langle O \rangle  = \sum^{ \infty}_{n=1} |c_n|^2 \lambda_n
\end{align}
Som gir generelt for hermitiske operatorer at 
\begin{align}
	\langle \psi| \hat{O} | \psi \rangle   &= 
  \Big(\sum^{ \infty}_{n=1} c_n  \langle \psi_n| \Big)
  \hat{O}\Big( \sum_{m=1}^{ \infty} c_m | \psi_m \rangle \Big)  \\
 &=\sum^{ \infty}_{n=1}\sum_{m=1}^{\infty}c_n^*c_m \langle \psi_n| \hat{O}| \psi_m \rangle \\
 &=\sum^{ \infty}_{n=1}\sum_{m=1}^{\infty}c_n^*c_m \lambda_m \langle \psi_n|  \psi_m \rangle \\
 &=\sum^{ \infty}_{n=1}\sum_{m=1}^{\infty}c_n^*c_m \lambda_m  \delta_{n,m} \\
 &=\sum^{ \infty}_{n=1}|c_m|^2 \lambda_m 
\end{align}
Slik at vi har
\begin{align}
	\langle O \rangle = \langle \psi| \hat{O} | \psi \rangle  
\end{align}

%Vi ser nå på forventningsverdien til en hermitisk operator $ \hat{O}$ og vi får da 
%\begin{align}
	% \langle O  \rangle &= \int_{- \infty}^{ \infty} \psi^* (\hat{O} \psi) dx \\
			    %&= \int_{- \infty}^{ \infty} (\hat{O}\psi)^* \psi dx \\
			    %&= \Big(\int_{- \infty}^{ \infty} \psi^* (\hat{O} \psi) dx \Big)^*\\ 
			    %&= \langle O  \rangle^* 
%\end{align}
%eller i Dirac notasjon 
Vi får da videre at når vi bruker at $ \hat{O} $ er hermitisk
\begin{align}
	 \langle O  \rangle = \langle \psi| \hat{O}\psi \rangle
 = \langle \psi| \hat{O} | \psi \rangle =
\langle \hat{O} \psi|\psi \rangle = \langle O  \rangle^* 
\end{align}
Med andre ord forventningsverdien til observable som representeres av hermitiske operatorer er 
reelle. Dvs. vi har at alle observerbare størrelser representeres av hermitiske operatorer 
og de vil alltid ha reelle forventningsverdier.
Hvis vi videre antar at vi har en eigenverdiligning for en hermitisk operator 
\begin{align}
	 \hat{O}\psi_{\lambda} = \lambda \psi_{\lambda}
\end{align}
Så får vi da at 
\begin{align}
	 \langle O  \rangle = \langle \psi_\lambda| \hat{O}\psi_{\lambda} \rangle
	 = \lambda \langle \psi_\lambda|\psi_{\lambda} \rangle = \lambda
\end{align}
hvor vi antar at eigentilstandene er normalisert. Dette gir da 
\begin{align}
	 \langle O  \rangle =  \langle O  \rangle^* \hspace{1cm} \Rightarrow \hspace{1cm}
	\lambda = \lambda^*
\end{align}
altså hermitiske operatorer har alltid reelle eigenverdier.

Til slutt nevner vi at vi kan skrive uskarpheten til en hvilken som helst observabel som 
\begin{align}
 (\Delta O)^2 = \langle \psi| \hat{O}^2 | \psi \rangle -  \langle \psi | \hat{O} |\psi \rangle^2
\end{align}


\section*{Kommuterende operatorer}

Vi vet at om vi multipliserer to reelle tall med hverandre så spiller rekkefølgen ingen rolle 
\begin{align}
	 a b = b a \hspace{1cm} \forall a,b \in \mathbb{R}
\end{align}
mens vi også vet at for to matriser spiller rekkefølgen en veldig viktig rolle når 
vi multipliserer de med hverandre og vi har generelt at 
\begin{align}
	 \mathbf{M_1} \mathbf{M_2} \neq \mathbf{M_2} \mathbf{M_1} 
\end{align} 
når $ \mathbf{M_1}$ og $ \mathbf{M_2} $ er generelle matriser og det er ikke nødvendigvis tilfellet 
at $ \mathbf{M_2} \mathbf{M_1} $ eksisterer selv om $ \mathbf{M_1} \mathbf{M_2} $ gjør det.
Vi sier at reelle tall kommuterer mens matriser generelt ikke gjør det. Vi skal se på det 
for generelle operatorer i det følgende.
og vi definerer nå den såkalte \textbf{kommutatoren} som 
\begin{align}
	 [ \hat{A}, \hat{B}] = \hat{A} \hat{B} - \hat{B} \hat{A}
\end{align}
Vi har da at to operatorer kommuterer når kommutatoren forsvinner 
\begin{align}
	 [ \hat{A}, \hat{B}] &= 0 \\
	\Rightarrow \hat{A} \hat{B} &= \hat{B} \hat{A}
\end{align}






\end{document}

