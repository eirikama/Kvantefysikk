\documentclass{article}

\usepackage{standalone}
\usepackage{pgfplots}
\pgfplotsset{compat=newest}
\usepackage{tikz}
\usetikzlibrary{decorations.markings}

\usepackage{subfig}
\usepackage[margin=2.5cm]{geometry}
\usepackage{amsmath}
\usepackage{amssymb}


\newcommand\greybox[1]{
	\vskip\baselineskip
	\par\noindent\colorbox{lightgray}{
		\begin{minipage}{\textwidth}#1\end{minipage}
	}
	\vskip\baselineskip
}

\title{Prinsipper i kvantemekanikk}


\begin{document}
\maketitle

\section*{Paritetsoperatoren}

Vi har allerede sett på Hamilton-, posisjons- og bevegelsesmengde-operatorene. Nå skal 
vi introdusere en ny operator, nemlig paritetsoperatoren. Paritetsoperatoren
er en operator som gjør transformasjonen $x \to -x$, vi definerer den som 
\begin{align}
	 \hat{\Pi}\psi(x) = \psi(-x)
\end{align}
Vi har da følgende eigenverdiligning 
\begin{align}
	 \hat{\Pi}\psi_{\lambda}(x) = \lambda\psi_{\lambda}(x)
\end{align}
Vi kan løse denne ligningen ved å merke oss at $ \hat{\Pi}^2=1$ siden hvis vi transformerer 
$x \to -x$ to ganger er tilbake der vi startet. Vi har
\begin{align}
	 \hat{\Pi}^2 \psi_{\lambda} = \hat{\Pi}\lambda \psi_{\lambda} = \lambda^2 \psi_{\lambda}
\end{align}
som gir 
\begin{align}
	 \lambda^2 &= 1 \\
	\Rightarrow \lambda &= \pm 1
\end{align}
Dvs. eigentilstandene til paritetsoperatoren med eigenverdi $+1$ er jevne tilstander 
og $-1$ eigenverdier har odde tilstander.
Vi har sett at for symmetriske potensialer $V(x) = V(-x)$ er energi eigentilstandene 
også symmetriske, enten jevne eller odde. Så vi har derfor generelt at energi eigentilstander 
er også paritets eigentilstander.

Det er enkelt å vise at alle tilstander kan skrives som en lineærkombinasjon av jevne og odde 
tilstander. Som vi kan se ved å skrive om en arbitrær $\psi(x)$ på følgende måte
\begin{align}
	 \psi(x) = \frac{1}{2}(\psi(x) + \psi(-x)) + \frac{1}{2}(\psi(x) - \psi(-x))
\end{align}
hvor vi nå ser at det første leddet er jevnt mens det andre er odde.
Det betyr at også eigentilstandene til paritetsoperatoren danner et komplett sett.


\section*{Hermitiske operatorer og observable}

Vi starter med litt notasjon; vi innfører notasjonen hvor 
vi skriver en kvante tilstand som $ | \psi \rangle = \psi$ (dette kaller vi en ket) 
og vi lar den kompleks konjugert være $ \langle \psi| = \psi^*$ (som vi kaller en bra).
Fordelen med en slik notasjon er at vi 
kan nå har en veldig kompakt måte å skrive indreproduktet mellom tilstandene 
$\psi_n$ og $\psi_m$ som $ \langle \psi_n | \psi_m \rangle $, dvs vi har 
\begin{align}
	 \langle \psi_n|\psi_m  \rangle = \int_{- \infty}^{ \infty} \psi_n^*\psi_m dx
\end{align}
Dette indreproduktet kalles nå en braket.
Vi har også for en  operator $\hat{O}$ at 
\begin{align}
	 | \hat{O}\psi \rangle &= \hat{O} | \psi \rangle = \hat{O} \psi \\
(| \hat{O}\psi \rangle)^* = \langle \hat{O} \psi|  &=\langle \psi| \hat{O}^*  = \hat{O}^* \psi^*
\end{align}
Denne notasjonen stammer fra Paul Dirac og kalles 
Dirac notasjon, eller bra-ket notasjon, og er en veldig effektiv måte å uttrykke kvantetilstander 
på.

Videre sier vi at en operator $ \hat{O}$ er \textbf{hermitisk} hvis vi har 
\begin{align}
	\langle \phi| \hat{O}\psi \rangle &= \langle \hat{O}\phi| \psi \rangle  
\hspace{0.5cm}	\Leftrightarrow \hspace{0.5cm} \int_{- \infty}^{ \infty} \phi^* \hat{O}\psi dx =
  \int_{- \infty}^{ \infty} (\hat{O}\phi)^* \psi dx 
\end{align}
Vi har videre at om vi har to distinkte eigenverdier til en hermitisk operator 
så er eigentilstandene deres ortogonal, som vi ser fra 
\begin{align}
	 \langle \psi_{\lambda_1}| \hat{O} \psi_{\lambda_2} \rangle 
	&= \langle \hat{O} \psi_{\lambda_1}| \psi_{\lambda_2} \rangle \\
	\Rightarrow \langle \psi_{\lambda_1}|\lambda_2 | \psi_{\lambda_2} \rangle 
	&= \langle \psi_{\lambda_1}| \lambda_1 | \psi_{\lambda_2} \rangle \\
	\Rightarrow (\lambda_2 - \lambda_1) \langle \psi_{\lambda_1}| \psi_{\lambda_2}\rangle &= 0 
\end{align}
Som gir 
\begin{align}
	 \langle \psi_{\lambda_1}| \psi_{\lambda_2} \rangle = 0 \hspace{1cm}\lambda_1 \neq \lambda_2
\end{align}
at hermitiske operatorer har ortogonale tilstander. Vi har allerede sett at dette gjelder for 
eigentilstandene til Hamilton operatoren, men nå har vi at dette gjelder helt generelt for 
hermitiske oepratorer. Siden vi også krever at tilstandene er normaliserte har vi at eigentilstander
utgjør et ortonormalt sett tilstander $ \{ | \psi_n \rangle  \} $ slik at 
\begin{align}
	 \langle \psi_n| \psi_m \rangle  = \delta_{n,m}
\end{align}
Vi ar også at $ | \psi_n \rangle  $ er et komplett sett 
\begin{align}
	 | \psi \rangle = \sum^{ \infty}_{n=1} c_n | \psi_n \rangle 
\end{align}
for en hvilken som helst tilstand $ | \psi \rangle  $. Når vi antar at $ | \psi \rangle  $ 
er normalisert får vi da 
\begin{align}
	 1 &= \langle \psi| \psi \rangle \\
  &=\Big(\sum^{ \infty}_{m=1} c_m^* \langle \psi_m | \Big)
\sum^{ \infty}_{n=1} c_n | \psi_n \rangle  \\
  &=\sum^{ \infty}_{m=1} \sum^{ \infty}_{n=1} c_m^*c_n \langle \psi_m  | \psi_n \rangle \\
  &=\sum^{ \infty}_{m=1} \sum^{ \infty}_{n=1} c_m^*c_n \delta_{n,m} \\
  &= \sum^{ \infty}_{n=1} |c_n|^2
\end{align}
hvor vi har at $|c_n|^2$ er sannsynligheten for å måle eigenverdien $\lambda_n$ til operatoren 
$ \hat{O} $.

Vi har igjen at vi kan finne vektene til lineærkombinasjonen ved 
\begin{align}
	 \langle \psi_n| \psi \rangle  &= 
	 \langle \psi_n| \Big( \sum_{m=1}^{ \infty} c_m | \psi_m \rangle \Big)  \\
	 &= \sum_{m=1}^{ \infty} c_m  \langle \psi_n|  \psi_m \rangle  \\
	 &= \sum_{m=1}^{ \infty} c_m  \delta_{n,m}  =  c_n
\end{align}
Siden $|C_n|^2$ er sannsynligheten for å måle eigenverdi $\lambda_n$ har vi 
\begin{align}
	 \langle O \rangle  = \sum^{ \infty}_{n=1} |c_n|^2 \lambda_n
\end{align}
Som gir generelt for hermitiske operatorer at 
\begin{align}
	\langle \psi| \hat{O} | \psi \rangle   &= 
  \Big(\sum^{ \infty}_{n=1} c_n  \langle \psi_n| \Big)
  \hat{O}\Big( \sum_{m=1}^{ \infty} c_m | \psi_m \rangle \Big)  \\
 &=\sum^{ \infty}_{n=1}\sum_{m=1}^{\infty}c_n^*c_m \langle \psi_n| \hat{O}| \psi_m \rangle \\
 &=\sum^{ \infty}_{n=1}\sum_{m=1}^{\infty}c_n^*c_m \lambda_m \langle \psi_n|  \psi_m \rangle \\
 &=\sum^{ \infty}_{n=1}\sum_{m=1}^{\infty}c_n^*c_m \lambda_m  \delta_{n,m} \\
 &=\sum^{ \infty}_{n=1}|c_m|^2 \lambda_m 
\end{align}
Slik at vi har
\begin{align}
	\langle O \rangle = \langle \psi| \hat{O} | \psi \rangle  
\end{align}

%Vi ser nå på forventningsverdien til en hermitisk operator $ \hat{O}$ og vi får da 
%\begin{align}
	% \langle O  \rangle &= \int_{- \infty}^{ \infty} \psi^* (\hat{O} \psi) dx \\
			    %&= \int_{- \infty}^{ \infty} (\hat{O}\psi)^* \psi dx \\
			    %&= \Big(\int_{- \infty}^{ \infty} \psi^* (\hat{O} \psi) dx \Big)^*\\ 
			    %&= \langle O  \rangle^* 
%\end{align}
%eller i Dirac notasjon 
Vi får da videre at når vi bruker at $ \hat{O} $ er hermitisk
\begin{align}
	 \langle O  \rangle = \langle \psi| \hat{O}\psi \rangle
 = \langle \psi| \hat{O} | \psi \rangle =
\langle \hat{O} \psi|\psi \rangle = \langle O  \rangle^* 
\end{align}
Med andre ord forventningsverdien til observable som representeres av hermitiske operatorer er 
reelle. Dvs. vi har at alle observerbare størrelser representeres av hermitiske operatorer 
og de vil alltid ha reelle forventningsverdier.
Hvis vi videre antar at vi har en eigenverdiligning for en hermitisk operator 
\begin{align}
	 \hat{O}\psi_{\lambda} = \lambda \psi_{\lambda}
\end{align}
Så får vi da at 
\begin{align}
	 \langle O  \rangle = \langle \psi_\lambda| \hat{O}\psi_{\lambda} \rangle
	 = \lambda \langle \psi_\lambda|\psi_{\lambda} \rangle = \lambda
\end{align}
hvor vi antar at eigentilstandene er normalisert. Dette gir da 
\begin{align}
	 \langle O  \rangle =  \langle O  \rangle^* \hspace{1cm} \Rightarrow \hspace{1cm}
	\lambda = \lambda^*
\end{align}
altså hermitiske operatorer har alltid reelle eigenverdier.

Til slutt nevner vi at vi kan skrive uskarpheten til en hvilken som helst observabel som 
\begin{align}
 (\Delta O)^2 = \langle \psi| \hat{O}^2 | \psi \rangle -  \langle \psi | \hat{O} |\psi \rangle^2
\end{align}


\section*{Kommuterende operatorer}

Vi vet at om vi multipliserer to reelle tall med hverandre så spiller rekkefølgen ingen rolle 
\begin{align}
	 a b = b a \hspace{1cm} \forall a,b \in \mathbb{R}
\end{align}
mens vi også vet at for to matriser spiller rekkefølgen en veldig viktig rolle når 
vi multipliserer de med hverandre og vi har generelt at 
\begin{align}
	 \mathbf{M_1} \mathbf{M_2} \neq \mathbf{M_2} \mathbf{M_1} 
\end{align} 
når $ \mathbf{M_1}$ og $ \mathbf{M_2} $ er generelle matriser og det er ikke nødvendigvis tilfellet 
at $ \mathbf{M_2} \mathbf{M_1} $ eksisterer selv om $ \mathbf{M_1} \mathbf{M_2} $ gjør det.
Vi sier at reelle tall kommuterer mens matriser generelt ikke gjør det. Vi skal se på det 
for generelle operatorer i det følgende.
og vi definerer nå den såkalte \textbf{kommutatoren} som 
\begin{align}
	 [ \hat{A}, \hat{B}] = \hat{A} \hat{B} - \hat{B} \hat{A}
\end{align}
Vi har da at to operatorer kommuterer når kommutatoren forsvinner 
\begin{align}
	 [ \hat{A}, \hat{B}] &= 0 \\
	\Rightarrow \hat{A} \hat{B} &= \hat{B} \hat{A}
\end{align}
Vi ser nå på et eksempel på to kommuterende operatorer, nemlig Hamilton operator for harmonisk 
oscillator og paritetsoperatoren
\begin{align}
 [ \hat{H}, \hat{\Pi}] \psi(x) 
&= ( -\frac{\hbar^2}{2m}\frac{\partial^2}{\partial x^2} 
+ \frac{1}{2}m\omega^2 x^2  ) \hat{\Pi} \psi(x) - \hat{\Pi} 
( -\frac{\hbar^2}{2m} \frac{\partial^2}{\partial x^2} + \frac{1}{2}m\omega^2 x^2)\psi(x) \\
&= ( -\frac{\hbar^2}{2m}\frac{\partial^2}{\partial x^2} + \frac{1}{2}m\omega^2 x^2  )\psi(-x) - 
( -\frac{\hbar^2}{2m} \frac{\partial^2}{\partial (-x)^2} + \frac{1}{2}m\omega^2 (-x)^2)\psi(-x) \\
&= ( -\frac{\hbar^2}{2m}\frac{\partial^2}{\partial x^2} + \frac{1}{2}m\omega^2 x^2  )\psi(-x) - 
( -\frac{\hbar^2}{2m} \frac{\partial^2}{\partial x^2} + \frac{1}{2}m\omega^2 x^2)\psi(-x) \\
&= 0
\end{align}
hvor vi ser at Hamilton og paritetsoperatoren kommuterer. Dette er generelt tilfellet for 
operatorer som er symmetriske, altså hvor $V(x) = V(-x)$, og vi har at Hamilton operator er 
invariant under transformasjonen $x\to -x$.

Vi ser nå på implikasjoner av at to operatorer $ \hat{A} $ og $ \hat{B} $ kommuterer. 
Vi har da eigenverdi ligningen for $ \hat{A} $ 
\begin{align}
	 \hat{A}\psi_a = a \psi_a
\end{align}
og lar $ \hat{B} $ virke på dette 
\begin{align}
	 \hat{B}\hat{A}\psi_a &= \hat{B}a \psi_a \\
	\Rightarrow \hat{A} (\hat{B}\psi_a) &= a(\hat{B} \psi_a)
\end{align}
siden $ \hat{A} \hat{B} = \hat{B} \hat{A} $ for kommuterende operatorer. Dvs. at for 
$ \hat{B}\psi_a$ er en eigenfunksjon for $ \hat{A} $.

Det er nå to muligheter å betrakte: enten finnes det en eigenfunksjon  $\psi_a$ med eigenverdi 
$ a $, da kaller vi den \textbf{ikke-degenerert} eigenfunksjon. Vi har da at siden 
$ \hat{B}\psi_a $ også er en eigenfunksjon av $ \hat{A}$ med eigenverdi $ a $ så må det være 
tilfellet at $ \hat{B}\psi_a$ og $\psi_a$ bare er forskjellige med en konstant. Vi kaller denne 
konstanten for $b$ og har 
\begin{align}
	 \hat{B}\psi_a = b\psi_a
\end{align}
hvor vi ser at $\psi_a$ også må være en eigenfunksjon av $ \hat{B} $. Det betyr at kommuterende 
operatorer har felles eigenfunksjoner, men ikke nødvendigvis de samme eigenverdiene.
Dette så vi allerede for Hamilton operatoren til H.O. og paritetsoperatoren,
hvor vi så at energi eigenfunksjonene også er eigenfunksjon til paritetsoperatoren siden 
de er symmetrisk.
Det som følger ifra dette er at vi kan vite verdien til både $ \hat{A} $ og $ \hat{B}$ samtidig,
siden vi vet at om systemet er i tilstanden $ \psi_a $ så vet vi at verdien av $ \hat{A} $ er 
$ a $ og  $ \hat{B} $ er $ b $.

Videre har vi om det finnes flere eigenfunksjoner med eigenverdi $ a $ sier vi at 
eigenfunksjonen er \textbf{degenerert}.
Vi kan se på en fri partikkel som et eksempel. Der har vi at det finnes 2 energi 
eigenfunksjoner med eigenverdi $E = \hbar^2 k^2 / 2m$, nemlig 
\begin{align}
	 \psi_{E_1}(x) &= A\cos(kx) \\
	 \psi_{E_2}(x) &= B\sin(kx) 
\end{align}
vi her da en 2-veis degenerasjon.
Det er dog ikke slik at energi eigenfunksjonene er eigenfunksjoner til 
bevegelsesmengde-operatorene siden 
\begin{align}
	 \hat{p} A\cos(kx) = \frac{\hbar}{i}kA\sin(kx)\\
	 \hat{p} B\sin(kx) = -\frac{\hbar}{i}kB\cos(kx)
\end{align}
Men vi kan dog lage en lineærkombinasjon av energi eigentilstander som er eigenfunksjon til 
bevegelsesmengde-operatoren som 
\begin{align}
 \psi_{p_1}(x) &= \psi_{E_1} + i\frac{A}{B} \psi_{E_2} = A\cos(kx) + iA\sin(kx) = A e^{ikx} \\
 \psi_{p_2}(x) &= \frac{B}{A}\psi_{E_1} - i \psi_{E_2} = B\cos(kx) - iB\sin(kx) = B e^{-ikx}
\end{align}
Her har vi nå at $ \psi_{p_1} $ og $ \psi_{p_2} $ både er eigenfunksjoner med  $E=\hbar^2k^2 /2m$ 
og til bevegelsesmengde-operatoren med eigenverdi $\hbar k$ og $-\hbar k$.
Dette gjelder generelt, og vi har at om $ \hat{A} $ har degenererte eigenfunksjoner er det 
alltid mulig å finne en lineærkombinasjon av eigenfunksjonene som er simulatan eigenfunksjon 
$ \hat{B} $ hvis $ \hat{A} $ og $ \hat{B} $ kommuterer.


\section*{Ikke-kommuterende operatorer og uskarphetsrelasjoner}
Vi ser nå på hva det betyr at to operatorer ikke kommuterer. Vi har at kommutatoren til to  
operatorer er selv en operator og vi har at om to hermitiske operatorer ikke kommuterer slik at 
\begin{align}
	 [ \hat{A}, \hat{B}] = i \hat{C}
\end{align}
og $ \hat{C} $ er en hermitisk operator så har vi en uskarphetsrelasjon 
\begin{align}
	 \Delta A \Delta B \ge \frac{ |\langle C  \rangle| }{2} 
\end{align}
Vi minner oss om at vi har
\begin{align}
	 (\Delta A)^2 &=  \langle (\hat{A} - \langle A  \rangle)^2  \rangle
	=\langle A^2 \rangle - \langle A  \rangle^2 \\
	 (\Delta B)^2 &=  (\langle \hat{B} - \langle B  \rangle)^2  \rangle  =
	\langle B^2 \rangle - \langle B  \rangle^2 
\end{align}
Vi definerer da følgende operatorere
\begin{align}
	 \hat{U} &= \hat{A} - \langle A  \rangle \\
	 \hat{V} &= \hat{B} - \langle B  \rangle 
\end{align}
som gir oss 
\begin{align}
	 \langle \psi | \hat{U}^2\psi \rangle = (\Delta A)^2 \\
	 \langle \psi | \hat{V}^2\psi \rangle = (\Delta B)^2
\end{align}
Vi har videre at 
\begin{align}
	 [ \hat{U}, \hat{V}] = i \hat{C}
\end{align}
slik at 
\begin{align}
	 \langle \psi | [ \hat{U}, \hat{V}] \psi \rangle = 
	 \langle \psi | i \hat{C} \psi \rangle = i \langle C  \rangle 
\end{align}
Videre vet vi at for en compleks funksjon $\phi$ har vi at 
\begin{align}
	 \langle \phi | \phi \rangle \ge 0
\end{align}
Hvis vi da definerer vølgende komplekse funksjon
\begin{align}
	 \phi = \hat{U}\psi + i\lambda \hat{V}\psi
\end{align}
hvor vi ser på $\phi$ som en funksjon av $ \lambda \in \mathbb{R} $ og vi betrakter
\begin{align}
	 I(\lambda) = \langle \phi | \phi \rangle 
\end{align}
Siden $ \hat{A} $ og  $ \hat{B} $ er hermitiske er også $ \hat{U} $ og $ \hat{V} $ og vi har 
\begin{align}
	 I(\lambda) &=  \int_{- \infty}^{ \infty} (\hat{U}\psi + i\lambda \hat{V}\psi)^*
	(\hat{U}\psi + i\lambda \hat{V}\psi) dx \\
	&=  \int_{- \infty}^{ \infty} \psi^* (\hat{U}^2 + \hat{V}^2 \lambda 
	 + i\lambda [ \hat{U}, \hat{V}] ) \psi dx \\
	&= (\Delta U)^2 + \lambda^2(\Delta V)^2 - \lambda \langle C  \rangle  \ge 0
\end{align}
Vi minimiserer så $I(\lambda)$ mhp $\lambda$ som vi vet at vi gjør med å derivere 
\begin{align}
	 \frac{\partial I}{\partial \lambda} = 2\lambda (\Delta V)^2 - \langle C  \rangle 
\end{align}
slik at $I(\lambda)$ er minimal når den derivert er lik 0 altså når
\begin{align}
	 \lambda_{min} = \frac{ \langle C  \rangle }{2 (\Delta V)^2}
\end{align}
Hvis vi nå setter inn i uttrykket for $I(\lambda)$ og får
\begin{align}
  I(\lambda_{min})
  &= (\Delta U)^2 
   + \frac{ \langle C  \rangle^2 }{4 (\Delta V)^4}(\Delta V)^2 
   - \frac{ \langle C  \rangle }{2 (\Delta V)^2} \langle C  \rangle  \\
  &= (\Delta U)^2 
   + \frac{ \langle C  \rangle^2 }{4 (\Delta V)^2} 
   - \frac{ \langle C  \rangle^2 }{2 (\Delta V)^2} \\
  &= (\Delta U)^2 
   - \frac{ \langle C  \rangle^2 }{4 (\Delta V)^2} \ge 0 
\end{align}
slik at vi har 
\begin{align}
	  (\Delta U)^2 (\Delta V)^2 =  \frac{ \langle C  \rangle^2 }{4 } 
\end{align}
Hvis vi tar roten av dette har vi utledet den generelle uskarphetsrelasjonen
\begin{align}
	  \Delta U\Delta V =  \frac{ |\langle C  \rangle| }{2} 
	  =  \frac{ | [ \hat{U}, \hat{V} ] | }{2} 
\end{align}


Vi ser nå at dette holder for 
bevegelsesmengde-operatorene og posisjons-operatoren, hvor vi har 
\begin{align}
 [ \hat{x}, \hat{p}] \psi &= [ x, \frac{\hbar}{i} \frac{\partial}{\partial x} ] \psi \\
 &=  x \frac{\hbar}{i} \frac{\partial}{\partial x} \psi 
 - \frac{\hbar}{i} \frac{\partial}{\partial x} (x\psi) \\
 &=  x \frac{\hbar}{i} \frac{\partial}{\partial x} \psi 
 - \frac{\hbar}{i} \psi - x \frac{\hbar}{i} \frac{\partial}{\partial x}\psi \\
 &= i\hbar \psi 
\end{align}
hvor $\psi$ er en arbitrær bølgefunksjon, og vi merker oss at bevegelsesmengde-operatorene 
deriverer alt som kommer til høyre for den, altså også posisjons-operatoren. Dette må vi generelt 
passe på når vi jobber med operatorer.
Vi får da videre 
\begin{align}
	 \Delta x \Delta p &\ge \frac{|[ \hat{x}, \hat{p}] | }{2} \\
 &= \frac{| i\hbar | }{2} = \frac{\hbar}{2} \\
\end{align}
Som vi allerede har sagt at er Heisenbergs uskarphetsrelasjon.


\section*{Tidsutvikling}
Vi skal nå se litt nærmere på tidsutviklingen til et kvantemekasnisk system. Vi vet 
at tidsutviklingen er bestemt av S.L. 
\begin{align}
	 \hat{H}\Psi = i\hbar \frac{\partial \Psi}{\partial t} 
\end{align}
Vi ser da på hvordan forventningsverdier utvikler seg i tid 
\begin{align}
	 \frac{\partial \langle A  \rangle }{\partial t} = \frac{\partial }{\partial t} 
	\int_{- \infty}^{ \infty} \Psi^* \hat{A} \Psi dx
\end{align}
Fra S.L. har vi at 
\begin{align}
	\frac{\partial \Psi}{\partial t} &= \frac{1}{i\hbar} \hat{H}\Psi \\ 
	\frac{\partial \Psi^*}{\partial t} &= -\frac{1}{i\hbar} (\hat{H}\Psi)^*
\end{align}
og vi vet at Hamilton operatoren er hermitisk, dvs $ \langle \hat{H}\Psi|\Psi \rangle  
= \langle\Psi| \hat{H}\Psi \rangle $. Vi finner da at 
\begin{align}
	 \frac{\partial \langle A  \rangle }{\partial t} &=
	\int_{- \infty}^{ \infty}  
	\frac{\partial \Psi^*}{\partial t} \hat{A} \Psi 
	+ \Psi^* \frac{\partial \hat{A}}{\partial t}  \Psi 
	+ \Psi^*\hat{A} \frac{\partial \Psi}{\partial t}  dx \\
	&= \int_{- \infty}^{ \infty}  
	-\frac{1}{i\hbar} (\hat{H}\Psi)^* \hat{A} \Psi 
	+ \Psi^* \frac{\partial \hat{A}}{\partial t}  \Psi 
	+ \Psi^*\hat{A} \frac{1}{i\hbar} \hat{H}\Psi  dx\\
	&= \int_{- \infty}^{ \infty}  
	-\frac{1}{i\hbar}( \Psi^* \hat{H}\hat{A} \Psi 
	- \Psi^*\hat{A}  \hat{H}\Psi  )
	+ \Psi^* \frac{\partial \hat{A}}{\partial t}  \Psi dx \\
	&= \frac{i}{\hbar} \int_{- \infty}^{ \infty}  
	 \Psi^* [\hat{H}, \hat{A} ] \Psi   dx
	+ \int_{- \infty}^{ \infty}  \Psi^* \frac{\partial \hat{A}}{\partial t}  \Psi dx
\end{align}
Hvor vi ser at generelt så bestemmer kommutatoren med Hamilton operatoren hvordan 
forventningsverdien til en observabel utvikler seg i tid. Videre har vi at om operatoren selv 
er tidsuavhengig, som veldig ofte er tilfellet, så får vi 
\begin{align}
	 \frac{\partial \langle A  \rangle }{\partial t} 
	= \frac{i}{\hbar} \int_{- \infty}^{ \infty}  
	 \Psi^* [\hat{H}, \hat{A} ] \Psi   dx
\end{align}
altså endringen i tid til forventningsverdien er forventningsverdien til kommutatoren med 
Hamilton operatoren. Vi har da videre at om en operator kommuterer med Hamilton operatoren 
så er det en konservert størrelse.


\subsection*{Energi-tid uskarphetsrelasjonen}

En uskarphetsrelasjoner med energi of tid er konseptuelt litt merkelig, siden tid ikke er en 
observabel i ikke-relativistisk kvantemekanikk. Vi ser på den generelle uskarphetsrelasjonen 
mellom Hamilton operatoren og en arbitrær operator $ \hat{A}$ som 
\begin{align}
	 \Delta H \Delta A  = \frac{| \langle C  \rangle |}{2}
\end{align}
hvor $ \hat{C} = -i [ \hat{A}, \hat{H}]$. Vi har da at 
\begin{align}
  | \langle C  \rangle | = |i\int_{- \infty}^{ \infty} \Psi^* [ \hat{A}, \hat{H}] \Psi dx
			 = |\hbar \frac{\partial \langle A  \rangle }{\partial t} |
\end{align}
Dette gir oss da 
\begin{align}
	 \Delta H \frac{\Delta A}{|\frac{\partial \langle A  \rangle }{\partial t} |} 
	\ge \frac{\hbar}{2} 
\end{align}
Vi definerer nå 
\begin{align}
	 \Delta t =  \frac{\Delta A}{|\frac{\partial \langle A  \rangle }{\partial t} |} 
\end{align}
slik at vi får 
\begin{align}
	 \Delta H \Delta t \ge \frac{\hbar}{2}
\end{align}




\end{document}

