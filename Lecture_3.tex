\documentclass{article}

\usepackage{standalone}
\usepackage{pgfplots}
\pgfplotsset{compat=newest}
\usepackage{tikz}
\usetikzlibrary{decorations.markings}

\usepackage{subfig}
\usepackage[margin=2.5cm]{geometry}
\usepackage{amsmath}
\usepackage{amssymb}


\newcommand\greybox[1]{
	\vskip\baselineskip
	\par\noindent\colorbox{lightgray}{
		\begin{minipage}{\textwidth}#1\end{minipage}
	}
	\vskip\baselineskip
}

\title{Forelesning 3}


\begin{document}
\maketitle


\section{Den tidsuavhengige Schrödingerligningen}

Vi skal nå i dette kapitellet massere S.L. litt slik at vi den blir enklere å jobbe med framover.
Vi starter med å skille tids- og posisjons-avhengigeten fra hverandre.

\subsection{Separasjon av variable}


Vi har S.L. som 
\begin{align}
 - \frac{\hbar^2}{2m}\frac{\partial^2 \Psi(x,t)}{\partial x^2} + V(x) \Psi(x,t) 
	= i\hbar \frac{\partial \Psi(x,t)}{\partial t} 
\end{align}
vi har allerede løst ligningen for tilfellet med $V(x)$, altså for en fri partikkel. Vi skal i 
det følgende løse S.L. for ikke trivielle potensialer.
Når vi antar (som vi nesten alltid gjør) at potensialet er tidsuavhengige ($V(x)=0$) 
så kan vi bruke separasjon av variabler til å løse ligningen. Dette er noe vi ofte gjør 
når vi kan skrive ligningen slik at all posisjons-avhengigeten til operatorene er på venstre og all 
tids-avhengigeten på høyre side av lighetstegnet.

Det gjøres ved å anta følgende Ansatz
\begin{align}
	 \Psi(x,t) = \psi(x)f(t)
\end{align}
dvs. vi antar at vi kan dekomponere bølgefunksjonen i en del som varierer som en funksjon av 
posisjon og en del som funksjon av tid.
Vi setter inn i S.L.
\begin{align}
 - \frac{\hbar^2}{2m}\frac{\partial^2}{\partial x^2} [\psi(x)f(t)] + V(x) \psi(x)f(t) 
	&= i\hbar \frac{\partial }{\partial t} [\psi(x)f(t)] \\
	\Rightarrow f(t)[-\frac{\hbar^2}{2m}\frac{\partial^2}{\partial x^2} \psi(x) + V(x) \psi(x)] 
	&= \psi(x) i\hbar \frac{\partial }{\partial t} f(t) \\
   \Rightarrow \frac{1}{\psi(x)} [-\frac{\hbar^2}{2m}\frac{\partial^2}{\partial x^2} \psi(x) 
   + V(x) \psi(x)] &= \frac{i\hbar}{f(t)}  \frac{\partial }{\partial t} f(t) \\
\end{align}
Vi har nå fått ligningen på en form slik at tids of posisjons-avhengigeten er på hver sin side av 
likhetstegnet. Den eneste måten at denne ligningen kan være oppfylt på helt generelt er om begge
sidene er lik en konstant. Vi kaller denne konstant $E$ siden vi skal assosiere den med partikkelens
energi, vi ser allerede at den må ha samme enhet som $V(x)$ som vi vet har enhet energi. Vi har  
\begin{align}
    \frac{1}{\psi(x)} [-\frac{\hbar^2}{2m}\frac{\partial^2}{\partial x^2} \psi(x) 
   + V(x) \psi(x)] = E
\end{align}
og 
\begin{align}
	   \frac{i\hbar}{f(t)}  \frac{\partial }{\partial t} f(t) = E
\end{align}
Den tidsavhengige ligningen er her veldig grei å løse og trengs bare å løse en gang for alle $V(x)$.
Vi har at 
\begin{align}
	 f(t) = f(0) e^{ -i \frac{Et}{\hbar} }
\end{align}
Vi absorberer $f(0)$ inn i $\psi(x)$, som vi normerer etterhvert uansett, og vi har at den tidsavhengige løsningen er 
\begin{align}
	 \Psi(x,t) = \psi(x)e^{-i \frac{Et}{\hbar}}
\end{align}
Så vi har nå redusert problemet til  å løse en ordinær differensialligning, hhilket er mye lettere 
enn partielle differensialligninger.


\end{document}
