\documentclass{article}

\usepackage{standalone}
\usepackage{pgfplots}
\pgfplotsset{compat=newest}
\usepackage{tikz}
\usetikzlibrary{decorations.markings}

\usepackage{subfig}
\usepackage[margin=2.cm]{geometry}
\usepackage{amsmath}
\usepackage{amssymb}


\newcommand\greybox[1]{
	\vskip\baselineskip
	\par\noindent\colorbox{lightgray}{
		\begin{minipage}{\textwidth}#1\end{minipage}
	}
	\vskip\baselineskip
}


\title{Plenary Exercises 3}


\begin{document}
\maketitle

\greybox{
\underline{\large \textbf{Exercise 1}} \\ 
Consider the following situation: A particle moving in one dimensional is described
by the wavefunction
\begin{align}
	 \psi(x) = C x e^{\frac{-x^2}{2x_ 0 ^2}}
\end{align}
where $C$ and $x_0$ are constants. \\ 

\hspace{0.5cm}\textbf{a)} Normalize the wavefunction $\psi(x)$. \\ 

\hspace{0.5cm}\textbf{b)} Calculate the uncertainty $\Delta x$ in the position of the particle. \\

\hspace{0.5cm}\textbf{c)} Where is the probability of finding the particle the largest? \\ 

\hspace{0.5cm}\textbf{d)} How much more or less probable is it to find the particle at $0.02 x_0$ as compared
at $0.01x_0$?

}

\hspace{0.5cm}\textbf{a)} \\ 

We start by using the normalization condition to determine $A$. We have that 
\begin{align}
	 \int_{-\infty}^{\infty} |\psi(x)|^2 dx =   \int_{-\infty}^{\infty} \psi(x)\psi(x)^* dx = 1  
\end{align}
We insert for our wavefunction
\begin{align}
	 \int_{-\infty}^{\infty} |C|^2 x^2 e^{(\frac{-x^2}{2x_ 0 ^2} + \frac{-x^2}{2x_ 0 ^2})} dx 
	&= |C|^2   \int_{-\infty}^{\infty} x^2 e^{\frac{-x^2}{x_ 0 ^2} } dx \\ 
	&= |C|^2  \frac{\sqrt{\pi} (\frac{1}{x_0^2})^{-\frac{3}{2}}}{2} \\ 
	&= |C|^2  \frac{\sqrt{\pi}x_0^3}{2}
\end{align}
where we have used that we know how to integrate the Gaussian over $\mathbb{R}$ as
\begin{align}
	 \int_{-\infty}^{\infty} x^2 e^{-\beta x^2} dx = \frac{1}{2}\sqrt{\pi}\beta^{-\frac{3}{2}}
\end{align}
We now get that
\begin{align}
	|C| = \frac{\sqrt{2}}{\pi^{\frac{1}{4}}x_0^{\frac{3}{2}}} 
\end{align}


\hspace{0.5cm}\textbf{b)} \\ 
We know that 
\begin{align}
	(\Delta x)^2 = \langle x^2 \rangle -  \langle x \rangle^2 
\end{align}
We calculate the expectation values to determine the uncertainty
\begin{align}
	 \langle x \rangle = \int_{-\infty}^{\infty} x |\psi(x)|^2 dx = 0
\end{align}
since we know that $|\psi(x)|^2$ is an even function, which then means that the integrand is odd 
and therefore evaluates to zero. Furthermore we have
\begin{align}
	 \langle x^2 \rangle = \int_{-\infty}^{\infty} x^2 |\psi(x)|^2 dx &= 
	  |C|^2 \int_{-\infty}^{\infty} x^4 e^{\frac{-x^2}{x_ 0 ^2} }dx \\
	&= |C|^2 \frac{3}{4}\sqrt{\pi}x_0^{5} \\
	&= \frac{2}{\sqrt{\pi}}\frac{1}{x_0^{3}} \frac{3}{4}\sqrt{\pi}x_0^{5} \\
	&=  \frac{3}{2}x_0^{2} 
\end{align}
This then give the uncertainty in the position
\begin{align}
	 \Delta x = \sqrt{\langle x^2  \rangle - \langle x \rangle^2} = \frac{3}{2}x_o^3 
\end{align}

\textbf{c)} \\ 
We consider the probability distribution for the position of the particle
\begin{align}
	 p_x(x) &= x |\psi(x)|^2 \\
		&= |C|^2 x_0^{\frac{3}{2}}x^2 e^{-\frac{x^2}{x_0^2}}
\end{align}
To find the position which is maximally likely to find the particle at we find the maxima of the distribution $p_x(x)$
by differentiating and finding the roots of the derivative. We differentiate
\begin{align}
	 \frac{d }{dx} p_x(x) 
	&= C \frac{d}{dx} \big( x^2 e^{-\frac{x^2}{x_0^2}} \big) \\
	&= C \big( 2xe^{-\frac{x^2}{x_0^2}}  - \frac{2x^3}{x_0^2} e^{-\frac{x^2}{x_0^2}}  \big) \\
\end{align}
Setting this to zero gives
\begin{align}
	C 2xe^{-\frac{x^2}{x_0^2}} &= C\frac{2x^3}{x_0^2}e^{-\frac{x^2}{x_0^2}} \\
	\Rightarrow \hspace{0.82cm} 2x &= \frac{2x^3}{x_0^2} \\ 
	\Rightarrow \hspace{1cm} x &= x_0
\end{align}
Where we note that the solution $x=0$ must be disregarded since we have that $|\psi(0)|^2 = 0$. \\

\textbf{d)} \\ 
We consider the ratio of the probability of finding the particle $0.01x_0$ and $0.02x_0$.
\begin{align}
	 \frac{p_x(0.02x_0)}{p_x(0.01x_0)} 
	&= \frac{|C|^2 x_0^{\frac{3}{2}}(0.02x_0)^2 e^{-\frac{(0.02x_0)^2}{x_0^2}}}{|C|^2 x_0^{\frac{3}{2}}(0.01x_0)^2 e^{-\frac{(0.01x_0)^2}{x_0^2}}} \\
	&= \frac{(0.02)^2 e^{-(0.02)^2}}{(0.01)^2 e^{-(0.01)^2}}\\  
	&= \frac{(2\times 10^{-2})^2 }{(1\times 10^{-2})^2} e^{-(0.02)^2 + (0.01)^2} \approx 4
\end{align}
where we have used that $\exp(-0.02^2 + 0.01^2) \approx 1$. We have thus found that it is four times as likely to find
particle at $x=0.02x_0$ as at $x=0.01x_0$.



%%%%%%-----------------------------------------------------------------------------------------------%%%%%%%%



\greybox{
\underline{\large \textbf{Exercise 2}} \\ 

An electron moves in an one-dimensional potential 
\begin{align}
	 V(x) = 
	\begin{cases}
		V_0 \big[ \frac{1 - \cos(\frac{x}{a})}{1 + \cos(\frac{x}{a})}\big] 
		\;\;\;\;\;\; ,x \in (-\pi a, \pi a) \\
		\infty \;\;\;\;\;\;\;\;\;\;\;\;\;\;\;\;\;\;\;\;\;\;  ,x \notin (-\pi a, \pi a)
	\end{cases}
\end{align}
where
\begin{align}
	 \alpha = 1Å = 10^{-10}m
\end{align}
The electron is described with the wavefunction
\begin{align}
	 \psi(x) =
	\begin{cases}
		 C(1 + \cos(\frac{x}{a}) \hspace{1cm} ,x \in (-\pi a, \pi a)  \\ 
		0 \hspace{3cm} ,x \notin (-\pi a, \pi a)
	\end{cases}
\end{align}

\vspace{.5cm}

\hspace{0.5cm}\textbf{a)}
Draw the shape of $V(x)$. Find values of $V_0$ and $E$ such that $\psi(x)$ is a solution to
Schrödinger's equation for the potential $V(x)$. Provide the answer in eV. \\

\hspace{0.5cm}\textbf{b)}
Find the interval in which the electron can move in classical theory, that is where $V(x) \le E$.
}


\textbf{a)} \\ 
We start by sketching the potential in Fig.~\ref{fig:potential}. We have that the potential is zero at $x=0$ and goes 
symmetrically towards infinity when $x$ goes towards $\pm a$.


\begin{figure*}
	\centering
	\documentclass[border=2mm]{standalone}
\usepackage{pgfplots}
\pgfplotsset{compat=newest}
\usepackage{tikz}
\usetikzlibrary{patterns}


\begin{document}
\begin{tikzpicture}

\tikzset{
	hatch distance/.store in=\hatchdistance
	hatch distance=10pt
	hatch thickness/.store in=\hatchdistance
	hatch thickness=2pt
}

  \begin{axis}%
    [grid=both,
     minor tick num=5,
     xlabel=$x$,
     ylabel=$V(x)$,
     grid style={line width=.1pt, draw=gray!10},
     major grid style={line width=.2pt,draw=gray!50},
     axis lines=middle,
     restrict y to domain=0:15.,
     enlargelimits={abs=1.12}
    ]
    \addplot[domain=-3.14:3.14,samples=50,smooth,red!70] {(1-cos(deg(x)))/(1+cos(deg(x)))} 
    node[] (sin);
    \path[name path=axis] (3, 0) -- (3, 3);
    %\addplot+[mark=none, 
	    %domain=-1:1,
	    %samples=50,
	    %hatch distance=5pt,
	    %hatch thickness=0.5pt,
	    %draw=red!70,
	    %pattern color=blue,
	    %area legend] {(1-cos(deg(pi*x)))/(1+cos(deg(pi*x)))} \closedcylcle;
  \end{axis}
\end{tikzpicture}
\end{document}

	\caption{Sketch of the potential $V(x)$, for $\alpha = 1$.}
	\label{fig:potential}
\end{figure*}

We find values of $V$ and $E$ for which the wavefunction $\psi(x)$ is a solution to the Scrödinger equation.
For reference we have Schrödinger's equation as

\begin{align}
	 -\frac{\hbar^2}{2m}\frac{d^2}{dx^2}\psi(x) = (E - V(x))\psi(x)
\end{align}
We insert for our potential and get
\begin{align}
	& -\frac{\hbar^2}{2m}\frac{d^2}{dx^2} C(1 + \cos(\frac{x}{a})  
	= \big(E -V_0 \big[ \frac{1 - \cos(\frac{x}{a})}{1 + \cos(\frac{x}{a})}\big] \big) C(1 + \cos(\frac{x}{a})  \\
	& \Rightarrow \hspace{1cm} \frac{\hbar^2}{2ma^2} \cos(\frac{x}{a})
	= E (1 + \cos(\frac{x}{a})) - V_0 (1 - \cos(\frac{x}{a}))  \\
	& \Rightarrow \hspace{1cm} (\frac{\hbar^2}{2ma^2} - E - V_0) \cos(\frac{x}{a})  
	= E - V_0   
\end{align}
where we see that Schrödinger's equation is satisfied for all $x$ if
\begin{align}
	 E = V_0
\end{align}
and 
\begin{align}
	 \frac{\hbar^2}{2ma^2} - V_0 - E &= 0 \\
	\Rightarrow \hspace{1cm} V_0 = E_0 &= \frac{\hbar}{4ma^2} \\
          &= \frac{(6.63 \times 10^{-34}Js)^2}{4\times (2\pi)^2 \times 9.11\times 10^{-21}kg \times 10^{-20}m^2} \\ 
	  &= (2\pi)^2 \times 1.2 \times 10^{-17}J \\ 
	  &= 1.9 eV
\end{align}


\textbf{b)} \\ 
We know that a classical particle must comply with
\begin{align}
	 E \ge V(x) 
\end{align}
This gives for our particle, where we know that we have $E=V_0$, the following interval which the particle can move
within classical theory
\begin{align}
	V_0 &\ge V_0  \big[ \frac{1 - \cos(\frac{x}{a})}{1 + \cos(\frac{x}{a})}\big] \\
	\Rightarrow \hspace{0.6cm} 1 + \cos(\frac{x}{a}) &\ge  1 - \cos(\frac{x}{a}) \\ 
	\Rightarrow \hspace{1cm} 2 \cos(\frac{x}{a}) &\ge  0  
\end{align}
That is we need to have the argument of the cosine function be smaller in absolute value than $\pi / 2$.
Which gives 
\begin{align}
	 |x| < \frac{\pi a}{2}
\end{align}



%%%%%------------------------------------------------------------------------------%%%%%%


\greybox{
\underline{\large \textbf{Exercise 3}} \\ 
we assume a particle is described by the following wavefunction
\begin{align}
	 \psi(x) = 
	\begin{cases}
		 0 \hspace{2cm} ,x \in (-\infty, -a) \\ 
		 A(a - |x|)  \hspace{0.6cm} ,x \in (-a, a) \\ 
		 0 \hspace{2cm} ,x \in (a, \infty) \\ 
	\end{cases}
\end{align}

\hspace{0.5cm} \textbf{a)} Determine $A$ by using the normalization condition. \\ 

\hspace{0.5cm} \textbf{b)} What is the probability of finding the particle in the
region $[0, a/2]$? \\ 

\hspace{0.5cm} \textbf{c)} Calculate $\langle x \rangle$ and $\langle x \rangle^2$ for this 
state.
}

\hspace{0.5cm} \textbf{a)} \\ 

We start by using the normalization condition to determine $A$. This gives 
\begin{align}
	 1 &= \int_{-\infty}^{\infty} \psi(x)\psi(x)^* \\
	   &= \int_{-a}^{a} |A|^2 (a - |x|)^2 dx
\end{align}
We may now utilize that the integrand is an even function to get
\begin{align}
	\int_{-a}^{a} |A|^2 (a - |x|)^2 dx &= 2|A|^2 \int_{0}^{a}  (a - x)^2 dx  \\ 
					   &= -2|A|^2 \int_{a}^{0}  y^2  dy  \\ 
					   &= 2|A|^2 \big[ -\frac{1}{3} y^3 \big]_a^0 \\
					   &= 2|A|^2 \frac{1}{3} a^3 
\end{align}
Which gives us 
\begin{align}
	 A = \pm \sqrt{\frac{3}{2 a^2}} 
\end{align}


\hspace{0.5cm} \textbf{b)} \\ 

We find the probability of finding the particle within $[0, a / 2]$ by integrating the probability 
amplitude over the region as
\begin{align}
	 \int_{0}^{\frac{a}{2}} |\psi(x)|^2 dx &=  \int_{0}^{\frac{a}{2}} |A|^2 (a - |x|)^2 dx \\ 
					       &= |A|^2 \int_{0}^{\frac{a}{2}} ( a^2 + x^2 - 2ax ) dx  \\
					       &= |A|^2 \big[ a^2x + \frac{x^3}{3} - ax^2 \big]_0^{\frac{a}{2}} \\
					       &= |A|^2 a^3 ( \frac{1}{2} + \frac{1}{24} - \frac{1}{4} ) \\
					       &= \frac{3}{2a^2} a^3 \frac{7}{24} \\
					       &= \frac{7}{16} 
\end{align}
That is the probability of finding the particle in the region between zero and $\frac{a}{2}$ is roughly $0.5$. \\ 


\hspace{0.5cm} \textbf{c)} \\ 

We now find the expectation value of the position as
\begin{align}
	 \langle x \rangle &= \int_{-\infty}^{\infty} x |\psi(x)|^2 dx \\ 
		&= 0
\end{align}
where we have made use of that $x$ is an odd function and $|\psi(x)|^2$ is even and thus the integrand 
is odd and evaluates to zero. 

We find the expectation value of the position squared, where we can make use of that the integrand will be odd to simplify our calculations
\begin{align}
	 \langle x \rangle^2 &= \int_{-\infty}^{\infty} x^2 |\psi(x)|^2 dx \\ 
			     &=  |A|^2 \int_{-a}^{a} x^2 (a - |x|)^2 dx \\ 
			     &=  2|A|^2 \int_{0}^{a} x^2 (a - x)^2 dx  \\
			     &=  2|A|^2 \int_{0}^{a} ( x^4 + a^2x^2 - 2ax^3 )  dx  \\
			     &=  2|A|^2 \big[ \frac{x^5}{5} + \frac{a^2x^3}{3} - \frac{2ax^4}{4} \big]_0^{a}  \\
			     &=  2|A|^2 ( \frac{a^5}{5} + \frac{a^5}{3} - \frac{a^5}{2} ) \\
			     &=  2 \frac{3}{2a^3} \frac{a^5}{30} \\ 
			     &=  \frac{a^2}{10} 
\end{align}



\vspace{1cm}
\textbf{Solving Integral from Lecture} \\


We shall now solve the following integral 
\begin{align}
	 \psi(x) = \frac{\sqrt{\sigma}}{\sqrt{2\pi^{\frac{3}{2}}}}  
	\int_{-\infty}^{\infty} e^{-(k-k_0)^2\frac{\sigma^2}{2}}
	e^{i(kx - \omega t)} dx
\end{align}
and we shall do this by first showing that this integral can be rewritten to the form
\begin{align}
	 \psi(x) = \frac{\sqrt{\sigma}}{\sqrt{2\pi^{\frac{3}{2}}}} e^{i(k_0 x - \omega_0 t)} 
	\int_{-\infty}^{\infty} e^{-(k-k_0)^2(\frac{\sigma^2}{2} + \frac{it\hbar}{2m})}
	e^{i(k-k_0)(x - \frac{k_0\hbar t}{m})} dx
\end{align}
We are going to be making use of $\omega = \hbar k^2 / 2m$ and rewrite the exponent
\begin{align}
	 & i(k_0 - \omega_0) - (k - k_0)^2(\frac{\sigma^2}{2} + \frac{it\hbar}{2m}) + i(k-k_0)(x - \frac{k_0\hbarr t}{m}) \\
	= & i(k_0x - \omega_0t) - k^2  \frac{it\hbar}{2m} + 2kk_0 \frac{it\hbar}{2m} - k_0^2 \frac{it\hbar}{2m} - (k-k_0)^2\frac{\sigma^2}{2} + i(k-k_0)x - \frac{ik k_0 \hbar t}{m} + \frac{ik_0^2 \hbar t}{m}	 \\ 
	= & i(k_0x - \omega_0t) - i\omega t + 2kk_0 \frac{it\hbar}{2m} - i\omega_0 t - (k-k_0)^2\frac{\sigma^2}{2} + i(k-k_0)x - \frac{ik k_0 \hbar t}{m}  + 2i\omega_0 t \\ 
	= & -(k-k_0)^2\frac{\sigma^2}{2} + i(kx-\omega t) 
\end{align}
which is what we would we wanted to show. We now make the following substitutions
\begin{align}
	 y &= k - k_0 \\ 
	a &= \frac{1}{\sqrt{\frac{\sigma^2}{2} + \frac{i\hbar t}{2m}}}  \\ 
	 b &= x - \frac{k_0 \hbar t}{m}
\end{align}
which gives
\begin{align}
	 \psi(x,t) = \frac{\sqrt{\sigma} }{\sqrt{2\pi^{\frac{3}{2}}} } e^{i(k_0x - \omega_0t)} 
	\int_{-\infty}^{\infty} e^{-(\frac{y}{a})^2}e^{iyb}dy
\end{align}
We can now use that we know 
\begin{align}
	\int_{-\infty}^{\infty} e^{-\frac{y^2}{a^2} + iby} du = \sqrt{\pi} a e^{-(\frac{ab}{2})^2} 
\end{align}
when we assume that $a^2> 0$. 
To show this we need that we know that 
\begin{align}
	 \int_{-\infty}^{\infty} e^{-(x - i \Lambda)^2} = \sqrt{\pi} 
\end{align}
We can use this result and rewrite the integrand by intropducing new intration variable $y=ax$ and define a new constant $b = \frac{2\Lambda}{a}$. This gives 
\begin{align}
	 \int_{-\infty}^{\infty} e^{-x^2 + \Lambda^2 + 2 x i \Lambda} dx 
	&= e^{\Lambda^2} \int_{-\infty}^{\infty} e^{-x^2  + 2 x i \Lambda} dx \\
	&= e^{\frac{a b }{2}} a^{-1} \int_{-\infty}^{\infty} e^{-(\frac{y}{a})^2  +  ib y  } dy = \sqrt{\pi} 
\end{align}
Inverting this gives what we wanted to show
\begin{align}
	\sqrt{\pi} a e^{-(\frac{ab}{2})^2} =  \int_{-\infty}^{\infty} e^{-(\frac{y}{a})^2 + iby} dy 
\end{align}
We now use this result by first noting that we have
\begin{align}
	 \big(\frac{ab}{2}\big)^2 = \big(\frac{1}{2}\frac{x - \frac{k_0 \hbar t}{m}}{\sqrt{\frac{\sigma ^2}{2} + \frac{i \hbar t}{2m}} }\big)^2
\end{align}
which gives us
\begin{align}
	 \psi(x,t) &= \frac{\sqrt{\sigma}}{\sqrt{2}\pi^{\frac{3}{2}}} e^{i(k_0x - \omega_0 t)} \sqrt{\pi}
	\frac{1}{\sqrt{\frac{\sigma ^2}{2} + 
	\frac{i\hbar t}{2m}} }e^{-\frac{1}{4}\frac{(x-\frac{k_0 t}{m})^2}{\frac{\sigma^2}{2} + \frac{i\hbar t}{2m}} } \\ 
	&= \frac{\sqrt{\sigma}}{\sqrt{\sigma ^2 + \frac{i\hbar t}{m}} \sqrt{\pi}} 
	e^{i(k_0x - \omega_0 t)} e^{-\frac{1}{2}\frac{(x-\frac{k_0 t}{m})^2}{\sigma^2 + \frac{i\hbar t}{m}}}
\end{align}
Which is our desired expression for the wavefunction. We now calculate its absolute square as
\begin{align}
	 |\psi(x,t)|^2 &= \frac{\sigma}{\sqrt{(\sigma^2 + \frac{i\hbar t}{m})(\sigma^2-\frac{i\hbar t}{m})} \sqrt{\pi}} 
	e^{-\frac{1}{2}\frac{(x-\frac{k_0 t}{m})^2}{\sigma^2 + \frac{i\hbar t}{m}}}  
	e^{-\frac{1}{2}\frac{(x-\frac{k_0 t}{m})^2}{\sigma^2 - \frac{i\hbar t}{m}}} \\ 
&= \frac{1}{\sqrt{\sigma^3 + \frac{\hbar t^2}{\sigma^2m^2}} \sqrt{\pi}} 
	e^{-\frac{(x-\frac{k_0 t}{m})^2}{\sigma^2 + \frac{\hbar^2 t^2}{\sigma^2m^2}}}  
\end{align}
We now see that we have 
\begin{align}
	 |\psi(x, t=0)|^2 = \frac{1}{\sqrt{\pi} \sigma}e^{-\frac{x^2}{\sigma^2}}
\end{align}
Where we see that this is equivalent to the stationary wave packet. Thus it must also hold that the uncertainty will be the same
\begin{align}
	 (\Delta x(t=0))^2 = \frac{\hbar}{2}
\end{align}
We can certainly calculate the uncertainty for the time-dependant wave packet through standard formulae, but we 
see that the probability distribution are the same if we make the change of variables
\begin{align}
	 \sigma^2 \to  \sigma^2 + \frac{\hbar^2 t^2}{\sigma^2 m^2}
\end{align}
This means that the uncertainty must be
\begin{align}
	 \Delta x(t) = \frac{1}{\sqrt{2}} \sqrt{\sigma^2 + \frac{\hbar^2t^2}{\sigma^2 m^2}}  
\end{align}
For the uncertainty in momentum we have that initially we have 
\begin{align}
	 \Delta p(t=0) = \frac{\hbar}{\sqrt{2} \sigma}
\end{align}
and if we consider the time dependent wavefunction in momentum space
\begin{align}
	 \psi(x) = \frac{\sqrt{\sigma}}{\sqrt{2\pi^{\frac{3}{2}}}}  
	\int_{-\infty}^{\infty} e^{-(k-k_0)^2\frac{\sigma^2}{2}}
	e^{i(kx - \omega t)} dx
\end{align}
we see that the part of the wave function which represents the width of the wave packet in momentum space is 
invariant in time. Since the only time dependence is in the phase factor. Therefore it must hold that the time
dependent uncertainty is the same as at $t=0$ for all times.
\begin{align}
	 \Delta p(t) = \frac{\hbar}{\sqrt{2} \sigma}
\end{align}
We now have that 
\begin{align}
	 \Delta p(t)  \Delta x(t) = \frac{\hbar}{2}\sqrt{1 + \frac{\hbar^2 t^2}{\sigma^{4}m^2}} 
\end{align}


\end{document}



