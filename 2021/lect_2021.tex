\documentclass{article}

\usepackage{standalone}
\usepackage{pgfplots}
\pgfplotsset{compat=newest}
\usepackage{tikz}
\usetikzlibrary{decorations.markings}

\usepackage{subfig}
\usepackage[margin=2.5cm]{geometry}
\usepackage{amsmath}
\usepackage{amssymb}


\newcommand\greybox[1]{
	\vskip\baselineskip
	\par\noindent\colorbox{lightgray}{
		\begin{minipage}{\textwidth}#1\end{minipage}
	}
	\vskip\baselineskip
}

\title{Lecture Quantum Physics \\ 
				The Time-independent Schrödinger Equation}


\begin{document}
\maketitle


\section{The Time-independent Schrödinger Equation}


We are going to be considering the Schrödinger equation 
\begin{align}
	\big[-\frac{\hbar^2}{2m} \nabla^2 + V(x, t) \big] \Psi(x, t) 
	= i\hbar  \frac{\partial}{\partial t} \Psi(x, t)
\end{align}
We are going to show that this can be separated into a time dependent and a time-independent 
part if the potential is time-independent, i.e. if $V(x, t) = V(x)$. We will use a standard 
approach for solving such differential equations called separation of variables. 
This is done by making the following Ansatz
\begin{align}
	 	\Psi(x,t) = \psi(x)f(t)
\end{align}
We insert this into Schrödinger's equation and get
\begin{align}
	 \big[-\frac{\hbar^2}{2m} \nabla^2 + V(x)\big] \psi(x) f(t) 
	= i\hbar  \frac{\partial}{\partial t} \big[ \psi(x)f(t) \big]
\end{align}
We can now use that $\nabla^2$ only act on $\psi(x)$ and then divide through by
$\psi(x)f(t)$ to get all the dependency on $x$ on the left-hand side and the time-dependence on the
left-hand side of the equality
\begin{align}
	 f(t)\big[-\frac{\hbar^2}{2m} \nabla^2 \psi(x) + V(x) \psi(x)\big] 
	&= i\hbar \psi(x) \frac{\partial}{\partial t} f(t) \\
	\Rightarrow \hspace{0.7cm}	 \frac{1}{\psi(x)}\big[ \frac{\hbar^2}{2m} \nabla^2 \psi(x) + V(x) \psi(x)\big]  
	&=  \frac{1}{f(t)} i\hbar \frac{\partial}{\partial t} f(t) \\ 
\end{align}
We now have that the two sides of the equation can vary completely independent of each other
and that can only be satisfied if they are both equal to a constant. That is, we have
\begin{align}
	\frac{1}{\psi(x)}\big[ \frac{\hbar^2}{2m} \nabla^2 \psi(x) + V(x) \psi(x)\big]  
	&= E = \frac{1}{f(t)} i\hbar \frac{\partial}{\partial t} f(t)
\end{align}
We now get two differential equations to solve
\begin{align}
	 \frac{i\hbar}{f(t)} \frac{\partial f(t)}{\partial x}  &= E \\ 
	\frac{1}{\psi(x)}\big[ \frac{\hbar^2}{2m} \nabla^2 \psi(x) + V(x) \psi(x)\big]  &= E
\end{align}
The equation depending on the time can be solved generally for all time-independent potentials. We have 
\begin{align}
	 \frac{\partial f(t)}{\partial t} = -\frac{iE}{\hbar} f(t)
\end{align}
We see immediately that we thus have
\begin{align}
	 f(t) = f(0) e^{-\frac{iE}{\hbar}t}
\end{align}
So, we see that for all time-independent potentials we have the time-dependence of the wavefunction
is a mere phase factor. Which will therefore often not influence expectation values of
observables, which simplifies matters significantly. To see this we write the wavefunction
\begin{align}
	 \Psi(x, t) = \psi(x) e^{-\frac{iEt}{\hbar}} 
\end{align}
where we see that the absolute square of this wavefunction is only functional on position
\begin{align}
	 |\Psi(x,t)|^2 = \psi^*(x)e^{\frac{iEt}{\hbar}} \psi(x)e^{-\frac{iEt}{\hbar}} = |\psi(x)|^2
\end{align}
Furthermore, we have named the separation constant $E$, and with good cause; it turns out that we can
associate the separation constant with the energy of the state described by the wavefunction $\Psi(x,t)$.
It also holds that 
\begin{align}
	 E = \hbar \omega
\end{align}
which allows us to write the time dependence as $f(t) = \exp(-i\omega t)$.

The equation for the position-dependent part of the wavefunction $\psi(x)$ must be calculated independently 
for each $V(x)$ under consideration. The wavefunction to the time-independent Schrödinger equation 
are often called stationary states.
We shall in the following encounter several such potentials, 
and calculate their resulting stationary state wavefunctions.


\section{Particle in a Box}


\subsection{Solving Time-independent Equation}

In this section we are going to encounter the first potential for which we shall solve Schrödinger's
equation in all its glory. We are going to consider the potential describing a particle in a box,
where the potential is described as
\begin{align}
	 V(x) = 
	\begin{cases}
		 0 			\hspace{1cm} ,x \in (0, L) \\ 
		\infty \hspace{1cm} ,x \notin (0, L)
	\end{cases}
\end{align}
where we have a box of size $L$, inside which the potential is zero and at the edges of the
potential abruptly grows infinitely large. In a classical setting where we have forces
$F = -\frac{\partial V}{\partial x}$ and we thus have an infinitely large force acting on the 
particle at the border of the box, meaning that it is confined to the region inside of the box.
This is a very interesting example, which is thoroughly 
simplified but we can still learn many fundamental properties of quantum mechanics from it. \\ 


We shall solve the time independent Schrödinger's equation for this potential inside of the box
\begin{align}
	 -\frac{\hbar^2}{2m}\frac{d^2\psi}{dx^2} = E\psi
\end{align}
since we know that $V=0$ inside of the box. Typically, we introduce the parameter
\begin{align}
	 k^2 = \frac{2mE}{\hbar^2}
\end{align}
which then yields the following
\begin{align}
	 \frac{d^2 \psi}{dx^2} = -k^2 \psi
\end{align}

The differential equation for the particle inside of the box we immediately recognize as the 
classical equation for a mass on a spring. We have that the wavefunction must be a function 
which twice differentiated remains the same up to a constant and a negative sign.
We know that the trigonometric function where we have that 
\begin{align}
	 \frac{d^2}{dx^2} \sin(kx) &= -k^2 \sin(kx) \\
	 \frac{d^2}{dx^2} \cos(kx) &= -k^2 \cos(kx)
\end{align}
Thus the general possible solution to the differential equation for the particle inside the is
\begin{align}
	 \psi(x) = A\sin(kx) + B\cos(kx)
\end{align}
At this point it is wortwhile to point out that we generally want that a second order differential 
equation such as this one should have solutions which are continuous and with continuous 
derivatives. For our particle in a potential box this is not quite satisfied, since we have singularities in
our potential $V(x)$, which leads to it being impossible for the derivatives to be continuous everywhere. 
To see this we consider Schrödinger's equation for a general potential 
\begin{align}
	 \frac{d^2\psi(x)}{dx^2} = - \frac{2m}{\hbar^2}(E - V(x))\psi(x)
\end{align}
where we can now integrate over a some small region, which we now take to be around $x=0$.
\begin{align}
	 \int_{0^-}^{0^+}\frac{d^2 \psi(x)}{dx^2} dx = \int_{0^-}^{0^+} - \frac{2m}{\hbar^2}(E - V(x))\psi(x) dx 
\end{align}
to which we can apply the fundamental theorem of calculus to, to get
\begin{align}
	\frac{d \psi(x)}{dx}\Big|_{0^+}-\frac{d \psi(x)}{dx}\Big|_{0^-}  
	= - \frac{2m}{\hbar^2}\int_{0^-}^{0^+} (E - V(x))\psi(x) dx 
\end{align}
Here we see that the lefthand side needs to be zero for the derivative to be continuous, and for most potentials
it would. However, for our case since we have the the potential is infinite in the limit coming from the left and 
zero in the limit coming from the right, it cannot be the case that both $\psi(x)$ and $d\psi(x)/ dx$ are 
continuous. \\


If there were no constraint on the solution for the particle in the box we would have that $k = 2\pi / \lambda$ where 
$\lambda$ would be the wavelength. This was also the reason for setting $2mE / \hbar^2 = k^2$.
We have off course that the particle is confined to the box which means that $\psi(x) = 0$ for 
$x\notin (0,L)$. We now have to apply the appropriate boundary conditions to our solutions. 
This means we have to path together the solutions outside and inside of the box to ensure that
the overall wavefunction is continuous. This also ensures that the probability of finding the 
particle is not allowed to be discontinuous, which is reassuring.

In particular we must have that $\psi(0) = \psi(L) = 0$ which means that 
\begin{align}
	 \psi(0) = A\sin(0) + B \cos(0) = B = 0
\end{align}
and
\begin{align}
	 \psi(L) = A\sin(kx) = 0
\end{align}
Where we disregard of the trivial solution $A=0$ since the wavefunction would then be zero 
everywhere, ie. there is no particle in the box.
Therefore, we consider the solution with
\begin{align}
	 kL = n\pi \hspace{0.5cm} , n \in \mathbb{N_+}
\end{align}
Thus the allowed values of
\begin{align}
	 k_n = \frac{n \pi}{L}
\end{align}
and we have the energies of the system as
\begin{align}
	 E_n = \frac{\hbar^2k_n^2}{2m} = \frac{n^2\hbar^2\pi^2}{2mL^2}
\end{align}
and the corresponding wavefunctions are
\begin{align}
	 \psi_n(x) = A_n \sin(kx)
\end{align}
We determine the amplitude $A_n$ through the normalization condition as
\begin{align}
	 \int_{-\infty}^{\infty} |\psi(x)|^2 dx &= \int_{0}^{L} |A_n|^2 \sin^2(\frac{n\pix}{L}) dx \\
											&= |A_n|^2 \int_{-\infty}^{\infty} \frac{1}{2}\big(1 - \cos(\frac{2n\pi x}{L}) \big) dx \\
											&= |A_n|^2 \big[ x - \frac{L}{2n\pi}\sin(\frac{2n\pi x}{L}) \big]_0^{L} \\ 
											&= \frac{|A_n|^2 L}{2} \\ 
											&\overset{!}{=} 1
\end{align}
where we have used that since 
\begin{align}
	 \sin(\alpha)\sin(\beta) = \frac{1}{2}\cos(\alpha - \beta) - \frac{1}{2}\sin(\alpha + \beta)
\end{align}
it holds that 
\begin{align}
	 \sin(\alpha)^2 = \frac{1}{2}(1 - \cos(2\alpha))
\end{align}
This means that we have $A_n = \sqrt{2 / L}$, and we thus have the wavefunction as 
\begin{align}
	 \psi_n(x) = 
	\begin{cases}
		 \sqrt{\frac{2}{L}} \sin(\frac{n\pi x}{L}) \hspace{1cm} ,x \in (0, L) \\ 
		 0  \hspace{3cm} ,x \notin (0, L)  
	\end{cases}
\end{align}


\subsection{Properties of Particle in Box}
There are a couple of things to note about the solutions of the particle in the box above whichare very
illustrative for quantum mechanics as a whole. Firstly, we see that the energies are quantized which is in
start contrast to classical mechanics where you can have any energies you please. This is a general feature
of quantum mechanics, where you always have energy states which are quantized, at least for confined systems.
In the particle in the box-case it was demanding that the wavefunctions must go to zero at the edges of the 
box which gave rise to the quantization of the wavefunctions and energies, and we shall see later on that
imposing boundary conditions is instrumental in determining the energy levels. \\ 

Second, we can easily see that the particles wavefunction is reminiscent of the modes of the vibration on a string.
However, there are significant difference between the quantum particle in a box and vibrational modes no a string.
For the particle it is appropriate to associate the energy with the frequency as $E = \hbar \omega$, whereas 
for the waves on a string the energy is dependant on the amplitude. \\ 

Furthermore, it is interesting to note that there is no $E=0$ solution, since also the ground state has to oscillate.
There cannot be a particle with zero energy, which from a classical point of view should be perfectly possible.
This could also be argued for with Heisenberg's uncertainty principle; where we have that if the particle had
zero energy and thus zero momentum, it would also have zero uncertainty in momentum, but Heisenberg's uncertainty
principle states that if $\Delta p = 0$ it would have had to be the case that $\Delta x = \infty$, but this cannot 
be the case since the particle must be in  $(0, L)$. Thus $\Delta p$, and thus the energy must be non-zero. \\ 



\subsection{Time Dependence}
The allowed energy states are stationary, but that is not the whole picture, since there is no dynamics in a
state. We shall now illustrate how dynamics can arise in such a system. To this end we start by considering the
following wavefunction 
\begin{align}
	 \psi(x) = \frac{1}{\sqrt{2}}\psi_1(x) + \frac{1}{\sqrt{2}}\psi_2(x)	
\end{align}
which is a normalized superposition of the ground state and the first excited state. The dynamics is now generated
by the appropriate phase factors $\exp(-iE_nt / \hbar)$ for each $\psi_n$, which we found by solving the time-part of Schrödinger's equation. For our above superposition this gives
\begin{align}
	 \Psi(x, t) &= \frac{1}{\sqrt{2}} \Psi_1(x,t) + \frac{1}{\sqrt{2}} \Psi_2(x,t) \\ 
					 &=  \frac{e^{\frac{-iE_1t}{\hbar}}}{\sqrt{2}} \psi_1(x) + \frac{e^{\frac{-iE_2t}{\hbar}}}{\sqrt{2}} \psi_2(x) 
\end{align}
This gives the probability amplitude as
\begin{align}
	 |\Psi(x, t)|^2 &= \Psi(x,t )^* \Psi(x,t) \\ 
									&=  \big(\frac{e^{\frac{iE_1t}{\hbar}}}{\sqrt{2}} \psi_1(x)^*
									+ \frac{e^{\frac{iE_2t}{\hbar}}}{\sqrt{2}} \psi_2(x)^*\big) 
									\big(\frac{e^{\frac{-iE_1t}{\hbar}}}{\sqrt{2}} \psi_1(x)
									+ \frac{e^{\frac{-iE_2t}{\hbar}}}{\sqrt{2}} \psi_2(x)\big) \\
									&=  e^{\frac{iE_1t}{\hbar}}\big(\frac{1}{\sqrt{2}} \psi_1(x)^*
									+ \frac{e^{\frac{i(E_2-E_1)t}{\hbar}}}{\sqrt{2}} \psi_2(x)^*\big) 
									e^{\frac{-iE_1t}{\hbar}}\big(\frac{1}{\sqrt{2}} \psi_1(x)
									+ \frac{e^{\frac{-i(E_2 - E_1)t}{\hbar}}}{\sqrt{2}} \psi_2(x)\big) \\
									&=  \big(\frac{1}{\sqrt{2}} \psi_1(x)^* 
									+ \frac{e^{\frac{i(E_2-E_1)t}{\hbar}}}{\sqrt{2}} \psi_2(x)^*\big) 
									\big(\frac{1}{\sqrt{2}} \psi_1(x)
									+ \frac{e^{\frac{-i(E_2 - E_1)t}{\hbar}}}{\sqrt{2}} \psi_2(x)\big) \\
									&=  \frac{1}{2} |\psi_1(x)|^2 + \frac{1}{2} |\psi_2(x)|^2 
									+ \frac{e^{\frac{i(E_2-E_1)t}{\hbar}}}{2} \psi_2(x)^*\psi_1(x)
									+ \frac{e^{\frac{-i(E_2 - E_1)t}{\hbar}}}{2} \psi_1^*(x)\psi_2(x) \\ 
									&=  \frac{1}{2} \psi_1(x)^2 + \frac{1}{2} \psi_2(x)^2 
									+ \cos(\frac{(E_2-E_1)t}{\hbar}) \psi_1(x)\psi_2(x)
\end{align}
Where we have made use of the fact that we in this particular case know that $\psi_n \in \mathbb{R}$ and 
that we know from Eulers identity $\cos(x) = (e^{ix} + e^{-ix})/2$. We thus see that the probability
density is harmonic in time. Furthermore, the wavefunctions vary their interference patterns in time 
and we can see that by comparing the wavefunction at $t=0$ and $t=\pi\hbar/(E_2-E_1)$.
\begin{align}
	 \Psi(x, t=0) &= \frac{1}{\sqrt{2}}\psi_1(x) + \frac{1}{\sqrt{2}}\psi_2(x) \\	
	 \Psi(x, t=\frac{\pi\hbarr}{E_2-E_1}) &= e^{\frac{-iE_1t}{\hbar}}
	\big( \frac{1}{\sqrt{2}}\psi_1(x) - \frac{1}{\sqrt{2}}\psi_2(x)	\big)
\end{align}
When we note that the phase factor cancels out when considering the probability amplitude, we 
see that dynamics evolves the wavefunction to be different superpositions of stationary states $\psi_1$ and
$\psi_2$.


\subsection{Statistical Interpretation of Quantum Mechanics}

We saw in the previous section an example of a wavefunction which was a linear combination 
two specific stationary states. We shall now try to generalize this slightly and we consider a 
completely general superposition of stationary states 
\begin{align}
	 \psi(x) = \sum_{n=1}^{\infty} c_n \psi_n(x)
\end{align}
and if we consider the general time-dependent wavefunction we have the time-dependence for
every stationary state as $\Psi_n(x,t) = \psi(x) exp(-iEt / \hbar)$ which gives
\begin{align}
	 \Psi(x,t) = \sum_{n=1}^{\infty} c_n \psi(x) e^{-\frac{iEt}{\hbar}}
\end{align}
We shall in the following argue for there being a very neat analogy between this expansion of the 
wavefunction $\psi(x)$	in the set of stationary states $\{\psi_n(x)\}$ and the expansion of a
vector $\mathbf{V}$ in the unit vectors $ \mathbf{\hat{i}_x}$, $\mathbf{\hat{i}_y}$ 
and $\mathbf{\hat{i}_z}$, which we assume to be e.g. the Cartesian unit vectors. 
We then have that any given general 
$ \mathbf{V} \in \mathbb{R}^3$ can be written as 
\begin{align}
	 			\mathbf{V} = \mathbf{ \hat{i}_x} V_x+ \mathbf{ \hat{i}_y} V_y + \mathbf{ \hat{i}_z} V_z 
\end{align}
where we know that the unit basis vectors of the Cartesian coordinate system have that 
\begin{align}
	 \mathbf{ \hat{i}_n } \cdot 	 \mathbf{ \hat{i}_m } = \delta_{n,m},   \hspace{1cm} n,m \in [x,y,z]
\end{align}
That is, we have that $ \mathbf{ \hat{i}_x } \cdot \mathbf{ \hat{i}_x } = \mathbf{ \hat{i}_y } 
\cdot\mathbf{ \hat{i}_y } =  \mathbf{ \hat{i}_z } \cdot 	 \mathbf{ \hat{i}_z } = 0$
and  $ \mathbf{ \hat{i}_z } \cdot \mathbf{ \hat{i}_x } = \mathbf{ \hat{i}_y } 
\cdot\mathbf{ \hat{i}_x } =  \mathbf{ \hat{i}_z } \cdot 	 \mathbf{ \hat{i}_y } = 1$
Very similar to this we have shown for the stationary states, which we assumed to have been normalized, 
that it holds 
\begin{align}
	 \int_{-\infty}^{\infty} \psi_n(x)^* \psi_n(x) dx = 1
\end{align}
since we demand normalized wavefunctions for it to be interpretable as a probability distribution.
It turns out there is also an orthogonality relation for the stationary states.
\begin{align}
	 \int_{-\infty}^{\infty} \psi_n(x)^* \psi_m(x) dx = 0,  \hspace{1cm} (n \neq m)
\end{align}
This relation can be shown to be true for the stationary states for the particle in an infinite 
well which we considered in the previous section, where we assume $n\neq m$ and we then have
\begin{align}
	 \int_{-\infty}^{\infty}  \psi_n^*(x) \psi_m(x) dx 
	&= \frac{2}{L} \int_{0}^{L}  \sin(\frac{n\pi x}{L}) \sin(\frac{m\pi x}{L}) dx \\
	&= \frac{1}{L} \int_{0}^{L} \big[ \cos(\frac{(n-m)\pi x}{L}) - \cos(\frac{(n+m)\pi x}{L})  \big] dx\\
	&= \frac{1}{L} \big[\frac{L}{(n-m)\pi} \sin(\frac{(n-m)\pi x}{L}) - 
											\frac{L}{(n+m)\pi} \sin(\frac{(n+m)\pi x}{L})  \big]_0^L \\
	&= \frac{\sin((n-m)\pi)}{(n-m)\pi}  - \frac{\sin((n+m)\pi )}{(n+m)\pi}   \\ 
	&= 0
\end{align}
where the final result comes from that we know that $m \pm n \in \mathbb{N}$ and for any integer times $\pi$ the
sine function evaluates to zero. We have also made use of the trigonometric identity 
\begin{align}
	 \frac{1}{2} (\cos(\alpha - \beta)) - \cos(\alpha + \beta)) = \sin(\alpha)\sin(\beta)
\end{align}
in the second line above.
We can succinctly write the orthogonality and normalization condition as the orthonormal it condition
\begin{align}
	 \int_{-\infty}^{\infty} \psi_n^*(x) \psi_m(x) = \delta_{n,m}
\end{align}
where we have defined the Kroenecker-delta 
\begin{align}
	 \delta_{n,m} =
	\begin{cases}
		 0, \hspace{1cm} n \neq m \\
		 1, \hspace{1cm} n  =  m
	\end{cases}
\end{align}
The stationary states have another property in common with the basis vectors  $ \mathbf{\hat{i}_x}$, $\mathbf{\hat{i}_y}$ 
and $\mathbf{\hat{i}_z}$, and that is completeness. Completeness means that every vector in the given space can be expanded 
as a linear combination of the complete set  $\{\mathbf{\hat{i}_x}, \mathbf{\hat{i}_y}, \mathbf{\hat{i}_z}\}$,
as we know we can for every vector in Cartesian 3D space.

The equivalent for this for our stationary states is that every possible allowed wavefunction is expressible as a linear
combination of the stationary states as 
\begin{align}
	 \psi(x) = \sum_{n=1}^{\infty} c_n \psi_n(x)
\end{align}
and since this holds for every possible wavefunction, we call the set of stationary states complete. 

This is a little finicky to prove, so we shall simply assume this to hold, and we shall shortly see that this is essential 
to the theory of quantum mechanics. For our particle in the box-example we have the superposition as
\begin{align}
	 \psi(x) = \sum_{n=1}^{\infty} c_n \sqrt{\frac{2}{L}} \sin(\frac{n\pi x}{L})
\end{align}
which we might also recognize as a standard Fourier series. \\ 

We shall now consider how we can find the coefficients, given that we know the wavefunction. We start by considering the 
Cartesian linear combination $ \mathbf{V} = \mathbf{ \hat{i}_x} V_x+ \mathbf{ \hat{i}_y} V_y + \mathbf{ \hat{i}_z} V_z $.
If we want to find the coefficient of a particular basis vector, we simply take the dot-product between the expansion $ \mathbf{V} $ and the corresponding basis vector as 
\begin{align}
	 \mathbf{ \hat{i}_y } \cdot \mathbf{V}  &= 	 \mathbf{ \hat{i}_y } \cdot  \mathbf{ \hat{i}_x} V_x
	+	\mathbf{ \hat{i}_y } \cdot  \mathbf{ \hat{i}_y} V_y + 	 \mathbf{ \hat{i}_y } \cdot  \mathbf{ \hat{i}_z} V_z = V_y \\
	 \mathbf{ \hat{i}_x } \cdot \mathbf{V}  &= V_x \\ 
	 \mathbf{ \hat{i}_z } \cdot \mathbf{V}  &= V_z  
\end{align}
where we use the orthogonality of the basis vectors. 

In a very similar manner we can use the orthogonality of the stationary states of a quantum system to find any given 
wavefunctions expansions coefficients in an orthonormal set of states. The procedure for finding the coefficients 
$c_n$'s is conceptually the same, but instead of taking the dot-product we multiply by the $\psi^*_n$ corresponding to 
the coefficient which we wish to determine and integrate. That is to determine any given coefficient $c_n$ in the expansion 
\begin{align}
	\psi(x) = \sum_{n=1}^{\infty} c_n \psi_n(x)
\end{align}
we calculate 
\begin{align}
	 \int_{-\infty}^{\infty} \psi^*_n(x) \psi(x) &= \int_{-\infty}^{\infty} \psi^*_n(x) \sum_{m=1}^{\infty} c_m \psi_m(x) dx \\ 
	 &= \sum_{m=1}^{\infty}c_m   \int_{-\infty}^{\infty} \psi^*_n(x)  \psi_m(x) dx \\
	 &= \sum_{m=1}^{\infty}c_m  \delta_{n,m} \\
	 &= 1
\end{align}
The coefficients $\{c_n\}$ play a very important role in quantum mechanics. Assuming $\psi_n(x)$ to be normalized, as well as
the wavefunction itself, we get 
\begin{align}
	 \int_{-\infty}^{\infty} |\psi(x)|^2	dx 
	&= \int_{-\infty}^{\infty} \big( \sum_{n=1}^{\infty} c_n^* \psi_n^*(x) \big) 
	\big( \sum_{m=1}^{\infty} c_m \psi_m(x) \big) dx \\ 
	&=  \big( \sum_{n=1}^{\infty} c_n^* \big) 
	\big( \sum_{m=1}^{\infty} c_m\big) \int_{-\infty}^{\infty} \psi_n^*(x) \psi_m(x)  dx \\ 
	&= ( \sum_{n=1}^{\infty} c_n^* \big)\big( \sum_{m=1}^{\infty} c_m\big) \delta_{n,m} \\
	&=  \sum_{m=1}^{\infty} c_n^*c_n = \sum_{m=1}^{\infty} 	|c_n|^2 \\
	&\overset{!}{=} 1
\end{align}
Where we in the last line used that $\psi(x)$ should be normalized. That is, we have 
\begin{align}
	 \sum_{n=!}^{\infty} |c_n|^2 = 1
\end{align}
We interpret this result as the absolute square of the coefficients being probabilities and we write 
\begin{align}
	 |c_n|^2 = P_n
\end{align}
where the probabilities $P_n$ are for finding the particle in state $\psi$ with energy $E_n$ when performing a measurement
on the system. We have that 
\begin{align}
	 \sum_{n=1}^{\infty} P_n = 1
\end{align}
as we should for probabilities. That is, the only possible values which we can measure for the energy is one of the $E_n$ 
associated with the stationary states, and the probability of obtaining each of the $E_n$'s is given by $P_n$.
We can us this to calculate the expectation value for the energy as 
\begin{align}
	\langle E \rangle = \sum_{n=1}^{\infty} |c_n|^2 E_n 
\end{align}
where we also need to accept that also the energy is generally not definitely defined in quantum mechanics.


\subsection{The Hamiltonian Operator}
The time-independent Schrödinger equation is often called the energy eigenvalue equation. An eigenvalue equation is 
an equation in which some operator, or matrix, acts on some function and that should equal that very function times a constant,
as in 
\begin{align}
	 A_{op}\psi_a = a \psi_a
\end{align}
where $a$ and $\psi_a$ are the eigenvalues and eigenfunctions to the operator $A_{op}$. Generally, an operator may have 
infinitely many eigenvalues and eigenfunctions. 

We now need to define what we mean by an operator. We have already encountered a couple of operators implicitly.
An operator is something which acts on something, meaning that it doesn't always necessarily have much significance on its own.
Firstly we have seen the position operator 
\begin{align}
	 x_{op} = x
\end{align}
which merely multiplies the wavefunction by $x$, and thus leading to a different function $x\psi(x)$. 

Furthermore, we have that the momentum operator is given by
\begin{align}
	 p_{op} = -i\hbar \frac{\partial}{\partial x} 
\end{align}
thus we have that
\begin{align}
	 p_{op} \psi(x) = -i\hbar \frac{\partial \psi(x)}{\partial x} 
\end{align}
Here we see that the operator on its own doesn't make too much physical sense on its own. 
We now consider the momentum eigenfunctions, and we recall that $p = h / \lambda$ which means that 
a wavefunction with a particular momentum is one with an particular wavelength $\lambda$. 
With this in mind we consider a wavefunction $A\exp(ikx)$ where $k = 2\pi / \lambda$. We consider 
now the momentum operator acting on this 
\begin{align}
	 p_{op} A e^{ikx} = -i\hbar \frac{\partial}{\partial x} A e^{ikx} = \hbar k e^{ikx}
\end{align}
from this we see that an eigenfunction of the momentum operator is $\exp(ikx)$ and the eigenvalue is $\hbar k$. 

We can also now use what we discovered about the position and momentum eigenfunctions and see why we call 
the time-independent Schrödinger equation an eigenvalue equation. We have in nonrelativistic 
quantum mechanics that the energy operator is the sum of the kinetic and potential energy operators 
\begin{align}
	 E_{op} = \frac{p_{op}^2}{2m} + V(x_{op})
\end{align}
Where we have the classical expression for the total energy and swap the classical variable $x$ and $p$ for the 
quantum operators for position $x_{op}$ and momentum $p_{op}$.
We now arrive at the quantum energy operator, which is called the Hamiltonian, and is given by 
\begin{align}
	 H \equiv E_{op} = \frac{1}{2m}\big(-i\hbar \frac{\partial}{\partial x} \big)\big(-i\hbar \frac{\partial}{\partial x} \big)
	+ V(x) = -\frac{\hbar^2}{2m} \frac{\partial^2}{\partial^2 x} + V(x)
\end{align}
Where we now see clearly that the Schrödinger equation is an energy eigenvalue equation 
\begin{align}
	 H \psi = E \psi 
\end{align}
These energy eigenvalues and eigenfunctions we have already found for the particle in the box, where we indexed 
the eigenvalues/functions with the subscripts $n$, we had 
\begin{align}
	 H \psi_n(x) = E_n \psi_n(x)
\end{align}
If we consider the particle in box wavefunction 
\begin{align}
	 \psi = c_1 \psi_1 + c_2 \psi_2
\end{align}
We argued for that $ \langle E \rangle = |c_1|^2 E_1 + |c_2|^2 E_2$ for this state.
We have that since the Hamiltonian $H$ is the energy operator we can write 
\begin{align}
	 H \psi = c_1 H\psi_1 + c_2 H\psi_2
\end{align}
where we have used that the Hamiltonian is linear. Therefore, we have 
\begin{align}
	 \int_{-\infty}^{\infty} \psi^* H \psi &= \int_{-\infty}^{\infty}  (c_1\psi_1 + c_2\psi_2)^*(c_1\psi_1 + c_2\psi_2) dx \\
	&= |c_1|^2 E_1 \int_{-\infty}^{\infty}  |\psi_1|^2 dx +  |c_2|^2 E_2 \int_{-\infty}^{\infty}  |\psi_2|^2 dx
	 +  c_1^* E_1c_2 \int_{-\infty}^{\infty}  \psi_1^*\psi_2 dx +  c_2^* E_2 c_1 \int_{-\infty}^{\infty}  \psi_2^*\psi dx \\
	&= |c_1|^2 E_1  +  |c_2|^2 E_2  
\end{align}
where we have exploited the orthonormality in the final line. So we have here in the last line the expectation value 
of the energy in the state $\psi$ and this holds also for a general number of terms in the linear combination for $\psi$. 
So we have 
\begin{align}
	 \langle E \rangle = \int_{-\infty}^{\infty} \psi^* H \psi \;dx
\end{align}
This is a general strategy for finding expectation values, which we shall come back to later, and we have also for 
the position operator
\begin{align}
	 \langle x \rangle = \int_{-\infty}^{\infty} \psi^* x \psi \;dx
\end{align}
which we already have seen. For the momentum we have 
\begin{align}
	 \langle p \rangle = \int_{-\infty}^{\infty} \psi^* \big(\frac{\hbar}{i}\frac{\partial }{\partial x} \big)  \psi \; dx
\end{align}







\end{document}

