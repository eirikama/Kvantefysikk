\documentclass{article}


\usepackage{pgfplots}
\pgfplotsset{compat=newest}
\usepackage{tikz}
\usetikzlibrary{decorations.markings}

\usepackage{subfig}
\usepackage[margin=2.5cm]{geometry}
\usepackage{amsmath}
\usepackage{amssymb}


\newcommand\greybox[1]{
	\vskip\baselineskip
	\par\noindent\colorbox{lightgray}{
		\begin{minipage}{\textwidth}#1\end{minipage}
	}
	\vskip\baselineskip
}

\title{Lecture Quantum Physics \\
Examples of One-Dimensional Potentials}


\begin{document}
\maketitle


\section{Dirac Delta Function Potential}
We start by considering the Dirac delta function, which can be used as a model potential.
We recall that the delta function is defined as
\begin{align}
	 \delta(x) = \lim_{b \to \infty} \sqrt{\frac{b}{\pi}} e^{-bx^2}
\end{align}
i.e. we can think of it as the limit of infinitely thin Gaussian curves.
It has been normalized such that
\begin{align}
	\int_{-\infty}^{\infty} \delta(x)dx = 1 
\end{align}
and it satisfies
\begin{align}
	 \delta(x) =
	\begin{cases}
		 0 \hspace{1cm} x \neq 0 \\
		 \infty \hspace{0.8cm} x = 0
	\end{cases}		 
\end{align}
The Dirac delta have the very useful property that they pick out one value from a
function with which it is multiplied and integrated over, i.e. we have
\begin{align}
	 \int_{\infty}^{\infty} f(x) \delta(x) dx = f(0)
\end{align}
or more generally
\begin{align}
	 \int_{\infty}^{\infty} f(x) \delta(x-a) dx = f(a)
\end{align}
where $f(x)$ can be any well-behaved function.

We now consider the delta function potential, which we define to be
\begin{align}
	 \frac{2m}{\hbar^2}V(x) = -\frac{\alpha}{a}\delta(x)
\end{align}
where $a$ has units of length and $\alpha$ is unitless and determines the strength of the potential.
We note that the Dirac delta function has units of $1/[L]$, which can be seen from the
normalization condition demanding that $\delta(x) dx$ is unit less, and knowing that $dx$ has units of length.
We have that since $\hbar$ has units of $[M] [L]^2 [T]^{-1}$. Knowing that energy has units of $[M][L]^2[T]^{-2}$
we get that $\frac{2m}{\hbar}V(x)$ has units of $1/[L]^2$.

We now consider the boundary conditions of the wave function There are in this case only two regions, namely
$x<0$ and $x>0$. The Schrödinger equation being a second-order differential equation we would like to demand
the wavefunction being continuous everywhere.
We consider the integral from infinitesimally above $x=0$ to infinitesimally above of both the wavefunction
and its first derivative. We start with the derivative
\begin{align}
	 \int_{0^-}^{0^+} \frac{d^2 \psi}{dx^2}
	&= \big(\frac{d\psi}{dx} \big)_{0^+} -  \big(\frac{d\psi}{dx} \big)_{0^-} \\
	&= \frac{2m}{\hbar^2} \int_{0^-}^{0^+} [V(x) - E]\psi(x) dx
\end{align}
using the Schrödinger equation. We use that the area of integration is infinitesimally
small and the intgrand is not infinite to obtain
\begin{align}
	 \frac{2mE}{\hbar^2} \int_{0^-}^{0^+} \psi(x)dx = 0
\end{align}
Furthermore, we have
\begin{align}
	 \frac{2m}{\hbar^2} \int_{0^-}^{0^+} V(x) \psi(x) dx =
	- \frac{\alpha}{a} \int_{0^-}^{0^+} \delta(x) \psi(x) dx = -\frac{\alpha}{a} \psi(0)
\end{align}
We therefore have
\begin{align}
(\frac{\partial \psi}{\partial x} )_{0^+} - (\frac{\partial \psi}{\partial x} )_{0^-} = -\frac{\alpha}{a}\psi(x)
\end{align}
where we see that the first derivative has a discontinuity at zero. We also see why the derivative is generally
continuous for non-singular potentials.

We now solve Schrödinger's equation for the Dirac delta potential. We have for $x\neq 0$ that
\begin{align}
	 \frac{d^2 \psi}{dx^2} = - \frac{2mE}{\hbar^2}\psi
\end{align}
We are searching bound states $(E<V_0)$ we have the solution
\begin{align}
	 \psi = Ae^{i\kappa x} + B e^{-i\kappa x}
\end{align}
where
\begin{align}
	 \kappa = \sqrt{\frac{-2mE}{\hbar^2}}
\end{align}
As for the square wells we demand the wavefunction to be continuous
which gives that $A=B$
\begin{align}
	 \psi(x) =
	\begin{cases}
		 Ae^{i\kappa x} \hspace{1cm} x\le 0 \\
		Ae^{-i\kappa x} \hspace{1cm} x \ge 0
	\end{cases}
\end{align}
We substitute the wavefunction into our condition for the derivatives' continuity and we find
\begin{align}
	 -2\kappa A = - \frac{\alpha}{a} A
\end{align}
or equivalently
\begin{align}
	 \sqrt{\frac{-2mE}{\hbar^2}} = \frac{\alpha}{2a}
\end{align}
This gives the energy of the bound state
\begin{align}
	 E = -\frac{\hbar^2\alpha^2}{8ma^2}
\end{align}
So we have found that the Dirac delta only take one bound state. This is very reasonable considering its
shape, and that there are no other shapes which we could conceive of which could be viable wavefunctions
for this systems, up to an overall phase. This is because we must demand the wavefunction being exponentially
dampened away from $x=0$.


	 
\section{Molecular Binding}

\subsection{Double Square Well}

Having seen that the finite square well can be a model system for a particle trapped by a simple atom
such as Hydrogen, we shall now run with this and consider a double finite square well, which we can
think of as modelling a simple molecule such as an Hydrogen molecule ion $H_2^+$. By studying this simple
toy system, we can gain insights into why molecules can exist.
We can see straight away that for bound states the wavefunctions must be oscillatory inside of the wells
and decrease exponentially away from the wells.

This is quite a bit more involved to solve compared to the square wells, since we have five different regions
and eight boundary conditions for the wavefunctions and its derivatives. However we can exploit the symmetry of the problem and note that the potential is an even function $V(-x) = V(x)$.
We also know that the ground state is even and knowing the shape of the solutions inside of the wells we
can sketch the shape of the ground state and the first excited state.

Classically one might expect that the electron would be trapped in the vicinity of either on or the other
protons, but here we see that the electron is equally likely to be at both protons, due to the symmetry of
potential and therefore also of the wavefunctions.
Furthermore, there is also a non-vanishing probability of finding the electron in the region between
the protons. It is this sharing of the electron between the wells which make the molecule stable by increasing
the binding, or actually by making it more energy efficient for the electron to be close to the protons.


\subsection{Delta Function Well}
We shall simplify the potential even further, and instead of the double square well consider the Dirac delta
function well. We define the following potential
\begin{align}
	 \frac{2m}{\hbar^2} V(x) = -\frac{\alpha}{a}[\delta(x-a) - \delta(x+a)]
\end{align}
That is, two delta functions which are separated by a distance $2a$.
The wavefunction must now satisfy
\begin{align}
	 \frac{d^2\psi}{dx^2} = -\frac{2mE}{\hbar^2}\psi
\end{align}
for the $(E<0)$ bound solutions.
We again define
\begin{align}
	 \kappa = \sqrt{\frac{-2mE}{\hbar^2}}
\end{align}
Since $E<0$ we have that $\kappa \in \mathbb{R}_+$. We have the exponential functions solving the
Schrödinger equations for $V(x) = 0$. We note that for the $E>0$ we would get complex exponentials
and oscillatory solutions.
We use that we must demand that the exponentials which diverge for
$|x|\to \infty$ must be discarded, which gives us
\begin{align}
	 \psi(x) =
	\begin{cases}
		 Ce^{\kappa x} \hspace{2cm} x<-a \\
		 A\cosh(\kappa x) \hspace{1.2cm} x \in (-a,a) \\
		 Ce^{-\kappa x} \hspace{1.9cm} x>a \\
	\end{cases}
\end{align}
where we have
\begin{align}
	 \cosh(\kappa x) = \frac{e^{\kappa x} + e^{-\kappa x}}{2}
\end{align}
We now apply boundary conditions. We only need to do it explicitly for $x=a$, since the system is symmetric.
Continuity of the wavefunction gives
\begin{align}
	 A\cosh(\kappa x) = C e^{-\kappa x}
\end{align}
For the first derivative we once again have a discontinuity
\begin{align}
(\frac{\partial \psi}{\partial x} )_{0^+} - (\frac{\partial \psi}{\partial x} )_{0^-}
					= -\frac{\alpha}{a}\psi(x)
\end{align}
We insert the wavefunctions and get
\begin{align}
	 -\kappa C e^{-\kappa x} - \kappa A \sinh(\kappa x) = -\frac{\alpha}{a}A\cosh(\kappa x)
\end{align}
where
\begin{align}
	 \sinh(\kappa x) = \frac{e^{\kappa x} - e^{-\kappa x}}{2}
\end{align}
Substituting the continuity equation for the wavefunction itself into the condition for the discontinuity of
the derivative and dividing by $\cosh(\kappa x)$ we obtain
\begin{align}
	 \tanh(\kappa x) = \frac{\alpha}{\kappa a} - 1
\end{align}
where
\begin{align}
	 tanh(\kappa x) = \frac{sinh(\kappa x)}{\cosh(\kappa x)}
\end{align}
We have now found the equation determining the allowed energies, and we see that it is an transcendental
equation which must be solved graphically in a manner similar to for the finite square well, where we must
determine where the curve $\frac{\alpha}{\kappa a} - 1$ and $\tanh(\kappa x)$ intersect. Since we know that
the tangent hyperbolicus functions is always less than one we can say with certainty the crossing will occur
somewhere such that
\begin{align}
	 \frac{\alpha}{\kappa a} - 1 < 1
\end{align}
Inserting from the definition of $\kappa$ we now find
\begin{align}
	 E < - \frac{\hbar^2\alpha^2}{8ma^2}
\end{align}
We now compare this with the single Dirac delta potential and we see that the double delta potential well
has lower energy. This means that it is energetically preferable for system to organize into a molecule than
single atoms.

It is not guaranteed that the double wells such as delta wells and the double square wells have excited states.
We stated previously that the double well will have an excited state, but for the delta well it will depend
on the strength of the potential $\alpha$ whether or not there are excited states. The hydrogen molecule does
not have any excited states.



\section{Scattering}
Until now we have mostly considered potentials which have, at least some, bound states which
have discrete energy spectra. We shall now begin studying one-particle systems for which the Schrödinger
equation takes solutions fir which the particle is not confined to a given region of space.
For such systems there exists a continuum of allowed energies, since there are no constraining condition
on the energies. That is, the energies are no longer quantized. We generate physically acceptable solutions
in such systems by superposition of several solutions to form wave packets, which will be normalizable.
These solutions will also be dynamic, i.e. be evolving in time, since they consists of superpositions of energy
eigenstates. \\

We consider as a start the step potential
\begin{align}
	 V(x) =
	\begin{cases}
		 0 \hspace{1cm} x < 0 \\
		V_0 \hspace{0.9cm} x > 0
	\end{cases}
\end{align}
Initially we start by assuming the energy is larger than the potential, i.e. $E > V_0$.
We then have to the left of the step barrier
\begin{align}
	 \frac{d^2 \psi}{dx^2} = - \frac{2mE}{\hbar^2}\psi \hspace{1cm} x < 0
\end{align}
For notational simplicity we define
\begin{align}
	 k = \sqrt{\frac{2mE}{\hbar^2}}
\end{align}
and we then have the solutions on the form
\begin{align}
	 \psi(x) = Ae^{ikx} + B e^{-ikx} \hspace{1cm} x<0
\end{align}
Here we note the importance of using complex exponentials instead of trigonometric functions, since we can then
associate them with incoming and outgoing waves.

We now consider the right side of the potential barrier, where we have the Schrödinger equation
\begin{align}
	 \frac{d^2 \psi}{dx^2} = - \sqrt{\frac{2m(E-V_0)}{\hbar^2}}  \hspace{1cm} x>0
\end{align}
where we define
\begin{align}
	 k_0 = \sqrt{\frac{2m(E-V_0)}{\hbar^2}} = \sqrt{k^2 - \frac{2mV_0}{\hbar^2}}
\end{align}
Since we know $E>V_0$ we have the solutions
\begin{align}
	 \psi(x) = Ce^{ik_0x} + D e^{-ik_0x} \hspace{1cm} x>0
\end{align}
We now use the requirement that the wavefunction must be continuous and have continuous first derivatives.
We consider the potential problematic point at $x=0$. Continuity of the wavefunction gives
\begin{align}
	 A + B = C+D
\end{align}
and the first derivatives
\begin{align}
	 ik(A-B) = ik_0(C-D)
\end{align}
We now see that we have only two equations for four unknowns. Which means we are free to chose some of the
constants as we please, or on the basis of physical considerations, since there are many configurations
of constants which satisfy the equations.
We therefore choose $D=0$. Which gives
\begin{align}
	 \psi(x) = C e^{ik_0x} \hspace{1cm} x>0
\end{align}
We may now easily solve the remaining equations by expressing $B$ and $C$ in terms of $A$
\begin{align}
	 B &= \frac{k-k_0}{k+k_0}A \\
	 C &=  \frac{2k}{k+k_0}A
\end{align}
Where we note that this solution works for any value of $k$, i.e. for any energy. Thus the energy is continuous
as there are no constrains which compels it to be quantized.
In this case we cannot determine the constant $A$ from normalization conditions but since the wavefunction
doesn't go to zero as $x$ goes to infinity, it is not normalizable.
We will therefore consider the probability current instead, which describes the probability of reflection and
transmission and is given as
\begin{align}
		j_x =	\frac{\hbar}{2mi} \big( \psi^* \frac{\partial \psi}{\partial x}
																		-\psi \frac{\partial \psi^*}{\partial x}    \big)
\end{align}
We have the wavefunction
\begin{align}
	 \psi(x) =
	\begin{cases}
		 Ae^{ikx} + Be^{-ikx} \hspace{1cm} x<0 \\
		 Ce^{ik_0x} \hspace{2.3cm} x>0 \\
	\end{cases}
\end{align}
Which gives the probability current
\begin{align}
	 j_x =
	\begin{cases}
		 \frac{\hbar k}{m}(|A|^2 - |B|^2) \hspace{1cm} x<0 \\
		 \frac{\hbar k_0}{m}|C|^2  \hspace{2.2cm} x>0 \\
	\end{cases}
\end{align}
Here the positive term represent current in the positive $x$-direction and the negative in the
direction towards negative values of $x$. Each term is equal to the probability density multiplied
by the velocity of propagation $\hbar k / m$ left of the origin and $\hbar k_0 / m$ right of the origin.
This is also why we chose to set $D=0$, as such a term would imply a probability current coming in from the
right. This is unphysical when we consider a scattering experiment with particles coming in from the left.
We now define the reflection coefficient $R$, the probability of reflection, as the ratio of the
magnitude of reflected and incoming current
\begin{align}
	 		R = \frac{j_{ref}}{j_{inc}} = \frac{\frac{\hbar k}{m}|B|^2}{\frac{\hbar k}{m}|A|^2} = \frac{|B|^2}{|A|^2}
\end{align}
Similarly, we define the transmission coefficient
\begin{align}
	 T = \frac{j_{trans}}{j_{inc}} = \frac{\frac{\hbar k_0}{m}|B|^2}{\frac{\hbar k}{m}|A|^2}
	= \frac{k_0|B|^2}{k|A|^2}
\end{align}
Where we note that the transmission coefficient is taking into account that the particle moves with a different
velocity for $x>0$ and this is why the transmission coefficient is not just the ratio of $|B|^2$ to $|A|^2$.
We substitute for the expressions we found previously for $A$ and $B$ and find
\begin{align}
	 R = \frac{(k - k_0)^2}{(k + k_0)^2}
\end{align}
and
\begin{align}
	 T = \frac{4kk_0}{(k+k_0)^2}
\end{align}
Where we can find comfort in conforming that
\begin{align}
	 R + T = 1
\end{align}
These result seem strange from a classical point of view, since a particle approaching the potential step
would slow down without being reflected. However, from a wave mechanics-point of view this makes sense,
knowing that a wave propagating on a rope would partly be reflected and partly transmitted if there were
a discontinuity in density and thus in the speed of reflection.

We stated previously that a way of generating physically viable solutions is to superimpose solutions
to Schrödinger's equation to a normalizable solution, just as we've already seen for the free particle.
For $E>V_0$ this superposition can be written as
\begin{align}
	 \psi(x, t) =
	\begin{cases}
		 \int_{-\infty}^{\infty} \big[A(k)e^{i(kx-\frac{Et}{\hbar})}
		+ B(k)e^{-i(kx + \frac{Et}{\hbar})} \big] \hspace{1cm} x<0 \\
		 \int_{-\infty}^{\infty} C(k)e^{i(k_0x - \frac{Et}{\hbar})} dx \hspace{3.6cm} x>0
	\end{cases}
\end{align}
Where each of the terms signify a wave packet travelling in a given direction in both of the regions. We once
again see why we set $D=0$ even though the equations still would hold for $D\new 0$, but on physical grounds
we don't want a wave travelling towards the origin from the right, as this doesn't make sense in the scattering
scenario we are considering. \\

We shall now consider the case where the energy is less than that of the potential step, i.e. $E<V_0$,
where we have, as before, the same general form of the elementary solutions
\begin{align}
	 \psi(x) =
	\begin{cases}
		 Ae^{ikx} + Be^{-ikx} \hspace{1cm} x<0 \\
		 Ce^{\kappa x} + De^{-\kappa x} \hspace{1cm} x>0
	\end{cases}
\end{align}
where we define
\begin{align}
	 \kappa = \sqrt{\frac{2m(V_0-E)}{\hbar^2}}
\end{align}
Again we set $D=0$, since we want a normalizable function and the corresponding term diverge as
$|x| \to \infty$. Continuity of the wavefunction and its derivatives now yield
\begin{align}
	 A + B = C
\end{align}
and
\begin{align}
	 ik (A-B) = -\kappa  C
\end{align}
We rearrange to get
\begin{align}
	 \frac{B}{A} = \frac{k - i\kappa}{k + i\kappa}
\end{align}
We then get
\begin{align}
	 R \equiv \frac{|B|^2}{|A|^2} = \frac{B^* B}{A^*A}
	= \big(\frac{k - i\kappa}{k + i\kappa}\big)\big(\frac{k + i\kappa}{k - i\kappa}\big) = 1
\end{align}
Thus 100\% of the incident wave is reflected. This might seem counter-intuitive, since the we have a non-zero
wavefunction to the right of the step. This can however be seen to hold form the probability current where we
have
\begin{align}
	 j_x =
	\begin{cases}
		 \frac{\hbar k}{m}(|A|^2 - |B|^2) \hspace{1cm}	x<0 \\
		 0 \hspace{3.4cm} x>0
	\end{cases}
\end{align}
So we have the transmission coefficient vanishing in this expression, i.e. $T=0$ and we thus also have
the probability conserved as $R+T=1$ as we want.


\section{Tunnelling}
We shall no consider the case where a particle with energy $E \in (0, V_0)$ is incident on an energy barrier
defined as
\begin{align}
	 V(x) =
	\begin{cases}
		 0 \hspace{1cm} x<0 \\
		 V_0 \hspace{0.9cm} x \in (0, a) \\
  	 0 \hspace{1cm} x>0
	\end{cases}
\end{align}
We thus consider the case where the potential barrier is larger than the energy, which in classical mechanics
would imply that the transmission probability would be zero. In quantum mechanics we have a non-zero
transmission probability and we call this phenomena tunnelling.
For a particle incident on the barrier from the left the wavefunction is
\begin{align}
	 \psi(x) =
	\begin{cases}
		 Ae^{ikx} + Be^{-ikx} \hspace{1cm} x < 0 \\
		 Fe^{\kappa x} + Ge^{-\kappa x} \hspace{1cm} x \in (0, a) \\
		 Ce^{i k x}  \hspace{2.5cm} x > a \\
	\end{cases}
\end{align}
where we once again defined
\begin{align}
	 \kappa = \frac{\sqrt{2m(V_0 - E)}}{\hbar}
\end{align}
and
\begin{align}
	 k = \frac{\sqrt{2mE}}{\hbar}
\end{align}
Boundary conditions from the continuity of the wavefunction gives
\begin{align}
	 A + B = F + G \hspace{1.7cm} \text{at }  x = 0
\end{align}
and
\begin{align}
	 Fe^{\kappa a} + Ge^{-\kappa a} = Ce^{ika} \hspace{ 0.8cm } \text{at } x=a
\end{align}
Continuity of the first derivative gives
\begin{align}
	 ik(A-B) = \kappa(F-G) \hspace{1.2cm} \text{at } x=0
\end{align}
and
\begin{align}
	 \kappa (Fe^{\kappa a} - Ge^{-\kappa a}) = ikCe^{ika} \hspace{0.8cm} \text{at } x=a
\end{align}
We now determine the transmission coefficient, which we know is defined as $T = |A|^2/ |C|^2$.
Since we have that
\begin{align}
	 T = \frac{j_{trans}}{j_{inc}} = \frac{|C|^2}{|A|^2}
\end{align}
We can use our four equations to determine the four unknowns.
We start by eliminating $B$ from the first and the third and we get
\begin{align}
	 2ikA = (ik-\kappa)F + (ik+\kappa)G
\end{align}
and eliminating $F$ from the second and forth gives
\begin{align}
	 2\kappa FGe^{\kappa a} = (\kappa + ik)Ce^{ik a}
\end{align}
and $G$ from the second and forth equations
\begin{align}
	 2\kappa Fe^{-\kappa a} = (\kappa -ik)Ce^{ika}
\end{align}

Substituting the expressions for $F$ and $G$ in the two latter subsidiary equations into the first one gives
\begin{align}
	 2ikA = (ik - \kappa)\big[ \frac{\kappa -i k}{2\kappa}Ce^{ika + \kappa a} \big]
						+ (ik + \kappa) \big[\frac{\kappa + ik}{2\kappa}Ce^{ika - \kappa a}\big]
\end{align}
This is equivalent to
\begin{align}
	 \frac{A}{C} &= \frac{e^{ika}}{4ik\kappa}\big[(-\kappa^2 + k^2 + 2ik\kappa)e^{\kappa a}
										+ (\kappa^2 - k^2 + 2ik\kappa)e^{-\kappa a}] \\
							 &= \frac{e^{ika}}{2ik\kappa}\big[(k^2 - \kappa^2)\sinh(\kappa a) + 2ik\kappa \cosh(\kappa a)]
\end{align}
Which then gives
\begin{align}
	 \frac{|A|^2}{|C|^2} &= \frac{1}{4k^2\kappa^2}\big[(k^2 - \kappa^2)^2\sinh(\kappa a)^2
																									+ 4k^2\kappa^2\cosh^2(\kappa a) \big] \\
											 &= 1 + \frac{(k^2+\kappa^2)^2}{4k^2\kappa^2}\sinh^2(\kappa a)
\end{align}
where we have used that
\begin{align}
	 \cosh^2(x) - \sinh^2(x) = 1
\end{align}
This gives us
\begin{align}
	 T = \big[1 + \frac{(k^2+\kappa^2)^2}{4k^2\kappa^2}\sinh^2(\kappa a)\big]^{-1}
\end{align}
Tunnelling is a very commonly occurring thing on a microscopic scale so this tunnelling transmission probability
is often very large for e.g. electrons, but for macroscopic objects the transmission probability is minuscule.

For a thick barrier the expression can be simplified slightly. If $a$ is very small compared to the energy of the
particle, such that $\kappa a \gg 1$. In this case we have
\begin{align}
	 \sinh(\kappa a) = \frac{e^{\kappa a} - e^{-\kappa a}}{2} \approx \frac{1}{2}e^{\kappa a} \gg 1
\end{align}
and therefore we have
\begin{align}
	 T = \big(\frac{4k\kappa}{k^2 + \kappa^2}\big)^2 e^{-\kappa a}
\end{align}
We see here that the transmission is very sensitive to the thickness of the barrier $a$ and the value of $\kappa$
which depends on $V_0-E$, since there is an exponential dependence on it.

If e.g. for a certain value of $E$ and $V_0$ and with $a=1nm$ we have $e^{-2\kappa a}= 10^{-10}$, which means that
there may certainly be some tunnelling since there are so many electrons per cubic centimeters (typically $10^{22}$).
However, if we were to have $a=5nm$ instead of the one nanometer barrier we would have $T=10^{-50}$, which  would
mean that no electrons will be able to tunnel.

This is how Scanning Tunnelling Microscopes work. They exploit that there is a $2\%$ change in current through the
tip of the microscope probe when the distance changes by $0.001nm$. The tip is then dragged over the surface to the
imaged and traces out the contours of the surface and thereby creates a sort of image of it.

\subsection{Field Emission of Electrons: Tunnelling through a Non-Square Barrier}
For $\kappa a \gg 1$
\begin{align}
	 \ln(T) &\approx \ln\big( (\frac{4k\kappa}{k^2 + \kappa^2})^2 \big) - 2\kappa a \\
					&\approx C - 2\kappa a
\end{align}
where $C$ is some constant.
Where we have used that in the last step the logarithmic term will not change much compared to the one which
is linear in energy.
If the barrier varies slowly with position compared with $\kappa^{-1}$ we can approximate it as a
series of square barriers, which are thick enough for us to make the "thick barrier"-approximation.
We write the approximation as
\begin{align}
	 \ln(T) \approx C -2 \int \sqrt{\frac{2m[V(x) - E]}{\hbar^2}} dx
\end{align}
for the whole barrier, where the limits of integration includes the region where $V(x) > E$. This should be
fairly accurate close to the classical turning points, where $V(x) \approx E$ and thus $\kappa$ is roughly zero.

An application of this is field emission of electrons. We have already discussed the photoelectric effect, where we had
that the energy needed for ejecting the most energetic electrons is the work function $W$.
Another way of liberating these electrons is by changing the shape of the potential well and thereby allow the electrons to tunnel out.
If some negative charge is put on the metal the surface near the metal will have a constant electric field of magnitude
$\mathcal{E}$ which is proportional to the surface charge density.
Since the force $e\mathcal{E}$ is presumed to be constant the potential energy in the vicinity of the surface is
$W - e\mathcal{E}x$, where $x$ is the perpendicular distance from the surface and $W$ is the potential energy at the
surface. The transmission is then roughly given by
\begin{align}
	 T = C e^{-2\sqrt{\frac{2m}{\hbar^2}} \int_{0}^{L} \sqrt{W - e\mathcal{E}x}dx }
\end{align}
where $C$ is a constant and $L = W / e\mathcal{E}$ is the distance the most energetic electrons must tunnel before
reaching a region outside the metal where $E>V$. Performing the integral gives
\begin{align}
	 T = C e^{-\frac{4\sqrt{2m} }{3\hbar e\mathcal{E}}W^{\frac{3}{2}}}
\end{align}
This is called the Fowler-Nordheim relation. It was one of the first predictions of quantum mechanics.

\end{document}
